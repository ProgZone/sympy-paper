% A description of some of the algorithms in SymPy. The list is not
% exhaustive.

% The sections here are preliminary. We may end up needing to cut some of
% this.

% XXX: Perhaps this should just be integrated into the features section.

\subsection{Numerics}

The \texttt{Float} class holds an arbitrary-precision binary floating-point value
and a precision in bits. An operation between two \texttt{Float}
inputs is rounded to the larger of the two precisions.

The preferred way to evaluate an expression numerically is with the
\texttt{evalf} method, which internally estimates the number of accurate
bits of the floating-point
approximation for each sub-expression, and adaptively increases the
working precision until the estimated accuracy of the
final result matches the sought number of decimal digits.

The internal error tracking does not provide rigorous error bounds
(in the sense of interval arithmetic) and cannot be used to track
uncertainty in measurement data in any meaningful way;
the sole purpose is to mitigate loss of accuracy that typically occurs
when converting symbolic expressions to numerical values, for example
due to catastrophic cancellation. This is illustrated by the following
example (the input 25 specifies that 25 digits are sought):

\begin{verbatim}
>>> cos(exp(-100)).evalf(25) - 1
0
>>> (cos(exp(-100)) - 1).evalf(25)
-6.919482633683687653243407e-88
\end{verbatim}

The \texttt{evalf} method works with complex numbers and supports
more complicated expressions, such as
special functions, infinite series and integrals.

SymPy does not track the accuracy of
approximate numbers outside of \texttt{evalf}.
The familiar dangers of floating-point arithmetic apply, and 
symbolic expressions containing floating-point numbers should be treated
with some caution.
This approach is similar to Maple and Maxima.

By contrast, Mathematica uses a form
of significance arithmetic~\cite{Sofroniou2005precise} for approximate numbers.
This offers further protection against numerical errors,
but leads to non-obvious semantics while
still not being mathematically rigorous
(for a critique of significance arithmetic, see Fateman~\cite{Fateman1992}).
SymPy's \texttt{evalf} internals are non-rigorous in the same sense,
but have no bearing on the semantics of floating-point
numbers in the rest of the system.

\subsubsection{Code generation}

SymPy's \texttt{lambdify} can be used to convert a symbolic expression to a
callable Python function for faster numerical evaluation.
Various back ends are supported. The following example
demonstrates creating a NumPy-based function from
a SymPy expression, which automatically supports
vectorized array evaluation [NumPy citation?]:

\begin{verbatim}
>>> f = lambdify((x, y), sin(x*y)**2, modules='numpy')
>>> from numpy import array
>>> f(array([1,2,3]), array([4,5,6]))
array([ 0.57275002,  0.29595897,  0.56398184])
\end{verbatim}

SymPy can also generate C, C++, Fortran77, Fortran90 and
Octave/Matlab source code, via the \texttt{codegen} function. [document this?]

\subsubsection{The mpmath library}

The implementation of arbitrary-precision floating-point arithmetic
is supplied by the mpmath library, which originally was developed
as a SymPy module but subsequently has been
moved to a standalone Python package. The basic datatypes in mpmath
are \texttt{mpf} and \texttt{mpc}, which respectively act as
multiprecision substitutes for Python's \texttt{float} and \texttt{complex}.
The floating-point precision is controlled by a global context:

\begin{verbatim}
>>> import mpmath
>>> mpmath.mp.dps = 30    # 30 digits of precision
>>> mpmath.mpf("0.1") + mpmath.exp(-50)
mpf('0.100000000000000000000192874984794')
>>> print(_)   # pretty-printed
0.100000000000000000000192874985
\end{verbatim}

For pure numerical computing, it is convenient to use mpmath directly
with \texttt{from mpmath import *} (it is best to avoid such an
import statement when using SymPy simultaneously, since numerical
functions such as \texttt{exp} will shadow the symbolic counterparts
in SymPy).

Like SymPy, mpmath is a pure Python library.
Internally, mpmath represents a floating-point number
$(-1)^s x \cdot 2^y$ by a tuple $(s, x, y, b)$ where
$x$ and $y$ are arbitrary-size Python integers
and the redundant integer $b$ stores the bit length of $x$ for quick access.
If GMPY [citation?] is installed, mpmath automatically switches to
using the \texttt{gmpy.mpz} type for $x$ and using GMPY helper methods
to perform rounding-related operations, improving performance.

The mpmath library includes support for
special functions, root-finding, linear algebra, polynomial approximation,
and numerical computation of limits, derivatives, integrals, infinite
series, and ODE solutions. All features work in arbitrary precision
and use algorithms that support computing hundreds of digits rapidly,
except in degenerate cases.

The double exponential (tanh-sinh) quadrature is used for numerical
integration by default. For smooth integrands, this algorithm usually
converges extremely rapidly, even when the integration interval is infinite
or singularities are present at the endpoints~\cite{takahasi1974double,bailey2005comparison}.
However, for good performance, singularities
in the middle of the interval must be specified
by the user.
To evaluate slowly converging limits and infinite series, mpmath
automatically attempts to apply Richardson extrapolation and the
Shanks transformation
(Euler-Maclaurin summation can also be used)~\cite{BenderOrszag1999}.
A function to evaluate oscillatory integrals by means of convergence
acceleration is also available.

A wide array of higher mathematical functions are implemented
with full support for complex values of all parameters and arguments,
including complete and incomplete gamma functions,
Bessel functions, orthogonal polynomials, elliptic functions and integrals,
zeta and polylogarithm functions,
the generalized hypergeometric function, and the Meijer G-function.

Most special functions are implemented as linear 
combinations of the generalized hypergeometric function ${}_pF_q$,
which is computed by a combination of direct summation,
argument transformations (for ${}_2F_1$, ${}_3F_2$, $\ldots$)
and asymptotic expansions
(for ${}_0F_1$, ${}_1F_1$, ${}_1F_2$, ${}_2F_2$, ${}_2F_3$)
to cover the whole complex domain.
Numerical integration and generic convergence acceleration
are also used in a few special cases.

In general, linear combinations and argument transformations
give rise to singularities that have to be removed for certain
combinations of parameters.
A typical example is the modified Bessel function of the second kind
$$K_{\nu}(z) = \frac{1}{2} \left[
            \left(\frac{z}{2}\right)^{-\nu}
                \Gamma(\nu)
                {}_0F_1\!\left(1-\nu, \frac{z^2}{4}\right)
             -
             \left(\frac{z}{2}\right)^{\nu}
                 \frac{\pi}{\nu \sin(\pi \nu) \Gamma(\nu)}
                 {}_0F_1\!\left(\nu+1, \frac{z^2}{4}\right)
            \right]$$
where the limiting value $\lim_{\varepsilon \to 0} K_{n+\varepsilon}(z)$
has to be computed when $\nu = n$ is an integer.
A generic algorithm is used to evaluate
hypergeometric-type linear combinations of the above type.
This algorithm automatically detects cancellation problems,
and computes limits numerically by perturbing parameters whenever
internal singularities occur (the perturbation size is automatically
decreased until the result is detected to converge numerically).

Due to this generic approach, particular combinations of hypergeometric
functions can be specified easily.
The implementation of the Meijer G-function takes only a few dozen lines of
code, yet covers the whole input domain in a robust way.
The Meijer G-function instance
$G_{1, 3}^{3, 0}\left(0 ; \tfrac{1}{2}, -1, - \tfrac{3}{2} | x \right)$
is a good test case~\cite{Toth2007}; past versions of both Maple and
Mathematica produced incorrect numerical values for large $x > 0$.
Here, mpmath automatically removes the internal singularity
and compensates for cancellations (amounting to 656 bits
of precision when $x = 10000$), giving correct values:
\begin{verbatim}
>>> mpmath.mp.dps = 15
>>> mpmath.meijerg([[],[0]],[[-0.5,-1,-1.5],[]],10000)
mpf('2.4392576907199564e-94')
\end{verbatim}

Equivalently, with SymPy's interface this function can be evaluated as:
\begin{verbatim}
>>> meijerg([[],[0]],[[-S(1)/2,-1,-S(3)/2],[]],10000).evalf()
2.43925769071996e-94
\end{verbatim}

We highlight the generalized hypergeometric functions and
the Meijer G-function, due to those functions' frequent appearance
in closed forms for integrals and sums [todo: crossref symbolic integration].
Via mpmath, SymPy has relatively good support for evaluating sums and integrals
numerically, using two complementary approaches: direct numerical evaluation,
or first computing a symbolic closed form involving special functions. [example?]

\subsubsection{Numerical simplification}

The \texttt{nsimplify} function in SymPy
(a wrapper of \texttt{identify} in mpmath)
attempts to find a simple symbolic
expression that evaluates to the same numerical value as the given
input.
It works by applying a few simple transformations
(including square roots, reciprocals, logarithms and exponentials) to
the input and, for each transformed value,
using the PSLQ algorithm~\cite{Ferguson1999} to search for
a matching algebraic number or optionally a linear combination
of user-provided base constants (such as $\pi$).

\begin{verbatim}
>>> x = 1 / (sin(pi/5)+sin(2*pi/5)+sin(3*pi/5)+sin(4*pi/5))**2
>>> nsimplify(x)
-2*sqrt(5)/5 + 1
>>> nsimplify(pi, tolerance=0.01)
22/7
>>> nsimplify(1.783919626661888, [pi], tolerance=1e-12)
pi/(-1/3 + 2*pi/3)
\end{verbatim}

\subsection{Polynomials}

\subsection{The Risch Algorithm}
% Also the Meijer-G algorithm, if someone can write about it

\subsection{The Gruntz Algorithm}

The limit module implements the Gruntz algorithm
\cite{gruntz1996computing}.

Examples:
\begin{verbatim}
In [1]: limit(sin(x)/x, x, 0)
Out[1]: 1

In [2]: limit((2*E**((1-cos(x))/sin(x))-1)**(sinh(x)/atan(x)**2), x, 0)
Out[2]: E
\end{verbatim}

\subsubsection{Details}

We first define comparability classes by calculating $L$:
\begin{equation}
L\equiv \lim_{x\to\infty} {\log |f(x)| \over \log |g(x)|}
\end{equation}
And then we define the $<$, $>$ and $\sim$ operations as follows: $f>g$ when
$L=\pm\infty$ ($f$ is more rapidly varying than $g$, i.e. $f$ goes to $\infty$
or $0$ faster than $g$, $f$ is greater than any power of $g$), $f<g$ when $L=0$
($f$ is less rapidly varying than $g$) and $f\sim g$ when $L\neq 0,\pm\infty$
(both $f$ and $g$ are bounded from above and below by suitable integral powers
of the other).

Examples:
$$2 < x < e^x < e^{x^2} < e^{e^x}$$
$$2\sim 3\sim -5$$
$$x\sim x^2\sim x^3\sim {1\over x}\sim x^m\sim -x$$
$$e^x\sim e^{-x}\sim e^{2x}\sim e^{x+e^{-x}}$$
$$f(x)\sim{1\over f(x)}$$

The Gruntz algorithm, on an example:
$$f(x) = e^{x+2e^{-x}} - e^x + {1\over x}$$
$$\lim_{x\to\infty} f(x) = ?$$

Strategy:
mrv set: the set of most rapidly varying subexpressions
$\{e^x, e^{-x}, e^{x+2e^{-x}}\}$, the same comparability class
Take an item $\omega$ from mrv, converging to 0 at infinity. Here
$\omega=e^{-x}$. If not present in the mrv set, use the relation
$f(x)\sim {1\over f(x)}$.

Rewrite the mrv set using $\omega$: $\{{1\over\omega}, \omega,
{1\over\omega}e^{2\omega}\}$, substitute back into $f(x)$ and expand in
$\omega$:
$$f(x) = {1\over x}-{1\over\omega}+{1\over\omega}e^{2\omega}
    = 2+{1\over x} + 2\omega + O(\omega^2)$$

The core idea of the algorithm: $\omega$ is from the mrv set, so in the limit
$\omega\to0$:
$$f(x) = {1\over x}-{1\over\omega}+{1\over\omega}e^{2\omega}
    = 2+{1\over x} + 2\omega + O(\omega^2)
    \to 2 + {1\over x}$$

We iterate until we get just a number, the final limit. Gruntz proved this
algorithm always works and converges in his Ph.D. thesis
\cite{gruntz1996computing}.

Generally:
$$ f(x) = \underbrace{O\left({1\over \omega^3}\right)}_\infty
    + \underbrace{C_{-2}(x)\over \omega^2}_\infty
    + \underbrace{C_{-1}(x)\over \omega}_\infty
    + {C_{0}(x)}
    + \underbrace{C_{1}(x)\omega}_0
    + \underbrace{O(\omega^2)}_0
$$
we look at the lowest power of $\omega$. The limit is one of: $0$,
$\lim_{x\to\infty} C_0(x)$, $\infty$.

\subsection{Logic}

\subsection{Other}
