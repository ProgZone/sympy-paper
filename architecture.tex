
%% I volunteer to write this section. --Aaron
%%
%% Representing symbolic expressions using Python objects

\subsection{The Core}

The core of a computer algebra system (CAS) refers to the module who is in
charge of resenting symbolic expressions and performing basic manipulations
with them. In SymPy, every symbolic expression is an instance of a Python class.
Expressions are represented by expression trees. The operators are represented
by the type of an expression and the child nodes are stored in the
\texttt{args} attribute. A leaf node in the expression tree has an empty
\texttt{args}.
The \texttt{args} attribute is provided by the class \texttt{Basic},
which is a superclass of all SymPy objects and
provides common methods to all SymPy tree-elements.
For example, consider the expression $xy + 2$:

\begin{verbatim}
>>> from sympy import *
>>> x, y = symbols('x y')
>>> expr = x*y + 2
\end{verbatim}

The expression \texttt{expr} is an addition, so it is of type \texttt{Add}. The child
nodes of \texttt{expr} are \texttt{x*y} and \texttt{2}.

\begin{verbatim}
>>> type(expr)
<class 'sympy.core.add.Add'>
>>> expr.args
(2, x*y)
\end{verbatim}

We can dig further into the expression tree to see the full expression. For
example, the first child node, given by \texttt{expr.args[0]} is
\texttt{2}. Its class is \texttt{Integer}, and it has empty \texttt{args},
indicating that it is a leaf node.

\begin{verbatim}
>>> expr.args[0]
2
>>> type(expr.args[0])
<class 'sympy.core.numbers.Integer'>
>>> expr.args[0].args
()
\end{verbatim}

The function \texttt{srepr} gives a string representing a valid Python code,
containing all the nested class constructor calls to create the given expression.

\begin{verbatim}
>>> srepr(expr)
"Add(Mul(Symbol('x'), Symbol('y')), Integer(2))"
\end{verbatim}

Every SymPy expression satisfies a key invariant, namely,
\texttt{expr.func(*expr.args) == expr}. This implies that expressions are
rebuildable from their \texttt{args}~\footnote{\texttt{expr.func} is used
  instead of \texttt{type(expr)} to allow the function of an expression to be
  distinct from its actual Python class. In most cases the two are the same.}.
Here, we note that in SymPy, the \texttt{==} operator represents exact
structural equality, not mathematical equality. This allows one to test if any
two expressions are equal to one another as expression trees.

Python allows classes to overload operators. The Python interpreter translates
the above \texttt{x*y + 2} to, roughly,
\verb|(x.__mul__(y)).__add__(2)|. \texttt{x} and \texttt{y}, returned from
the \texttt{symbols} function, are \texttt{Symbol} instances. The \texttt{2}
in the expression is processed by Python as a literal, and is stored as
Python's builtin \texttt{int} type. When \texttt{2} is called by the
\verb|__add__| method, it is converted to the SymPy type \texttt{Integer(2)}. In
this way, SymPy expressions can be built in the natural way using Python
operators and numeric literals.

One must be careful in one particular instance. Python does not have a builtin
rational literal type. Given a fraction of integers such as \texttt{1/2},
Python will perform floating point division and produce
\texttt{0.5}~\footnote{This is the behavior in Python 3. In Python 2,
  \texttt{1/2} will perform integer division and produce \texttt{0}, unless
  one uses \texttt{from \_\_future\_\_ import division}.}. Python uses eager
evaluation, so expressions like \texttt{x + 1/2} will produce \texttt{x +
  0.5}, and by the time any SymPy function sees the \texttt{1/2} it has
already been converted to \texttt{0.5} by Python. However, for a CAS like
SymPy, one typically wants to work with exact rational numbers whenever
possible. Working around this is simple, however:  one can wrap one of the
integers with \texttt{Integer}, like \texttt{x + Integer(1)/2}, or using
\texttt{x + Rational(1, 2)}. SymPy provides a function \texttt{S} which can be
used to convert objects to SymPy types with minimal typing, such as \texttt{x
  + S(1)/2}. This gotcha is a small downside to using Python directly instead
of a custom domain specific language (DSL), and we consider it to be worth it
for the advantages listed above.

%%
%% Assumptions
\subsection{Assumptions}

An important feature of the SymPy core is the assumptions system. The
assumptions system allows users to specify that symbols have certain common
mathematical properties, such as being positive, imaginary, or integer. SymPy
is careful to never perform simplifications on an expression unless the
assumptions allow them. For instance, the identity $\sqrt{x^2} = x$ holds if
$x$ is nonnegative ($x\ge 0$). If $x$ is real, the identity $\sqrt{x^2}=|x|$
holds. However, for general complex $x$, no such identity holds.

By default, SymPy performs all calculations assuming that variables are
complex valued. This assumption makes it easier to treat mathematical problems
in full generality.

\begin{verbatim}
>>> x = Symbol('x')
>>> sqrt(x**2)
sqrt(x**2)
\end{verbatim}

By assuming symbols are complex by default, SymPy avoids performing
mathematically invalid operations. However, in many cases users will wish to
simplify expressions containing terms like $\sqrt{x^2}$.

Assumptions are set on \texttt{Symbol} objects when they are created. For
instance \texttt{Symbol('x', positive=True)} will create a symbol named
\texttt{x} that is assumed to be positive.

\begin{verbatim}
>>> x = Symbol('x', positive=True)
>>> sqrt(x**2)
x
\end{verbatim}

Some common assumptions that SymPy allows are \texttt{positive},
\texttt{negative}, \texttt{real}, \texttt{nonpositive}, \texttt{nonnegative},
\texttt{real}, \texttt{integer}, and \texttt{commutative}~\footnote{If $A$ and
  $B$ are Symbols created with \texttt{commutative=False} then SymPy will keep
  $A\cdot B$ and $B\cdot A$ distinct.}. Assumptions on any object can be checked with the
\verb|is_|\texttt{\textit{assumption}} attributes, like \verb|x.is_positive|.

Assumptions are only needed to restrict a domain so that certain
simplifications can be performed. It is not required to make the domain match
the input of a function. For instance, one can create the object
$\sum_{n=0}^m f(n)$ as \texttt{Sum(f(n), (n, 0, m))} without setting
\texttt{integer=True} when creating the Symbol object \texttt{n}.

The assumptions system additionally has deductive capabilities. The
assumptions use a three-valued logic using the Python builtin objects
\texttt{True}, \texttt{False}, and \texttt{None}. \texttt{None} represents the
``unknown'' case. This could mean that the given assumption could be either
true or false under the given information, for instance,
\verb|Symbol('x', real=True).is_positive| will give \texttt{None} because a real
symbol might be positive or it might not. It could also mean not enough is
implemented to compute the given fact, for instance,
\verb|(pi + E).is_irrational| gives \texttt{None}, because SymPy does not know
how to determine if $\pi + e$ is rational or irrational, indeed, it is an open
problem in mathematics. % ref?


Basic implications between the facts are used to deduce assumptions. For
instance, the assumptions system knows that being an integer implies being
rational, so \verb|Symbol('x', integer=True).is_rational| returns
\texttt{True}. Furthermore, expressions compute the assumptions on themselves
based on the assumptions of their arguments. For instance, if \texttt{x} and
\texttt{y} are both created with \texttt{positive=True}, then \verb|(x + y).is_positive|
will be \texttt{True}.

SymPy also has an experimental assumptions system where facts are stored
separate from objects, and deductions are made with a SAT solver. We will not
discuss this system here.

%%
%% Extensibility
\subsection{Extensibility}

Extensibility is an important feature for SymPy. Because the same language,
Python, is used both for the internal implementation and the external usage by
users, all the extensibility capabilities available to users are also used by
functions that are part of SymPy.

The typical way to create a custom SymPy object is to subclass an existing
SymPy class, generally either \texttt{Basic}, \texttt{Expr}, or
\texttt{Function}. All SymPy classes used for expression trees~\footnote{Some
  internal classes, such as those used in the polynomial module, do not follow
  this rule for efficiency reasons.} should be subclasses of the base class
\texttt{Basic}, which defines some basic methods for symbolic expression
trees. \texttt{Expr} is the subclass for mathematical expressions that can be
added and multiplied together. Instances of \texttt{Expr} are typically expressions involving
complex numbers, but may also include other ``rings'' like matrix expressions.
Not all SymPy classes are subclasses of \texttt{Expr}. For instance, logic expressions, such
as \texttt{And(x, y)} are subclasses of \texttt{Basic} but not of \texttt{Expr}.

The \texttt{Function} class is a subclass of \texttt{Expr} which makes it
easier to define mathematical functions called with arguments. This includes
named functions like $\sin(x)$ and $\log(x)$ as well as undefined functions
like $f(x)$. Subclasses of \texttt{Function} should define a
class method \texttt{eval}, which returns values for which the function should
be automatically evaluated, and \texttt{None} for arguments that shouldn't be
automatically evaluated.

The behavior of classes in SymPy with various other SymPy functions is defined
by defining a relevant \verb|_eval_|\texttt{\textit{*}} method on the class. For instance, an
object can tell SymPy's \texttt{diff} function how to take the derivative of
itself by defining the \verb|_eval_derivative(self, x)| method. The most
common \verb|_eval_|\texttt{\textit{*}} methods relate to the assumptions.
\verb|_eval_is_|\texttt{\textit{assumption}} defines the assumptions for
\textit{assumption}.

As an example of the notions presented in this section, we present below
a stripped down version of the gamma function $\Gamma(x)$ from SymPy,
which evaluates itself on positive integer arguments, has the positive and
real assumptions defined, can be rewritten in terms of factorial with
\texttt{gamma(x).rewrite(factorial)}, and can be differentiated.
\texttt{fdiff} is a convenience method for subclasses of \texttt{Function}.
\texttt{fdiff} returns the derivative of the function without worrying about
the chain rule. \texttt{self.func} is used throughout instead of referencing
\texttt{gamma} explicitly so that potential subclasses of \texttt{gamma} can
reuse the methods.

\begin{verbatim}
from sympy import Integer, Function, floor, factorial, polygamma

class gamma(Function)
    @classmethod
    def eval(cls, arg):
        if isinstance(arg, Integer) and arg.is_positive:
            return factorial(arg - 1)

    def _eval_is_real(self):
        x = self.args[0]
        # noninteger means real and not integer
        if x.is_positive or x.is_noninteger:
            return True

    def _eval_is_positive(self):
        x = self.args[0]
        if x.is_positive:
            return True
        elif x.is_noninteger:
            return floor(x).is_even

    def _eval_rewrite_as_factorial(self, z):
        return factorial(z - 1)

    def fdiff(self, argindex=1):
        from sympy.core.function import ArgumentIndexError
        if argindex == 1:
            return self.func(self.args[0])*polygamma(0, self.args[0])
        else:
            raise ArgumentIndexError(self, argindex)
\end{verbatim}

The actual gamma function defined in SymPy, \texttt{scipy.stats.gamma}, has many more capabilities, such as evaluation at rational points and series expansion.
