Software architecture is of central importance in any large software
project because it establishes predictable patterns of usage and
development~\cite{Shaw1996}. This section describes the essential
structural components of SymPy, provides justifications for the design
decisions that have been made, and gives example
user-facing code as appropriate.

\subsection{The Core}
\label{sec:core}
A computer algebra system stores mathematical expressions as data structures.
For example, the mathematical expression $x + y$ is represented as a tree with
three nodes, $+$, $x$, and $y$, where $x$ and $y$ are ordered children of $+$.
As users manipulate mathematical expressions with traditional mathematical
syntax, the CAS manipulates the underlying data structures. Symbolic
computations such as integration, simplification, etc.\ are all functions that
consume and produce expression trees.

In SymPy every symbolic expression is an instance of the class
\texttt{Basic},\footnote{Some internal classes, such as those used in the
  polynomial submodule, do not follow this rule for efficiency reasons.} the
superclass of all SymPy types providing common methods to all SymPy
tree-elements, such as traversals. The children of a node in the tree are held
in the \texttt{args} attribute. A leaf node in the expression tree
has empty \texttt{args}.

For example, consider the expression $xy + 2$:
\begin{verbatim}
>>> x, y = symbols('x y')
>>> expr = x*y + 2
\end{verbatim}
By order of operations, the parent of the expression tree for \texttt{expr} is
an addition. It is of type \texttt{Add}. The child nodes of \texttt{expr} are
\texttt{2} and \texttt{x*y}.
\begin{verbatim}
>>> type(expr)
<class 'sympy.core.add.Add'>
>>> expr.args
(2, x*y)
\end{verbatim}

Descending further down into the expression tree yields the full expression. For
example, the next child node (given by \texttt{expr.args[0]}) is
\texttt{2}. Its class is \texttt{Integer}, and it has an empty \texttt{args}
tuple, indicating that it is a leaf node.
\begin{verbatim}
>>> expr.args[0]
2
>>> type(expr.args[0])
<class 'sympy.core.numbers.Integer'>
>>> expr.args[0].args
()
\end{verbatim}
Symbols or symbolic constants, like $e$ or $\pi$, are other examples of
leaf nodes.
\begin{verbatim}
>>> exp(1)
E
>>> exp(1).args
()
>>> x.args
()
\end{verbatim}

A useful way to view an expression tree is using the \texttt{srepr} function, which
returns a string representation of an expression as valid Python code\footnote{
\label{note:dotprint}
The \texttt{dotprint} function from the \texttt{sympy.printing.dot} submodule
prints output to dot format, which can be rendered with Graphviz to
visualize expression trees graphically.}
with all the nested class constructor calls to create the given expression.
\begin{verbatim}
>>> srepr(expr)
"Add(Mul(Symbol('x'), Symbol('y')), Integer(2))"
\end{verbatim}

Every SymPy expression satisfies a key identity invariant:
\begin{verbatim}
expr.func(*expr.args) == expr
\end{verbatim}
This means that expressions are
rebuildable from their \texttt{args}.\footnote{\texttt{expr.func} is used
instead of \texttt{type(expr)} to allow the function of an expression to be
distinct from its actual Python class. In most cases the two are the same.}
Note that in SymPy the \texttt{==} operator represents exact
structural equality, not mathematical equality. This allows testing if any two
expressions are equal to one another as expression trees. For example, even
though ${(x + 1)}^2$ and $x^2 + 2x + 1$ are equal mathematically, SymPy gives
\begin{verbatim}
>>> (x + 1)**2 == x**2 + 2*x + 1
False
\end{verbatim}
because they are different as expression trees (the former is a \verb|Pow|
object and the latter is an \verb|Add| object).

Another important property of SymPy expressions is that they are immutable.
This simplifies the design of SymPy, and enables expression interning. It also
enables expressions to be hashed, which allows expressions to be used as keys
in Python dictionaries, and is used to implement caching in SymPy.

Python allows classes to override mathematical operators. The Python
interpreter translates the above \texttt{x*y + 2} to, roughly,
\verb|(x.__mul__(y)).__add__(2)|. Both \texttt{x} and \texttt{y}, returned
from the \texttt{symbols} function, are \texttt{Symbol} instances. The
\texttt{2} in the expression is processed by Python as a literal, and is
stored as Python's built in \texttt{int} type. When \texttt{2} is passed to the
\verb|__add__| method of \texttt{Symbol}, it is converted to the SymPy type
\verb|Integer(2)| before being stored in the resulting expression tree. In
this way, SymPy expressions can be built in the natural way using Python
operators and numeric literals.

%% TODO: describe how assumptions are implemented

%%
%% Extensibility
\subsection{Extensibility}

While the core of SymPy is relatively small, it has been extended to a wide variety
of domains by a broad range of contributors.
This is due, in part, to the fact that the same language, Python,
is used both for the internal implementation and the external usage by users.
All of the extensibility capabilities available to
users are also utilized by SymPy itself. This eases the transition pathway from
SymPy user to SymPy developer.

The typical way to create a custom SymPy object is to subclass an existing
SymPy class, usually \texttt{Basic}, \texttt{Expr}, or \texttt{Function}. As
it was stated before, all SymPy classes used for expression trees should be
subclasses of the base class \texttt{Basic}. \texttt{Expr} is the
\texttt{Basic} subclass for mathematical objects that can be added and
multiplied together. The most commonly seen classes in SymPy are subclasses of
\texttt{Expr}, including \texttt{Add}, \texttt{Mul}, and \texttt{Symbol}.
Instances of \texttt{Expr} typically represent complex numbers, but may also
include other ``rings'', like matrix expressions. Not all SymPy classes are
subclasses of \texttt{Expr}. For instance, logic expressions, such as
\verb|And(x, y)|, are subclasses of \texttt{Basic} but not of
\texttt{Expr}.\footnote{See the supplement for more information on the
  \texttt{sympy.logic} submodule.}

The \texttt{Function} class is a subclass of \texttt{Expr} which makes it
easier to define mathematical functions called with arguments. This includes
named functions like $\sin(x)$ and $\log(x)$ as well as undefined functions
like $f(x)$. Subclasses of \texttt{Function} should define a
class method \texttt{eval}, which returns an evaluated value for the function
application (usually an instance of some other class, e.g., a \texttt{Number}),
or \texttt{None} if for the given arguments it should not be
automatically evaluated.

Many SymPy functions perform various evaluations down the expression tree.
Classes define their behavior in such functions by defining a relevant
\verb|_eval_|\texttt{\textit{*}} method. For instance, an object can indicate
to the \texttt{diff} function how to take the derivative of itself by defining
the \verb|_eval_derivative(self, x)| method, which may in turn call
\texttt{diff} on its \texttt{args}. (Subclasses of \texttt{Function} should
implement the \texttt{fdiff} method instead; it returns the derivative of the function
without considering the chain rule.) The most common
\verb|_eval_|\texttt{\textit{*}} methods relate to the assumptions:
\verb|_eval_is_|\texttt{\textit{assumption}} is used to deduce
\textit{assumption} on the object.

Listing~\ref{fig:gamma-example} presents an example of this extensibility. It
gives a stripped down version of the \texttt{gamma} function $\Gamma(x)$ from
SymPy. The methods defined allow it to evaluate itself on positive integer
arguments, define the real assumption, allow it to be rewritten in terms of
factorial (with \verb|gamma(x).rewrite(factorial)|), and allow it to be
differentiated. \texttt{self.func} is used throughout instead of referencing
\texttt{gamma} explicitly so that potential subclasses of \texttt{gamma} can
reuse the methods.

\lstset{
  basicstyle=\ttfamily,
}

\begin{lstlisting}[caption={A minimal implementation of \texttt{sympy.gamma}.},label=fig:gamma-example]
from sympy import Integer, Function, floor, factorial, polygamma

class gamma(Function)
    @classmethod
    def eval(cls, arg):
        if isinstance(arg, Integer) and arg.is_positive:
            return factorial(arg - 1)

    def _eval_is_real(self):
        x = self.args[0]
        # noninteger means real and not integer
        if x.is_positive or x.is_noninteger:
            return True

    def _eval_rewrite_as_factorial(self, z):
        return factorial(z - 1)

    def fdiff(self, argindex=1):
        from sympy.core.function import ArgumentIndexError
        if argindex == 1:
            return self.func(self.args[0])*polygamma(0, self.args[0])
        else:
            raise ArgumentIndexError(self, argindex)
\end{lstlisting}
The gamma function implemented in SymPy has many more capabilities than the
above listing, such as evaluation at rational points and series expansion.


\subsection{Performance}
\label{sec:performance}

Due to being written in pure Python without the use of extension modules,
SymPy's performance characteristics are generally poorer than
its commercial competitors. For many applications,
the performance of SymPy, as measured by clock cycles, memory usage, and memory
layout, is sufficient.
However, the boundaries for when SymPy's pure Python strategy becomes
insufficient are when the user requires handling of very long expressions or many
small expressions. Where this boundray lies depends on the system at hand, but tends
to be within the range of $10^4$--$10^6$ symbols for modern computers.

For this reason, a new project called SymEngine~\cite{SymEngine} has been started.
The aim of this poject is to develop a library with better performance
characteristics for symbolic manipulation. SymEngine is a pure C++ library,
which allows it fine grain control over the memory layout of expressions.
SymEngine has thin wrappers to other languages (Python, Ruby,
Julia, etc.) whose aim is to be the fastest symbolic manipulation library. Preliminary
benchmarks suggest that SymEngine is as performant as its commercial and
open source competitors.

SymPy has recently started to use SymEngine as an
optional backend, initially in \texttt{sympy.\allowbreak{}physics.\allowbreak{}mechanics} only.
Future work will involve
allowing more algorithms in SymPy to use SymEngine as a backend.
