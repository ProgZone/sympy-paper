%TODO put some leading text here.

\subsection{Basic Usage}

% symbols, various ways to declare them

Because SymPy is built on Python, it requires that all variable names be
defined prior to use. The following statement
imports all SymPy functions into the global Python namespace.
From here on, all examples in this paper assume that this statement has been
run.

\begin{verbatim}
>>> from sympy import *
\end{verbatim}

Symbolic variables, called symbols, must be defined and assigned to
Python variables before they can be used. This is typically done through the
\texttt{symbols} function, which may create multiple symbols in a single call. For
instance,
\begin{verbatim}
>>> x, y, z = symbols('x y z')
\end{verbatim}
creates three symbols representing variables named $x$, $y$, and $z$. In this
particular instance, these symbols are all assigned to Python variables of the
same name. However, the user is free to assign them to different
Python variables, while representing the same symbol, such as
\texttt{a, b, c = symbols(\textquotesingle{}x y z\textquotesingle{})}.
In order to minimize potential confusion, though, all examples in this paper will
assume that
the symbols \verb|x|, \verb|y|, and \verb|z| have been assigned to Python variables
identical to their symbolic names.

Expressions are created from symbols using Python mathematical syntax. Note
that in Python, exponentiation is represented by \verb|**|. For instance, the
following Python code creates the expression $(x^2 - 2x + 3)/y$.

\begin{verbatim}
>>> (x**2 - 2*x + 3)/y
(x**2 - 2*x + 3)/y
\end{verbatim}

Importantly, SymPy expressions are immutable. This simplifies the design of
SymPy by allowing expression interning. It also enables expressions to be
hashed and stored in Python dictionaries, thereby permitting caching and
other features.

%% I volunteer to write this section. --Aaron
%%
%% Representing symbolic expressions using Python objects

\subsection{The Core}

A computer algebra system (CAS) represents mathematical expressions as data
structures.  For example, the mathematical expression $x + y$ is represented as
a tree with three nodes, $+$, $x$, and $y$, where $x$ and $y$ are ordered
children of $+$.  As users manipulate
mathematical expressions with traditional mathematical syntax, the CAS
manipulates the underlying data structures.  Automated optimizations and
computations such as integration, simplification, etc.\ are all functions that
consume and produce expression trees.

In SymPy every symbolic expression is an instance of a Python \texttt{Basic}
class, a superclass of all SymPy types providing common methods to all SymPy
tree-elements, such as traversals.  The children of a node in the
tree are held in the \texttt{args} attribute.  A terminal or leaf node in the
expression tree has empty \texttt{args}.

For example, consider the expression $xy + 2$:
\begin{verbatim}
>>> expr = x*y + 2
\end{verbatim}
By order of operations, the parent of the expression tree for \texttt{expr} is
an addition, so it is of type \texttt{Add}. The child nodes of \texttt{expr} are
\texttt{2} and \texttt{x*y}.
\begin{verbatim}
>>> type(expr)
<class 'sympy.core.add.Add'>
>>> expr.args
(2, x*y)
\end{verbatim}

Traversing further into the expression tree grants the full expression. For
example, the first child node, given by \texttt{expr.args[0]}, is
\texttt{2}. Its class is \texttt{Integer}, and it has empty an \texttt{args}
tuple, indicating that it is a leaf node.
\begin{verbatim}
>>> expr.args[0]
2
>>> type(expr.args[0])
<class 'sympy.core.numbers.Integer'>
>>> expr.args[0].args
()
\end{verbatim}

A useful way to view an expression tree is with the \texttt{srepr} function.
This returns a string representation of an expression as valid Python code
with all the nested class constructor calls to create the given expression.
\begin{verbatim}
>>> srepr(expr)
"Add(Mul(Symbol('x'), Symbol('y')), Integer(2))"
\end{verbatim}

Every SymPy expression satisfies a key identity invariant:
\begin{verbatim}
expr.func(*expr.args) == expr
\end{verbatim}
This means that expressions are
rebuildable from their \texttt{args}.\footnote{\texttt{expr.func} is used
instead of \texttt{type(expr)} to allow the function of an expression to be
distinct from its actual Python class. In most cases the two are the same.}
We note that in SymPy, the \texttt{==} operator represents exact
structural equality, not mathematical equality. This allows one to test if
any two expressions are equal to one another as expression trees.

Python allows classes to override mathematical operators. The Python
interpreter translates the above \texttt{x*y + 2} to, roughly,
\verb|(x.__mul__(y)).__add__(2)|. Both \texttt{x} and \texttt{y}, returned
from the \texttt{symbols} function, are \texttt{Symbol} instances. The
\texttt{2} in the expression is processed by Python as a literal, and is
stored as Python's built in \texttt{int} type. When \texttt{2} is passed to the
\verb|__add__| method of \texttt{Symbol}, it is converted to the SymPy type
\verb|Integer(2)| before being stored in the resulting expression tree. In
this way, SymPy expressions can be built in the natural way using Python
operators and numeric literals.

%%
%% Assumptions
\subsection{Assumptions}
\label{sec:assumptions}

SymPy performs logical inference through its assumptions system. The
assumptions system allows users to specify that symbols have certain common
mathematical properties, such as being positive, imaginary, or integral. SymPy
is careful to never perform simplifications on an expression unless the
assumptions allow them. For instance, the identity $\sqrt{t^2} = t$ holds if
$t$ is nonnegative ($t\ge 0$). If $t$ is real, the identity $\sqrt{t^2}=|t|$
holds. However, for general complex $t$, no such identity holds.

By default, SymPy performs all calculations assuming that symbols are
complex valued. This assumption makes it easier to treat mathematical problems
in full generality.
\begin{verbatim}
>>> t = Symbol('t')
>>> sqrt(t**2)
sqrt(t**2)
\end{verbatim}

By assuming the most general case, that symbols are complex by default, SymPy
avoids performing mathematically invalid operations. However, in many cases
users will wish to simplify expressions containing terms like $\sqrt{t^2}$.

Assumptions are set on \texttt{Symbol} objects when they are created. For
instance \verb|Symbol('t', positive=True)| will create a symbol named
\texttt{t} that is assumed to be positive.
\begin{verbatim}
>>> t = Symbol('t', positive=True)
>>> sqrt(t**2)
t
\end{verbatim}

Some of the common assumptions that SymPy allows are \texttt{positive},
\texttt{negative}, \texttt{real}, \texttt{nonpositive}, \texttt{nonnegative},
\texttt{real}, \texttt{integer}, and \texttt{commutative}.\footnote{If $A$ and
$B$ are Symbols created with \texttt{commutative=False} then SymPy will keep
$A\cdot B$ and $B\cdot A$ distinct.} Assumptions on any object can be checked with the
\verb|is_|\texttt{\textit{assumption}} attributes, like \verb|t.is_positive|.

Assumptions are only needed to restrict a domain so that certain
simplifications can be performed. It is not required to make the domain match
the input of a function. For instance, one can create the object
$\sum_{n=0}^m f(n)$ as \verb|Sum(f(n), (n, 0, m))| without setting
\texttt{integer=True} when creating the Symbol object \texttt{n}.

The assumptions system additionally has deductive capabilities. The
assumptions use a three-valued logic using the Python built in objects
\texttt{True}, \texttt{False}, and \texttt{None}. \texttt{None} represents the
``unknown'' case. This could mean that the given assumption could be either
true or false under the given information, for instance,
\verb|Symbol('x', real=True).is_positive| will give \texttt{None} because a real
symbol might be positive or it might not. It could also mean not enough is
implemented to compute the given fact. For instance,
\verb|(pi + E).is_irrational| gives \texttt{None}, because SymPy does not know
how to determine if $\pi + e$ is rational or irrational, indeed, it is an open
problem in mathematics.
% TODO: ref?


Basic implications between the facts are used to deduce assumptions. For
instance, the assumptions system knows that being an integer implies being
rational, so \verb|Symbol('x', integer=True).is_rational| returns
\texttt{True}. Furthermore, expressions compute the assumptions on themselves
based on the assumptions of their arguments. For instance, if \texttt{x} and
\texttt{y} are both created with \texttt{positive=True}, then
\verb|(x + y).is_positive| will be \texttt{True}.

%% TODO: describe how assumptions are implemented

%%
%% Extensibility
\subsection{Extensibility}

While the core of SymPy is relatively small, it has been extended to a wide variety
of domains by a broad range of contributors.  This is due in part because the
same language, Python, is used both for the internal implementation and the
external usage by users.  All of the extensibility capabilities available to
users are also utilized by SymPy itself. This eases the transition pathway from
SymPy user to SymPy developer.

The typical way to create a custom SymPy object is to subclass an existing
SymPy class, usually  \texttt{Basic}, \texttt{Expr}, or
\texttt{Function}. All SymPy classes used for expression trees\footnote{Some
  internal classes, such as those used in the polynomial module, do not follow
  this rule for efficiency reasons.} should be subclasses of the base class
\texttt{Basic}, which defines some basic methods for symbolic expression
trees. \texttt{Expr} is the subclass for mathematical expressions that can be
added and multiplied together. Instances of \texttt{Expr} typically represent
complex numbers, but may also include other ``rings'' like matrix expressions.
Not all SymPy classes are subclasses of \texttt{Expr}. For instance, logic expressions such
as \verb|And(x, y)| are subclasses of \texttt{Basic} but not of \texttt{Expr}.

The \texttt{Function} class is a subclass of \texttt{Expr} which makes it
easier to define mathematical functions called with arguments. This includes
named functions like $\sin(x)$ and $\log(x)$ as well as undefined functions
like $f(x)$. Subclasses of \texttt{Function} should define a
class method \texttt{eval}, which returns values for which the function should
be automatically evaluated, and \texttt{None} for arguments that should not be
automatically evaluated.

Many SymPy functions perform various evaluations down the expression tree.
Classes define their behavior in such functions by defining a relevant
\verb|_eval_|\texttt{\textit{*}} method. For instance, an object can indicate
to the \texttt{diff} function how to take the derivative of itself by defining
the \verb|_eval_derivative(self, x)| method, which may in turn call
\texttt{diff} on its \texttt{args}. The most common
\verb|_eval_|\texttt{\textit{*}} methods relate to the assumptions.
\verb|_eval_is_|\texttt{\textit{assumption}} defines the assumptions for
\textit{assumption}.

As an example of the notions presented in this section,
Listing~\ref{fig:gamma-example} presents a stripped down version of the gamma
function $\Gamma(x)$ from SymPy, which evaluates itself on positive integer
arguments, has the positive and real assumptions defined, can be rewritten in
terms of factorial with \verb|gamma(x).rewrite(factorial)|, and can be
differentiated. \texttt{fdiff} is a convenience method for subclasses of
\texttt{Function}. \texttt{fdiff} returns the derivative of the function
without considering the chain rule. \texttt{self.func} is used throughout
instead of referencing \texttt{gamma} explicitly so that potential subclasses
of \texttt{gamma} can reuse the methods.

\lstset{
  basicstyle=\ttfamily,
}

\begin{lstlisting}[caption={A stripped down version of \texttt{sympy.gamma}.},label=fig:gamma-example]
from sympy import Integer, Function, floor, factorial, polygamma

class gamma(Function)
    @classmethod
    def eval(cls, arg):
        if isinstance(arg, Integer) and arg.is_positive:
            return factorial(arg - 1)

    def _eval_is_positive(self):
        x = self.args[0]
        if x.is_positive:
            return True
        elif x.is_noninteger:
            return floor(x).is_even

    def _eval_is_real(self):
        x = self.args[0]
        # noninteger means real and not integer
        if x.is_positive or x.is_noninteger:
            return True

    def _eval_rewrite_as_factorial(self, z):
        return factorial(z - 1)

    def fdiff(self, argindex=1):
        from sympy.core.function import ArgumentIndexError
        if argindex == 1:
            return self.func(self.args[0])*polygamma(0, self.args[0])
        else:
            raise ArgumentIndexError(self, argindex)
\end{lstlisting}
The actual gamma function defined in SymPy has many more capabilities, such as
evaluation at rational points and series expansion.
