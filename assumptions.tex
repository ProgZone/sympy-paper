The assumptions system allows users to specify that symbols have certain
common mathematical properties, such as being positive, imaginary, or
integer. SymPy is careful to never perform simplifications on an expression
unless the assumptions allow them. For instance, the simplification $\sqrt{t^2}
= t$ holds if $t$ is nonnegative ($t\ge 0$), but it does not hold for a general
complex $t$.%
\footnote{In SymPy, $\sqrt{z}$ is defined on the usual principal branch with
the branch cut along the negative real axis.}


By default, SymPy performs all calculations assuming that symbols are
complex valued. This assumption makes it easier to treat mathematical problems
in full generality.
\begin{verbatim}
>>> t = Symbol('t')
>>> sqrt(t**2)
sqrt(t**2)
\end{verbatim}

By assuming the most general case, that \texttt{t} is complex by default, SymPy
avoids performing mathematically invalid operations. However, in many cases
users will wish to simplify expressions containing terms like $\sqrt{t^2}$.

Assumptions are set on \texttt{Symbol} objects when they are created. For
instance \texttt{Symbol(\textquotesingle{}t\textquotesingle{}, positive=True)} will create a symbol named
\texttt{t} that is assumed to be positive.
\begin{verbatim}
>>> t = Symbol('t', positive=True)
>>> sqrt(t**2)
t
\end{verbatim}

Some of the common assumptions are
\texttt{negative}, \texttt{real}, \texttt{nonpositive}, \texttt{integer},
\texttt{prime} and \texttt{commutative}.\footnote{SymPy assumes that two
  expressions $A$ and $B$ commute with each other multiplicatively, that is,
  $A\cdot B=B\cdot A$, unless they both have \texttt{commutative=False}. Many
  algorithms in SymPy require special consideration to work correctly with
  noncommutative products.} Assumptions on any SymPy object can be checked with the
\verb|is_|\texttt{\textit{assumption}} attributes, like \verb|t.is_positive|.

Assumptions are only needed to restrict a domain so that certain
simplifications can be performed. They are not required to make the domain match
the input of a function. For instance, one can create the object
$\sum_{n=0}^m f(n)$ as \verb|Sum(f(n), (n, 0, m))| without setting
\texttt{integer=True} when creating the Symbol object \texttt{n}.

The assumptions system additionally has deductive capabilities. The
assumptions use a three-valued logic using the Python built in objects
\texttt{True}, \texttt{False}, and \texttt{None}. Note that \texttt{False}
is returned if the SymPy object doesn't or can't have the assumption.
For example, both \texttt{I.is\_real} and \texttt{I.is\_prime} return
\texttt{False} for the imaginary unit \texttt{I}.

\texttt{None} represents the ``unknown'' case. This could mean that given
assumptions do not unambiguously specify the truth of an attribute. For
instance, \texttt{Symbol(\textquotesingle{}x\textquotesingle{},
  real=True).is\_positive} will give \texttt{None} because a real symbol might
be positive or negative. \texttt{None} could also mean that not enough is
known or implemented to compute the given fact. For instance, \texttt{(pi +
  E).is\_irrational} gives \texttt{None}---indeed, the rationality of
$\pi + e$ is an open problem in mathematics~\cite{lang1966introduction}.

Basic implications between the facts are used to deduce assumptions.
Deductions are made using the Rete
algorithm~\cite{doorenbos1995production}.\footnote{\label{note:assumptions}For
  historical reasons, this algorithm is distinct from the \texttt{sympy.logic}
  submodule, which is discussed in section~\ref{S-suppsec:Logic} of the
  supplementary material. SymPy also has an experimental assumptions system
  which stores facts separate from objects, and uses \texttt{sympy.logic} and
  a SAT solver for deduction. We will not discuss this system here.} For
instance, the assumptions system knows that being an integer implies being
rational.
\begin{verbatim}
>>> i = Symbol('i', integer=True)
>>> i.is_rational
True
\end{verbatim}
Furthermore, expressions compute the assumptions on themselves
based on the assumptions of their arguments. For instance, if \texttt{x} and
\texttt{y} are both created with \texttt{positive=True}, then
\texttt{(x + y).is\_positive} will be \texttt{True} (whereas
\texttt{(x - y).is\_positive} will be \texttt{None}).
