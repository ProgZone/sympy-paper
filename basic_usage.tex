% symbols, various ways to declare them

Symbolic variables, called symbols, must be defined and assigned to
Python variables before they can be used. This is typically done through the
\texttt{symbols} function, which may create multiple symbols in a single
function call. For instance,
\begin{verbatim}
>>> x, y, z = symbols('x y z')
\end{verbatim}
creates three symbols representing variables named $x$, $y$, and $z$. In this
particular instance, these symbols are all assigned to Python variables of the
same name. However, the user is free to assign them to different
Python variables, while representing the same symbol, such as
\texttt{a, b, c = symbols(\textquotesingle{}x y z\textquotesingle{})}.
In order to minimize potential confusion, though, all examples in this paper will
assume that
the symbols \verb|x|, \verb|y|, and \verb|z| have been assigned to Python variables
identical to their symbolic names.

Expressions are created from symbols using Python's mathematical syntax.  For
instance, the following Python code creates the expression $(x^2 - 2x + 3)/y$.
Note that the expression remains unevaluated: it is represented symbolically.

\begin{verbatim}
>>> (x**2 - 2*x + 3)/y
(x**2 - 2*x + 3)/y
\end{verbatim}

Importantly, SymPy expressions are immutable. This simplifies the design of
SymPy by allowing expression interning. It also enables expressions to be
hashed, which is used to implement caching in SymPy.
