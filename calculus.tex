
% Calculus (differentiation, integration, limits). Note that algorithm
% descriptions will go in the algorithms section.

Derivatives can be computed with the \verb|diff| function.
\begin{verbatim}
>>> x = symbols("x")
>>> diff(sin(x), x)
cos(x)
\end{verbatim}

Unevaluated \verb|Derivative| objects are also supported.
\begin{verbatim}
>>> expr = Derivative(sin(x), x)
>>> expr
Derivative(sin(x), x)
\end{verbatim}

Unevaluated expressions can be evaluated with the \verb|doit| method.
\begin{verbatim}
>>> expr.doit()
cos(x)
\end{verbatim}

% TODO: A more interesting example here
Integrals can be analogously, calculated either with the \verb|integrate|
function, or the unevaluated \verb|Integral| objects.
\begin{verbatim}
>>> integrate(sin(x), x)
-cos(x)
>>> expr = Integral(sin(x), x)
>>> expr
Integral(sin(x), x)
>>> expr.doit()
-cos(x)
\end{verbatim}
Definite integration can be calculated with the same method, by specifying a
range of the integration variable. The following computes $\int_0^1\sin(x)\,dx$.
\begin{verbatim}
>>> integrate(sin(x), (x, 0, 1))
-cos(1) + 1
\end{verbatim}

SymPy implements a combination of the Risch
algorithm~\cite{bronstein2005integration}, table lookups, a reimplementation
of Manuel Bronstein's ``Poor Man's Integrator''~\cite{Bronstein2005pmint}, and
an algorithm for computing integrals based on Meijer G-functions. These allow
SymPy to compute a wide variety of indefinite and definite integrals.
% TODO: What is the best citation for the Meijer G-function algorithm.
% TODO: Add some examples here.

Summations and products are also supported, via the evaluated \verb|summation|
and \verb|product| and unevaluated \verb|Sum| and \verb|Product|, and use the
same syntax as \verb|integrate|. Summations are computed using a combination
of Gosper's algorithm and an algorithm that uses Meijer G-functions. Products
are computed via some heuristics.
% TODO: Citations for Gosper and Meijer G-function algorithms
% TODO: Are there other summation algorithms implemented?

The limit module implements the Gruntz algorithm~\cite{Gruntz1996limits} for
computing symbolic limits. For example, the following computes
$\lim\limits_{x\to \infty} x\sin(\frac{1}{x})=1$ (note that $\infty$ is
\verb|oo| in SymPy).
\begin{verbatim}
>>> limit(x*sin(1/x), x, oo)
1
\end{verbatim}
As a more complicated example, SymPy computes $\lim\limits_{x\to 0}{\left(2 e^{\frac{1 - \cos{\left (x \right )}}{\sin{\left (x \right )}}} -
  1\right)}^{\frac{\sinh{\left (x \right )}}{\operatorname{atan}^{2}{\left (x
      \right )}}} = e$.
\begin{verbatim}
>>> limit((2*E**((1-cos(x))/sin(x))-1)**(sinh(x)/atan(x)**2), x, 0)
E
\end{verbatim}
