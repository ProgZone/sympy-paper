Integrals are calculated either with the \verb|integrate| function. SymPy
implements a combination of the Risch
algorithm~\cite{bronstein2005integration}, table lookups, a reimplementation
of Manuel Bronstein's ``Poor Man's Integrator''~\cite{Bronstein2005pmint}, and
an algorithm for computing integrals based on Meijer G-functions. These allow
SymPy to compute a wide variety of indefinite and definite integrals.
% TODO: What is the best citation for the Meijer G-function algorithm.
% TODO: A more interesting examples here
\begin{verbatim}
>>> integrate(sin(x), x)
-cos(x)
\end{verbatim}
Definite integrals are calculated with the same function by specifying a
range of the integration variable. The following computes $\int_0^1\sin(x)\,dx$.
\begin{verbatim}
>>> integrate(sin(x), (x, 0, 1))
-cos(1) + 1
\end{verbatim}

Derivatives are computed with the \verb|diff| function. Derivatives are
computed recursively using the various differentiation rules.
\begin{verbatim}
>>> diff(sin(x)*exp(x), x)
exp(x)*sin(x) + exp(x)*cos(x)
\end{verbatim}

Summations and products are also supported, via \verb|summation| and
\verb|product|. Summations are computed using a combination of Gosper's
algorithm, an algorithm that uses Meijer G-functions, and heuristics. Products
are computed via some heuristics.
% TODO: Citations for Gosper and Meijer G-function algorithms
% TODO: Are there other summation algorithms implemented?
% TODO: A good summation example or two

Limits are computed with the \verb|limit| function. The limit module
implements the Gruntz algorithm~\cite{Gruntz1996limits} for computing symbolic
limits. For example, the following computes
$\lim\limits_{x\to \infty} x\sin(\frac{1}{x})=1$ (note that $\infty$ is
\verb|oo| in SymPy).
\begin{verbatim}
>>> limit(x*sin(1/x), x, oo)
1
\end{verbatim}
As a more complicated example, SymPy computes $\lim\limits_{x\to 0}{\left(2 e^{\frac{1 - \cos{\left (x \right )}}{\sin{\left (x \right )}}} -
  1\right)}^{\frac{\sinh{\left (x \right )}}{\operatorname{atan}^{2}{\left (x
      \right )}}} = e$.
\begin{verbatim}
>>> limit((2*E**((1-cos(x))/sin(x))-1)**(sinh(x)/atan(x)**2), x, 0)
E
\end{verbatim}

Integrals, derivatives, summations, products, and limits that can't be
computed return unevaluated objects. These can also be created directly if the
user chooses.
\begin{verbatim}
>>> integrate(x**x, x)
Integral(x**x, x)
\end{verbatim}
