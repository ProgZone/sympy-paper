SymPy includes a submodule for dealing with categories---abstract mathematical
objects representing classes of structures as classes of objects (points) and
morphisms (arrows) between the objects. It was
designed with the following two goals in mind:

\begin{enumerate}
\item automatic typesetting of diagrams given by a collection of
  objects and of morphisms between them, and
\item specification and semi-automatic derivation of properties
  using commutative diagrams.
\end{enumerate}

As of version 1.0, SymPy only implements the first goal, while a partially
working draft of implementation of the second goal is available at
\url{https://github.com/scolobb/sympy/tree/ct4-commutativity}.

In order to achieve the two goals, the submodule \texttt{sympy.categories} defines
several classes representing some of the essential concepts: objects, morphisms,
categories, and diagrams.  In category theory, the inner structure of objects is
often discarded in the favor of studying the properties of morphisms, so the
class \texttt{Object} is essentially a synonym of the class \texttt{Symbol}.
There are several morphism classes which do not have a particular internal
structure either, though an exception is \texttt{CompositeMorphism}, which
essentially stores a list of morphisms.

The class \texttt{Diagram} captures the properties of morphisms. This class
stores a family of morphisms, the corresponding source and target objects,
and, possibly, some properties of the morphisms. Generally, no restrictions
are imposed on what the properties may be---for example, one might use strings
of the form ``forall'', ``exists'', ``unique'', etc. Furthermore, the
morphisms of a diagram are grouped into \textit{premises} and
\textit{conclusions} in order to be able to represent logical implications of
the form ``for a collection of morphisms $P$ with properties $p:P\to \Omega$
(the premises), there exists a collection of morphisms $C$ with properties
$c:C\to \Omega$ (the conclusions)'', where $\Omega$ is the universal
collection of properties. Finally, the class \texttt{Category} includes a
collection of diagrams which are deemed commutative and which therefore define
the properties of this category.

Automatic typesetting of diagrams takes a \texttt{Diagram} and produces \LaTeX{}
code using the \texttt{Xy-pic} package~\cite{rose1999xy}.  Typesetting is done in two stages:
layout and generation of \texttt{Xy-pic} code.  The layout stage is taken care
of by the class \texttt{DiagramGrid}, which takes a \texttt{Diagram} and lays out
the objects in a grid, trying to reduce the average length of the arrows in the
final picture.  By default, \texttt{DiagramGrid} uses a series of triangle-based
heuristics to produce a rectangular grid.  A linear layout can also be imposed.
Furthermore, groups of objects can be given; in this case, the groups will be
treated as atomic cells, and the member objects will be typeset independently of
the other objects.

The second phase of diagram typesetting consists in actually drawing the picture
and is carried out by the class \texttt{XypicDiagramDrawer}.  An example of a
diagram automatically typeset by \texttt{DiagramgGrid} and
\texttt{XypicDiagramDrawer} is given in Figure~\ref{fig:cat:loops}.
\begin{figure}[h]
  \centerline{
    \xymatrix{
      A \ar[r]_{f} \ar@/^3mm/[rr]^{h_{2}} \ar@(u,l)[]^{l_{A}} \ar@/^3mm/@(l,d)[]^{n_{A}} & B \ar[d]^{g} & D \ar[l]^{k} \ar@/_7mm/[ll]_{h} \ar@/_11mm/[ll]_{h_{1}} \ar@(r,u)[]^{l_{D}} \ar@/^3mm/@(d,r)[]^{n_{D}} \\
      & C \ar@(l,d)[]^{l_{C}} \ar@/^3mm/@(d,r)[]^{n_{C}} &
    }
  }
  \caption{An automatically typeset commutative diagram}\label{fig:cat:loops}
\end{figure}

As far as the second main goal of \texttt{sympy.categories} is concerned, the principal idea
consists in automatically deciding whether a diagram is commutative or not,
given a collection of ``axioms'': diagrams known to be commutative. The
implementation is based on graph embeddings (injective maps): whenever an
embedding of a commutative diagram into a given diagram is found, one
concludes that the subdiagram is commutative. Deciding commutativity of the
whole diagram is therefore based (theoretically) on finding a ``cover'' of the
target diagram by embeddings of the axioms. The na\"{\i}ve implementation
proved to be prohibitively slow; a better optimized version is therefore in
order, as well as application of heuristics.
