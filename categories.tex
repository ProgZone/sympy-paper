SymPy includes a basic version of the module for dealing with
categories — abstract mathematical objects representing classes of
structures as classes of objects (points) and morphisms (arrows)
between the objects.  This version of the module was designed with the
following two goals in mind:

\begin{enumerate}
\item automatic typesetting of diagrams given by a collection of
  objects and of morphisms between them, and
\item specification and (semi-)automatic derivation of properties
  using commutative diagrams.
\end{enumerate}

At the time of writing of this paper, the version in the {\tt master}
branch only implements the first goal, while a (very partially
working) draft of implementation of the second goal is available at
\url{https://github.com/scolobb/sympy/tree/ct4-commutativity}.

In order to achieve the two goals, the module {\tt categories} defines
several classes representing some of the essential concepts: objects,
morphisms, categories, diagrams.  Since in category theory the inner
structure of its objects is often discarded (in the favour of studying
the properties of morphisms), the class {\tt Object} is essentially a
synonym of the class {\tt Symbol}.  There are several morphism classes
which do not have a particular internal structure either, except for
{\tt CompositeMorphism}, which essentially stores a list of morphisms.
To capture the properties of morphisms, the class {\tt Diagram} is
expected to be used.  This class stores a family of morphisms, the
corresponding source and target objects, and, possibly, some
properties of the morphisms.  Generally, no restrictions are imposed
on what the properties may be --- for example, one might use strings
of the form ``forall'', ``exists'', ``unique'', etc.  Furthermore, the
morphisms of a diagram are grouped into {\em premises} and {\em
  conclusions}, in order to be able to represent logical implications
of the form ``for a collection of mophisms $P$ with properties $p:P\to
\Omega$ (the premises), there exists a collection of morphisms $C$
with properties $c:C\to \Omega$ (the conclusions)'', where $\Omega$ is
the universal collection of properties.  Finally, the class {\tt
  Category} includes a collection of diagrams which are deemed
commutative and which therefore define the properties of this
category.

Automatic typesetting of diagrams takes a {\tt Diagram} and produces
\LaTeX~code using the {\tt Xy-pic} package.  Typesetting is done in
two stages: layout and generation of {\tt Xy-pic} code.  The layout
stage is taken care of by the class {\tt DiagramGrid} which takes a
{\tt Diagram} and lays out the objects in a grid, trying to reduce the
average length of the arrows in the final picture.  By default, {\tt
  DiagramGrid} uses a series of triangle-based heuristics to produce a
rectangular grid.  A linear layout can also be imposed.  Furthermore,
groups of objects can be given; in this case, the groups will be
treated as atomic cells, and the member objects will be typeset
independently of the other objects.

The second phase of diagram typesetting consists in actually drawing
the picture and is carried out by the class {\tt XypicDiagramDrawer}.
An example of a diagram automatically typeset by {\tt DiagramgGrid}
and {\tt XypicDiagramDrawer} in given in Figure~\ref{fig:cat:loops}.
\begin{figure}[h]
  \centerline{
    \xymatrix{
      A \ar[r]_{f} \ar@/^3mm/[rr]^{h_{2}} \ar@(u,l)[]^{l_{A}} \ar@/^3mm/@(l,d)[]^{n_{A}} & B \ar[d]^{g} & D \ar[l]^{k} \ar@/_7mm/[ll]_{h} \ar@/_11mm/[ll]_{h_{1}} \ar@(r,u)[]^{l_{D}} \ar@/^3mm/@(d,r)[]^{n_{D}} \\
      & C \ar@(l,d)[]^{l_{C}} \ar@/^3mm/@(d,r)[]^{n_{C}} &
    }
  }
  \caption{An automatically typeset commutative diagram}
  \label{fig:cat:loops}
\end{figure}

As far as the second main goal of the module is concerned, a
(non-working) draft of an implementation is in
\url{https://github.com/scolobb/sympy/tree/ct4-commutativity}.  The
principal idea consists in automatically deciding whether a diagram is
commutative or not, given a collection of ``axioms'' --- diagrams {\em
  known} to be commutative.  The approach to implementation is based
on graph embeddings (injective maps): whenever an embedding of a
commutative diagram into a given diagram is found, one concludes that
that subdiagram is commutative.  Deciding commutativity of the whole
diagram is therefore based (theoretically) on finding a ``cover'' of
the target diagram by embeddings of the axioms.  The naïve
implementation proved to be prohibitively slow; a better optimised
version is therefore in order, as well as application of heuristics.

Contributions to automatic inference of commutativity of diagrams are
welcome.  The source code (both the one in master and in {\tt
  ct4-commutativity}) is extensively documented.  Even more extensive
explanations (including some literary chatter) are given in
\url{https://scolobb.wordpress.com/}.
