
Wolfram Mathematica is a popular proprietary CAS\@.
It features highly advanced algorithms.
Mathematica has a core implemented in C++~\cite{Wolfram2016}
which interprets its own programming language, Wolfram Language.

% M-expressions

Analogous to Lisp S-expressions,
Mathematica uses its own style of M-expres\-sions,
which are arrays of either atoms or other M-expressions.
The first element of the expression identifies the type of the expression
and is indexed by zero, and the first argument is indexed starting with one.
In SymPy, expression arguments are stored in a Python tuple
(that is, an immutable array),
while the expression type is identified by the type of the object storing the
expression.

% Attributes

Mathematica can associate attributes to its atoms.
Attributes may define mathematical properties and behavior of the nodes
associated to the atom.
In SymPy, the usage of static class fields is roughly similar to Mathematica's
attributes, though other programming patterns may also be used the achieve an
equivalent behavior, such as class inheritance.

% Expression mutability

Unlike SymPy, Mathematica's expressions are mutable,
that is, one can change parts of the expression tree without
creating a new object.
The mutability of Mathematica expressions allows for a lazy updating of any references
to a given data structure.

% * comparison with Mathematica: commutativity, associative expressions, one-identity. Advantage of SymPy: multiplicative commutativity defined on symbols.
% Products and commutativity

Products in Mathematica are determined by some built in node types, such as
\texttt{Times}, \texttt{Dot}, and others.  \texttt{Times} is a representation of
the \texttt{*} operator, and is always meant to represent a commutative product
operator.  The other notable product is \texttt{Dot}, which represents the
\texttt{.} operator.  This product represents matrix multiplication, it is not
commutative.  In general, SymPy uses the same node for both scalar and matrix
multiplication, the only exception being with abstract matrix symbols.  Unlike
Mathematica, SymPy determines commutativity with respect to multiplication from
the factor's expression type.  Mathematica puts the \texttt{Orderless} attribute
on the expression type.

% Associative expressions.

Regarding associative expressions,
SymPy handles associativity by making associative expressions inherit the
class \texttt{AssocOp},
while Mathematica specifies the \texttt{Flat}~\cite{WolframRefFlat} attribute on the expression type.

% One identity


% Pattern matching

Mathematica relies heavily on pattern matching---even the so-called equivalent of function declaration is in reality
the definition of a pattern matching generating an expression tree transformation
on input expressions.
%
Mathematica's pattern matching is sensitive to
associative~\cite{WolframRefFlat}, commutative~\cite{WolframRefOrderless}, and
one-identity~\cite{WolframRefOneIdentity} properties of its expression tree
nodes~\cite{WolframRefFlatAndOrderlessFunctions}.
%
SymPy has various ways to perform pattern matching.
All of them play a lesser role in the CAS than in Mathematica
and are basically available as a tool to rewrite expressions.
The differential equation solver in SymPy somewhat relies on pattern matching to
identify the kind of differential equation, but it is envisaged to replace
that strategy with analysis of Lie symmetries in the future.
Mathematica's real advantage is the ability to add new overloading to the
expression builder at runtime, or for specific subnodes.
Consider for example:
\begin{verbatim}
In[1]:= Unprotect[Plus]

Out[1]= {Plus}

In[2]:= Sin[x_]^2 + Cos[y_]^2 := 1

In[3]:= x + Sin[t]^2 + y + Cos[t]^2

Out[3]= 1 + x + y
\end{verbatim}
This expression in Mathematica defines a substitution rule that overloads
the functionality of the \texttt{Plus} node (the node for additions in Mathematica).
The trailing underscore after a symbol means that it is to be considered a
wildcard.
This example may not be practical, one may wish to keep this identity
unevaluated.  Nevertheless, it clearly illustrates the potential to define
one's own immediate transformation rules.
In SymPy, the operations constructing the addition node in the expression tree
are Python class constructors
and cannot be modified at runtime.\footnote{In reality, Python supports monkey patching,
nonetheless, it is a discouraged programming pattern.}
The way SymPy deals with extending the missing runtime overloadability functionality
is by subclassing the node types.
Subclasses may redefine the class constructor to yield the proper
extended functionality.


%% TODO list:
% * comparison with Mathematica: MatrixExp, product not always commutative, type inheritance (polymorphism) and advantage in unifying the product symbol * for symbols and matrices, pattern matching vs. single dispatch.

% Type inheritance and polymorphism

Unlike SymPy, Mathematica does not support type inheritance or poly\-morph\-ism~\cite{Fateman1992}.
% cite examples of class inheritance in SymPy:
%
SymPy relies heavily on class inheritance, but for the most part,
class inheritance is used to make sure that SymPy objects inherit the proper
methods and implement the basic hashing system.
Associativity of expressions can be achieved by inheriting the class \texttt{AssocOp},
which may appear a more cumbersome operation than Mathematica's attribute setting.
%There are also cases where inheritance is used to extend the mathematical meaning of an expression.

% Matrices

Matrices in SymPy are types on their own.
In Mathematica, nested lists are interpreted as matrices whenever the sublists
have the same length.
The main difference to SymPy is that ordinary operators and functions
do not get generalized the same way as used in traditional mathematics.
Using the standard multiplication in Mathematica performs an element-wise
product, this is compatible with Mathematica's convention of commutativity of
\texttt{Times} nodes.
Matrix product is expressed by the \textit{dot} operator,
or the \texttt{Dot} node.
The same is true for the other operators, and even functions,
most notably calling the exponential function \texttt{Exp} on a matrix
returns an element-wise exponentiation of its elements.
The real matrix exponential is available through the \texttt{MatrixExp}
function.

% * comparison with Mathematica: avoid misspelling variables through forced declaration (check that you can't do it in Mathematica).
% * evaluate=False vs HoldForm

Unevaluated expressions in Mathematica can be achieved in various ways,
most commonly with the \texttt{HoldForm} or \texttt{Hold} nodes,
that block the evaluation of subnodes by the parser.
Note that such a node cannot be expressed in Python, because of greedy evaluation.
Whenever needed in SymPy, it is necessary to add the parameter \texttt{evaluate=False}
to all subnodes, or put the input expression in a string.

% * comparison with Mathematica: == is structural equality, not

In Mathematica, the operator \texttt{==} returns a boolean whenever it is able
to immediately evaluate the truth of the equality, otherwise it returns an
\texttt{Equal} expression.  In SymPy, \texttt{==} means structural equality and
is always guaranteed to return a boolean expression.  To express an equality in
SymPy it is necessary to explicitly construct an object of the \texttt{Equality}
class.

% * comparison with Mathematica: polynomial module.
% * comparison with Mathematica: space is product, ** vs ^

SymPy, in accordance with Python and unlike the usual programming convention,
uses \texttt{**} to express the power operator, while Mathematica uses the more
common \verb|^|.

% * comparison with Mathematica: ( ) is Sequence, functions are generally uppercase.
% * comparsion with Mathematica: table of comparison?
% * comparison with Mathematica: Wolfram language has loads of operator overloading, functional paradigm.

SymPy's use of floating-point numbers is similar to that of most
other CASs, including Maple and Maxima.
By contrast, Mathematica uses a form
of significance arithmetic~\cite{Sofroniou2005precise} for approximate numbers.
This offers further protection against numerical errors,
although it comes with its own set of problems
(for a critique of significance arithmetic, see Fateman~\cite{Fateman1992}).
Internally, SymPy's \texttt{evalf} method works similarly to Mathematica's
significance arithmetic, but the semantics are isolated from the rest of the system.
