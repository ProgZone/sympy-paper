SymPy is a robust computer algebra system that provides a wide spectrum of
features both in traditional computer algebra and in a plethora of scientific
disciplines. SymPy can be used in a first-class way with other
Python projects, including the scientific Python stack.


SymPy supports a wide array of mathematical facilities. This includes
functions for assuming and deducing common mathematical facts,
simplifying expressions, performing common calculus operations, manipulating polynomials, pretty printing
expressions, solving equations, and representing symbolic matrices. Other supported
facilities
include discrete math, concrete math, plotting, geometry, statistics,
sets, series, vectors, combinatorics, group theory, code
generation, tensors, Lie algebras, cryptography, and special functions.
SymPy has strong support for arbitrary precision numerics, backed by the
mpmath package. Additionally, SymPy contains submodules targeting certain specific physics domains,
such as classical mechanics and quantum mechanics.  This breadth of domains has
been engendered by a strong and vibrant user community.
Anecdotally, many of these users chose SymPy because of its ease of access.
SymPy a dependency of many external projects across a wide
spectrum of domains.

SymPy expressions are immutable trees of Python objects. Unlike many other
CAS's, SymPy is designed to be used in an extensible way: both as an end-user
application and as a library. SymPy uses Python both as the internal language
and the user language. This permits users to access to the same methods used
by the library itself in order to extend it for their needs.

% Future work:

Some of the planned future work for SymPy includes work on improving code
generation, improvements to the speed of SymPy using SymEngine, improving the
assumptions system, and improving the solvers submodule.

% TODO: Maybe one sentence for each item

% Feel free to add stuff here.
