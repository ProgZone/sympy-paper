SymPy is a robust computer algebra system that provides a wide array of
features both in traditional computer algebra and in broad scientific
disciplines. It is written in the general purpose Python language
which allows it to be used in a first-class way with other Python projects,
including the scientific Python stack. SymPy is designed to be used in an
extensible way and, unlike many other CASs, both as an end-user application and
as a library.

SymPy expressions are immutable trees of Python objects. SymPy uses Python both
as the internal language and the user language, meaning users can use the same
methods that the library implements to extend it. SymPy has an assumptions
system for declaring and deducing mathematical properties on expressions.

SymPy has submodules for many areas of mathematics. It has functions for
simplifying expressions, doing common calculus operations, pretty printing
expressions, solving equations, and symbolic matrices. Other included areas
are discrete math, concrete math, plotting, geometry, statistics,
polynomials, sets, series, vectors, combinatorics, group theory, code
generation, tensors, Lie algebras, cryptography, and special functions.
Additionally, SymPy contains submodules targeting certain specific domains,
such as classical mechanics and quantum mechanics.  This breadth of domains is
due to a strong and vibrant user community that were attracted to SymPy because
of its ease of access.

% Future work:

Some of the planned future work for SymPy includes work on improving code
generation, improvements to the speed of SymPy, and improving the solvers
module.
% Feel free to add stuff here.

% TODO: Mention SymEngine.
