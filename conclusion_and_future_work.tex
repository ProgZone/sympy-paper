SymPy is a robust CAS that provides a wide array of features. It is written in
a general purpose programming language, Python, which allows it to be used in
a first-class way with other Python projects, including the scientific Python
stack. It is designed to be used in an extensible way. Unlike many other CASs,
it is designed to be used both as a end-user application and as a library.

SymPy expressions are built from immutable trees of Python classes. It uses
Python both as the internal language and the user language, meaning users can
use the same methods that the library implements to extend it. SymPy has an
assumptions system for declaring and deducing mathematical properties on
expressions.

The numerics of SymPy are implemented in the mpmath library, which uses
arbitrary precision floating point arithmetic implemented in pure Python. This
allows expressions to be evaluated with concrete data as needed.

SymPy has submodules for many areas of mathematics. It has functions for
simplifying expressions, doing common calculus operations, pretty printing
expressions, solving equations, and symbolic matrices. Other included areas
are discrete math, concrete math, plotting, geometry, statistics,
polynomials, sets, series, vectors, combinatorics, group theory, code
generation, tensors, Lie algebras, cryptography, and special functions.
Additionally, SymPy contains submodules targeting certain specific domains,
such as classical mechanics and quantum mechanics.

% Future work:

Some of the planned future work for SymPy includes work on improving code
generation, improvements to the speed of SymPy, and improving the solvers
module.
% Feel free to add stuff here.

% TODO: Mention SymEngine.
