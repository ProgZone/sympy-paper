SymPy includes several packages that allow users to solve domain specific
problems. For example, a comprehensive physics package is included that is
useful for solving problems in classical mechanics, optics, and quantum
mechanics along with support for manipuating physical quantities with units.

\subsection{Vector Algebra}

The vector package provides reference frame, time, and space aware vector and
dyadic objects that allow for three dimensional operations such as addition,
subtraction, scalar multiplication, inner and outer products, cross products,
etc. Both of these objects can be written in very compact notation that make it
easy to express the vectors and dyadics in terms of multiple reference frames
with arbitrarily defined relative orientations. The vectors are used to specify
the positions, velocities, and accelerations of points, orientations, angular
velocities, and angular accelerations of reference frames, and force and
torques. The dyadics are essentially reference frame aware 3 x 3 tensors. The
vector and dyadic objects can be used for any one, two, or three dimensional
vector algebra and provide a strong framework to build physics and engineering
tools atop.

\begin{listing}
  \begin{minted}{pycon}
>>> A = ReferenceFrame('A')
>>> B = ReferenceFrame('B')
>>> C = ReferenceFrame('C')
>>> B.orient(A, 'body', (1, 2, 3), 'xzx')
>>> C.orient(B, 'axis', (4, B.x))
>>> v = 1 * A.x + 2 * B.z + 3 * C.y
>>> v
A.x + 2*B.z + 3*C.y
>>> v.express(A)
(-3*sin(2)*cos(3)*cos(4) + 3*sin(2)*sin(3)*sin(4) + 2*sin(2)*sin(3) + 1)*A.x + (3*(-sin(3)*cos(1)*cos(2) - sin(1)*cos(3))*sin(4) + 3*(-sin(1)*sin(3) + cos(1)*cos(2)*cos(3))*cos(4) - 2*sin(3)*cos(1)*cos(2) - 2*sin(1)*cos(3))*A.y + (2*cos(1)*cos(3) + 3*(sin(3)*cos(1) + sin(1)*cos(2)*cos(3))*cos(4) - 2*sin(1)*sin(3)*cos(2) + 3*(cos(1)*cos(3) - sin(1)*sin(3)*cos(2))*sin(4))*A.z
  \end{minted}
  \caption{Python interpreter session showing how a vector is created using the
    orthogonal unit vectors of three reference frames that are oriented with
    respect to each other and the result of expressing the vector in one frame.}
  \label{lis:physics-vector}
\end{listing}

\subsection{Classical Mechanics}

The \verb|physics.mechanics| package utilizes the \verb|physics.vector| package
to populate time aware particle and rigid body objects to fully describe the
kinematics and kinetics of a rigid multi-body system. These objects store all
of the information needed to derive the ordinary differential or differential
algrebraic equations that govern the motion of the system, i.e. the equations
of motion. These equations of motion abide by Newton's laws of motion and can
handle any arbitrary kinematical constraints or complex loads. The package
offers two automated methods for formulating the equations of motion based on
Lagrangian Dynamics~\cite{XXX} and Kane's Method~\cite{Kane1985}. Lastly, there
are autmated linearization routines for constrained dynamical
systems based on ~\cite{Peterson2015}.

\subsection{Quantum Mechanics}

The \verb|sympy.physics.quantum| package provides quantum functions, states,
operators, and computation of standard quantum models.

% TODO : This needs some help from someone that knows something about quantum
% physics. I wasn't able to understand much from the documentation.

\subsection{Optics}

The \verb|physics.optics| package provides Gaussian optics functions.

% TODO : This needs some help from someone that knows something about optics.

\subsection{Units}

The \verb|physics.units| module provides around two hundred predefined prefixes
and SI units that are commonly used in the sciences. Additionally, it provides
the \verb|Unit| class which allows the user to define their own units.  These
prefixes and units are multiplied by standard SymPy objects to make expressions
unit aware, allowing for algebraic and calculus manipulations to be applied to
the expressions while the units are tracked in the manipulations.  The units of
the expressions can be easily converted to other desired units.  There is also
a new units system in \verb|sympy.physics.unitsystems| that allows the user to
work in specified unit systems.
