
\subsection{Simplification}
\begin{itemize}
\item \texttt{expand}:
\begin{verbatim}
>>> expand((x + y)**3)
x**3 + 3*x**2*y + 3*x*y**2 + y**3
\end{verbatim}

\item \texttt{factor}:
\begin{verbatim}
>>> factor(x**3 + 3*x**2*y + 3*x*y**2 + y**3)
(x + y)**3
\end{verbatim}

\item \texttt{collect}:
\begin{verbatim}
>>> collect(y*x**2 + 3*x**2 - x*y + x - 1, x)
x**2*(y + 3) + x*(-y + 1) - 1
\end{verbatim}

\item \texttt{cancel}:
\begin{verbatim}
>>> cancel((x**2 + 2*x + 1)/(x**2 - 1))
(x + 1)/(x - 1)
\end{verbatim}

\item \texttt{apart}:
\begin{verbatim}
>>> apart((x**3 + 4*x - 1)/(x**2 - 1))
x + 3/(x + 1) + 2/(x - 1)
\end{verbatim}

\item \texttt{trigsimp}:
\begin{verbatim}
>>> trigsimp(cos(x)**2*tan(x) - sin(2*x))
-sin(2*x)/2
\end{verbatim}
\end{itemize}

\subsection{Polynomials}
%% TODO - explain why these matter
\begin{itemize}
\item Factorization:
\begin{verbatim}
>>> t = symbols('t')
>>> f = (2115*x**4*y + 45*x**3*z**3*t**2 - 45*x**3*t**2 -
...      423*x*y**4 - 47*x*y**3 + 141*x*y*z**3 + 94*x*y*z*t -
...      9*y**3*z**3*t**2 + 9*y**3*t**2 - y**2*z**3*t**2 +
...      y**2*t**2 + 3*z**6*t**2 +  2*z**4*t**3 - 3*z**3*t**2 -
...      2*z*t**3)
>>> factor(f)
(t**2*z**3 - t**2 + 47*x*y)*(2*t*z + 45*x**3 - 9*y**3 - y**2 +
 3*z**3)
\end{verbatim}

\item Gr\"{o}bner bases:
\begin{verbatim}
>>> x0, x1, x2 = symbols('x:3')
>>> I = [x0 + 2*x1 + 2*x2 - 1,
...      x0**2 + 2*x1**2 + 2*x2**2 - x0,
...      2*x0*x1 + 2*x1*x2 - x1]
>>> groebner(I, order='lex')
GroebnerBasis([7*x0 - 420*x2**3 + 158*x2**2 + 8*x2 - 7,
7*x1 + 210*x2**3 - 79*x2**2 + 3*x2,
84*x2**4 - 40*x2**3 + x2**2 + x2], x0, x1, x2, domain='ZZ',
order='lex')
\end{verbatim}

\item Root isolation:
\begin{verbatim}
>>> f = 7*z**4 - 19*z**3 + 20*z**2 + 17*z + 20
>>> intervals(f, all=True, eps=0.001)
([],
 [((-425/1024 - 625*I/1024, -1485/3584 - 2185*I/3584), 1),
  ((-425/1024 + 2185*I/3584, -1485/3584 + 625*I/1024), 1),
  ((3175/1792 - 2605*I/1792, 1815/1024 - 10415*I/7168), 1),
  ((3175/1792 + 10415*I/7168, 1815/1024 + 2605*I/1792), 1)])
\end{verbatim}
\end{itemize}
\subsection{Solvers}

\begin{itemize}
\item Single solution:
\begin{verbatim}
>>> solveset(x - 1, x)
{1}
\end{verbatim}

\item Finite solution set, quadratic equation:
\begin{verbatim}
>>> solveset(x**2 - pi**2, x)
{-pi, pi}
\end{verbatim}

\item No solution:
\begin{verbatim}
>>> solveset(1, x)
EmptySet()
\end{verbatim}

\item Interval solution:
\begin{verbatim}
>>> solveset(x**2 - 3 > 0, x, domain=S.Reals)
(-oo, -sqrt(3)) U (sqrt(3), oo)
\end{verbatim}

\item Infinitely many solutions:
% SymPy 1.0 sstr() prints S.Complexes incorrectly
% no-doctest
\begin{verbatim}
>>> solveset(x - x, x, domain=S.Reals)
(-oo, oo)
>>> solveset(x - x, x, domain=S.Complexes)
S.Complexes
\end{verbatim}

\item Linear systems (\texttt{linsolve})
\begin{verbatim}
>>> A = Matrix([[1, 2, 3], [4, 5, 6], [7, 8, 10]])
>>> b = Matrix([3, 6, 9])
>>> linsolve((A, b), x, y, z)
{(-1, 2, 0)}
>>> linsolve(Matrix(([1, 1, 1, 1], [1, 1, 2, 3])), (x, y, z))
{(-y - 1, y, 2)}
\end{verbatim}
\end{itemize}

Below are examples of \texttt{solve} applied to problems not yet handled by \texttt{solveset}.

\begin{itemize}
\item Nonlinear (multivariate) system of equations (the intersection of a circle
and a parabola):
\begin{verbatim}
>>> solve([x**2 + y**2 - 16, 4*x - y**2 + 6], x, y)
[(-2 + sqrt(14), -sqrt(-2 + 4*sqrt(14))),
 (-2 + sqrt(14), sqrt(-2 + 4*sqrt(14))),
 (-sqrt(14) - 2, -I*sqrt(2 + 4*sqrt(14))),
 (-sqrt(14) - 2, I*sqrt(2 + 4*sqrt(14)))]
\end{verbatim}

\item Transcendental equations:
\begin{verbatim}
>>> solve((x + log(x))**2 - 5*(x + log(x)) + 6, x)
[LambertW(exp(2)), LambertW(exp(3))]
>>> solve(x**3 + exp(x))
[-3*LambertW((-1)**(2/3)/3)]
\end{verbatim}
\end{itemize}

\subsection{Matrices}

\begin{itemize}
\item Matrix expressions
\begin{verbatim}
>>> m, n, p = symbols('m n p', integer=True)
>>> R = MatrixSymbol('R', m, n)
>>> S = MatrixSymbol('S', n, p)
>>> T = MatrixSymbol('T', m, p)
>>> U = R*S + 2*T
>>> U.shape
(m, p)
>>> U[0, 1]
2*T[0, 1] + Sum(R[0, _k]*S[_k, 1], (_k, 0, n - 1))
\end{verbatim}

\item Block Matrices
\begin{verbatim}
>>> n, m, l = symbols('n m l')
>>> X = MatrixSymbol('X', n, n)
>>> Y = MatrixSymbol('Y', m ,m)
>>> Z = MatrixSymbol('Z', n, m)
>>> B = BlockMatrix([[X, Z], [ZeroMatrix(m, n), Y]])
>>> B
Matrix([
[X, Z],
[0, Y]])
>>> B[0, 0]
X[0, 0]
>>> B.shape
(m + n, m + n)
\end{verbatim}
\end{itemize}
