% Features to discuss in-depth:

SymPy has an extensive feature set that encompasses too much to cover
in-depth here. Bedrock areas, such a Calculus, receive their own sub-sections
below. Additionally, Table \ref{features-table} describes other capabilities
present in the SymPy code base. This gives a sampling from the breadth of
topics and application domains that SymPy services.

\begin{table}
\label{features-table}
\caption{SymPy Features and Descriptions}
\begin{tabular}[htbc]{|l|p{0.7\linewidth}|}
\hline
\textbf{Feature} & \textbf{Description} \\
\hline
Discrete Math & Summations, products, binomial coefficients,
    prime number tools, integer factorization, Diophantine equation solving, and
    boolean logic representation, equivalence testing, and inference.\\
Concrete Math & Tools for determining whether summation and product
    expressions are convergent, absolutely convergent, hypergeometric, and
    other properties. May also compute Gosper's normal form
    \cite{petkovvsek1996bak} for two univariate polynomials.\\
Plotting & Hooks for visualizing expressions via matplotlib \cite{Hunter:2007}
    or as text drawings when lacking a graphical back-end.\\
Geometry & Allows the creation of 2D geometrical entities,
    such as lines and circles. Enables queries on these entities, including
    asking the area of an ellipse, checking for collinearity of a set of
    points, or finding the intersection between two lines.\\
Statistics & Support for a random variable type as well as the ability to
    declare this variable from prebuilt distribution functions such as
    Normal, Exponential, Coin, Die, and other custom distributions.\\
Polynomials & Computes polynomial algebras over various coefficient domains
    ranging from the simple (e.g. polynomial division) to the advanced
    (e.g. Gr\"obner bases \cite{adams1994introduction} and multivariate
    factorization over algebraic number domains).\\
Sets & Representations of empty, finite, and infinite sets. This includes
    special sets such as for all natural, integer, and complex numbers.\\
Series & Implements series expansion, sequences, and limit of sequences.
    This includes special series, such as Fourier and power series.\\
Vectors & Provides basic vector math and differential calculus with respect
    to 3D Cartesian coordinate systems.\\
Matrices & Tools for creating matrices of symbols and expressions.
    This is capable of both sparse and dense representations and performing
    symbolic linear algebraic operations (e.g. inversion and factorization).\\
Combinatorics \& Group Theory & Implements permutations, combinations,
    partitions, subsets,
    various permutation groups (such as polyhedral, Rubik, symmetric,
    and others), Gray codes \cite{Nijenhuis1978combinatorial},
    and Prufer sequences \cite{biggs1976graph}.\\
Code Generation & Enables generation of compilable and executable
    code in a variety of different programming languages directly from
    expressions. Target languages include C, Fortran, Julia, JavaScript,
    Mathematica, Matlab and Octave, Python, and Theano.\\
Tensors & Symbolic manipulation of indexed objects.\\
Lie Algebras & Represents Lie algebras and root systems.\\
Cryptography & Represents block and stream ciphers, including
    shift, Affine, substitution, Vigenere's, Hill's, bifid, RSA, Kid RSA,
    linear-feedback shift registers, and Elgamal encryption\\
Special Functions & Implements a number of well known special functions,
    including Dirac delta, Gamma, Beta, Gauss error functions, Fresnel
    integrals, Exponential integrals, Logarithmic integrals, Trigonometric
    integrals, Bessel, Hankel, Airy, B-spline, Riemann Zeta, Dirichlet eta,
    polylogarithm, Lerch transcendent, hypergeometric, elliptic integrals,
    Mathieu, Jacobi polynomials, Gegenbauer polynomial, Chebyshev polynomial,
    Legendre polynomial, Hermite polynomial, Laguerre polynomial, and
    spherical harmonic functions.\\
\hline
\end{tabular}
\end{table}


% Basic operations (the core)
\subsection{Basic Operations}


\subsubsection{Expression manipulation}

% symbols, various ways to declare them

Symbols are instances of the class \texttt{Symbol}.
They may be declared by invoking the class constructor with the symbol string
representation, or through the faster \texttt{symbols} and \texttt{var}
functions.

% polynomial expressions

% functions

% expand( ), factor( ), collect( ), together( ), apart( )
%% maybe a table best suits this part.
Common functions for polynomial expression manipulations are listed in the
following table:
\begin{tabular}{l|p{0.7\linewidth}}
expand & expand the expression \\
factor & recognize factors \\
collect & . \\
together & . \\
apart & . \\
\end{tabular}

% simplification: simplify, sqrt denest, fu, trigsimp

The generic way to simplify an expression is by calling the \texttt{simplify}
function, or equivalently, calling it as a method \texttt{expr.simplify()}.
It must be emphasized that simplification is not an unambigously defined
mathematica operation, nevertheless full simplification may require a huge
amount of computational power.

There are specific algorithms for special simplification cases,
such as \texttt{fu}, which calls a powerful simplification algorithm for
trigonometric exressions~\cite{fu2006automated}.
For trigonometric expressions there is furthermore a \texttt{trigsimp} method,
acting as a wrapper for specific algorithms.
\texttt{sqrtdenest} may help by denesting square roots inside other square roots.


% substitution, wilds, replacement

Substitutions are performed through the \texttt{.subs} method, which accepts
wildcards and is sensible to some mathematical properties while matching,
such as associativity, commutativity,  additive and multiplicative inverses,
and matching of powers.

\texttt{.replace} provides more basic matching algorithm,
though it allows for costum matching functions to be passed to it.

\texttt{.xreplace} is an expression tree structural replacement routing,
unaware of any mathematical property.

%% TODO: add examples

% unevaluated expressions (evaluate=False), global_evaluate

Expression constructors accept in most cases the boolean parameter \texttt{evaluate},
setting it to false will prevent automatic evaluation of the expression.
The \verb|global_evaluate| variable may be employed to globally block any kind
of evaluation.


\subsubsection{Assumptions system}

% Including the assumptions system


SymPy has two assumptions systems, referred to as new-style and old-style assumptions.

In the old-style assumptions system propositions are assigned to symbols
upon class construction, for example, to declare the symbol $i$ as positive integer,
one would call

\begin{verbatim}
i = Symbol("i", integer=True, positive=True)
\end{verbatim}

querying the assumptions is handled through attributes

\begin{verbatim}
i.is_positive
i.is_integer
\end{verbatim}

These methods return either a boolean, indicating whether the preposition is true or false,
or a None, when it is impossible to determine the truth value of the queried preposition.

Despite the fact that assumptions can only be declared on symbols, querying can
happen on every expression.

\begin{verbatim}
In [1]: x,y = symbols('x y', positive=True)

In [2]: (x*y).is_positive
Out[2]: True

In [3]: z = symbols('z')

In [4]: (x*z).is_positive

In [5]: w = symbols('w', positive=False)

In [6]: (x*w).is_positive
Out[6]: False
\end{verbatim}

The output 2 is true because SymPy's algorithms can deduce that the product of
two positive numbers is positive, while there is no output for input 4, as the
symbol $z$ doesn't have any information about its sign, and the product
$x\cdot z$ may be positive as well as negative.
Finally, output 6 is false as the product of positive and negative numbers is
negative.

The new-style assumptions are an assumptions system that exists alongside with
the old-style, but is significantly different in the way predicates are
used.
Predicates in the new-style assumptions system are located under the \textit{Q}
namespace, they appear as
\verb|Q.positive|, \verb|Q.integer| and so on.

Querying is provided through the \verb|ask| functions.
The previous example in the new-style assumptions can be written as
\begin{verbatim}
In [1]: ask(Q.positive(x*y), Q.positive(x) & Q.positive(y))
Out[1]: True

In [2]: ask(Q.positive(x*y), Q.positive(x))

In [3]: ask(Q.positive(x*y), Q.positive(x) & Q.negative(y))
Out[3]: False
\end{verbatim}
%
That is, \verb|ask| returns the truth value of its first parameter assuming
that its latter argument is true.

Expressions like \verb|Q.positive| are instances of the class \verb|Predicate|,
while the same expression with a parameter, such as \verb|Q.positive(x)| is an
instance of \verb|AppliedPredicate|.

Logical connectors can be expressed through operator overloading,
such as in \verb|Q.positive(x) & Q.positive(y)|,
or by directly constructing the identical expression through the 
logical connector class, in this case \verb|And(Q.positive(x), Q.positive(y))|.


\subsubsection{Calculus}

% Calculus (differentiation, integration, limits). Note that algorithm
% descriptions will go in the algorithms section.

Derivations can be computed with the \verb|diff| function,
or using the method with the same name on the expressions:

\begin{verbatim}
In [1]: diff(sin(x), x)
Out[1]: cos(x)

In [2]: sin(x).diff(x)
Out[2]: cos(x)
\end{verbatim}

The class \verb|Derivative| is a container for unevaluated derivatives

\begin{verbatim}
In [3]: expr = Derivative(sin(x), x)

In [4]: expr
Out[4]: 
d         
--(sin(x))
dx        
\end{verbatim}

To evaluate such a held expression, simply call the \verb|doit| method:

\begin{verbatim}
In [5]: expr.doit()
Out[5]: cos(x)
\end{verbatim}

Integrals can be analogously calculated either with the \verb|integrate| function
or with the method with the same name on expressions:
\begin{verbatim}
>>> integrate(sin(x), x)
-cos(x)
\end{verbatim}
This expression returns an expression whose derivative is the original expression.
Notice that integrals are defined up to an integration constant,
for the sake of simplicity SymPy will not display the full generic expression.

Definite integration can be calculated with the same method, by specifying a
range of the integration variable:
\begin{verbatim}
>>> integrate(sin(x), (x, 0, 1))
-cos(1) + 1
\end{verbatim}

To express unevaluated integrals, the class \verb|Integral| may help
\begin{verbatim}
Integral(sin(x), x)
\end{verbatim}
as in the case of derivatives, the method \verb|doit| will cause such an expression
to be evaluated.

Limits:
\begin{verbatim}
In [9]: limit(sin(x)/x, x, 0)
Out[9]: 1
\end{verbatim}
for unevaluated expressions, \verb|Limit|.

TODO: right and left limits.

Sums and products are handled by the \texttt{Sum} and \texttt{Product} classes,
respectively.
Analogously with \texttt{Integral}, the first argument is the expression to be
summed over, whereas the following arguments represent the summation and 
multiplication indices, respectively, provided with integer ranges.

It may be noted the existence of the \texttt{IndexedBase} class,
which provides the construction of indexed symbols, that is symbols that are
treated as different if their indices are different.


\subsubsection{Expression outputs}

Alongside with its parsers, SymPy has a rich collection of expression printers.

% LaTeX printer
Expressions may be readily transformed into a LaTeX form with the \texttt{latex( )}
function.

% pretty printer
Pretty printer outputs the expression in traditional form with characters,
outputs can be visualized in monospace fonts.



\subsection{Calculus}

% Sets
\subsection{Sets}
%% Sets

SymPy supports representation of a wide variety of set, this is achieved by
first defining abstract representation for a smaller number of atomic set
classes and then combining and transforming them using various set operations.

Each of the set class inherits from the base Set class and defines
rules to check membership of a SymPy object, to calculate union, intersection,
set different. In cases we are not able to evaluate these
operations to atomic set classes they are represented as abstract unevaluated
objects.


\begin{itemize}

    % Description of Empty Set sounds too obvious, not sure if we need to keep
    % it.
    \item \textbf{Empty Set}: Nothing is a member of Empty Set. Union with
another set returns the other set and intersection leads to an Empty Set.

    \item \textbf{Universal Set}: Everything is a member of Universal Set.
Union of Universal Set with any set gives Universal Set and intersection leads
to the other set itself.

    \item \textbf{Finite Set} is functionally equivalent to python's set
object. Its members can be any object including strings and other sets
themselves.

    \item \textfb{Range} implements a range of integers and is defined by
specifying a start value, an end value and a step size. Range is functionally
equivalent to python's range except the fact that it accepts infinity at end
points allowing us to represent infinite ranges.


    \item \textbf{Real Interval} is specified by giving the start and end point
and specifying if it is open or closed in these respective ends. The set of
real numbers is represented as a special case of a real interval where the
start point is negative infinite and the end point is positive infinite.

\end{itemize}


%% Operations

Other than unevaluated classes of of Union, Intersection and Set Difference
operations, we have following set classes.

\begin{itemize}

    \item Product Set abstractly defines the Cartesian product of two or more
sets. Product Set is useful when representing higher dimensional spaces. For
example to represent a three dimensional space we simple take the Cartesian
product of three Real sets.

    \item Image Set represents the set of image a function when applied to a
particular set. In notation Image Set of a function F w.r.t a set S is \{ F(x)
| x \in S \} In particular we use Image Set to represent the set of infinite
solutions from trigonometric equations.


    \item Condition Set represents subset of a set who's members which
satisfies a particular condition. In notation Condition Set of set S w.r.t to a
condition H is \{x | H(x), x \in S \}. We use Condition Set to represent the
set of solutions of an equation or an inequality where the equation or the
inequality is the condition and the set is the domain in which we aim to find
the solution.


\end{itemize}





%% Explaining it later because it is a special case of Image Set rather being
%% something atomic

    \textbf{Complex Region}

%% Representations achievable through application of Operations on atomic set
%% types mentioned above.


%% Special Cases


% Solvers (regular equations, maybe also mention other types of solvers like ODEs/recurrence/Diophantine)
\subsection{Solvers}
SymPy includes a module of equation solvers for symbolic equations. There are two
functions for solving algebraic equations in SymPy: \texttt{solve}
and \texttt{solveset}.
\texttt{solveset} has several design changes with respect to the older
\texttt{solve} function. This distinction is present in order to resolve the
usability issues with the
previous \texttt{solve} function API while maintaining backward compatibility
with earlier versions of SymPy.
\texttt{solveset} only requires essential input information from the user.
The function signatures of \texttt{solve} and \texttt{solveset} are
\begin{verbatim}
solve(f, *symbols, **flags)
solveset(f, symbol, domain=S.Complexes)
\end{verbatim}
The \texttt{domain} parameter is typically either \texttt{S.Complexes} (the
default) or \texttt{S.Reals}; the latter causes \texttt{solveset} to only return real solutions.
Both functions implicitly assume that expressions are equal to 0. For
instance, \texttt{solveset(x - 1, x)} solves $x - 1 = 0$ for $x$.

It should be noted that \texttt{solve} has an inconsistent output API for different types
of inputs. For instance, depending on the input, it may return a Python
list or a Python dictionary. On the other hand,
\texttt{solveset} has a canonical output API: it always returns
a SymPy set object.

\texttt{solveset} is under active development as a planned replacement for
\texttt{solve}. There are certain features which are implemented in
\texttt{solve} that are not yet implemented in \texttt{solveset}. Notably,
these include nonlinear multivariate system and transcendental equations.
More examples of \texttt{solveset} and \texttt{solve} can be found in the
supplement.


Diophantine equations play a central and an important role in number theory.
A Diophantine equation has the form, $f(x_1, x_2, \ldots x_n) = 0$
where $n \geq 2$ and $x_1, x_2, \ldots x_n$ are integer variables. If we can find
$n$ integers $a_1, a_2, \ldots a_n$ such that $x_1 = a_1, x_2 = a_2, \ldots x_n = a_n$
satisfies the above equation, we say that the equation is solvable.

Currently, following five types of Diophantine equations can be solved using
SymPy's Diophantine module.

\begin{itemize}
    \item Linear Diophantine equations: $a_1x_1 + a_2x_2 + \cdots + a_{n}x_{n} = b$
    \item General binary quadratic equation: $ax^2 + bxy + cy^2 + dx + ey + f = 0$
    \item Homogeneous ternary quadratic equation: $ax^2 + by^2 + cz^2 + dxy + eyz + fzx = 0$
    \item Extended Pythagorean equation: $a_{1}x_{1}^2 + a_{2}x_{2}^2 + \cdots + a_{n}x_{n}^2 = a_{n+1}x_{n+1}^2$
    \item General sum of squares: $x_{1}^2 + x_{2}^2 + \cdots + x_{n}^2 = k$
\end{itemize}

When an equation is fed into Diophantine module, it factors the equation (if
possible) and solves each factor separately. Then all the results are combined to
create the final solution set. Following examples illustrate some of the basic
functionalities of the Diophantine module.

\begin{minted}{pycon}
>>> from sympy import symbols
>>> x, y, z = symbols("x, y, z", integer=True)

>>> diophantine(2*x + 3*y - 5)
set([(3*t_0 - 5, -2*t_0 + 5)])

>>> diophantine(2*x + 4*y - 3)
set()

>>> diophantine(x**2 - 4*x*y + 8*y**2 - 3*x + 7*y - 5)
set([(2, 1), (5, 1)])

>>> diophantine(x**2 - 4*x*y + 4*y**2 - 3*x + 7*y - 5)
set([(-2*t**2 - 7*t + 10, -t**2 - 3*t + 5)])

>>> diophantine(3*x**2 + 4*y**2 - 5*z**2 + 4*x*y - 7*y*z + 7*z*x)
set([(-16*p**2 + 28*p*q + 20*q**2, 3*p**2 + 38*p*q - 25*q**2, 4*p**2 - 24*p*q + 68*q**2)])

>>> from sympy.abc import a, b, c, d, e, f
>>> diophantine(9*a**2 + 16*b**2 + c**2 + 49*d**2 + 4*e**2 - 25*f**2)
set([(70*t1**2 + 70*t2**2 + 70*t3**2 + 70*t4**2 - 70*t5**2, 105*t1*t5, 420*t2*t5, 60*t3*t5, 210*t4*t5, 42*t1**2 + 42*t2**2 + 42*t3**2 + 42*t4**2 + 42*t5**2)])

>>> diophantine(a**2 + b**2 + c**2 + d**2 + e**2 + f**2 - 112)
set([(8, 4, 4, 4, 0, 0)])
\end{minted}


% Matrices (worth including to stress that they are symbolic)
\subsection{Matrices}



SymPy supports matrices with symbolic expressions as elements. There are two
types of matrices, Mutable and Immutable. Mutable classes are the default in
SymPy as mutability is important for performance, but it means that standard
matrices can not interact well with the rest of SymPy. This is because the
Basic object, from which most SymPy classes inherit, is immutable.

Immutable matrix classes inherit from Basic and can thus interact more
naturally with the rest of SymPy.

\begin{verbatim}
In [1]: from sympy import Matrix, symbols, MatrixSymbol

In [2]: x, y = symbols('x y', positive=True)

In [3]: t = Matrix(2, 2, [x, x + y, y, x])

In [4]: t

Out[4]:
Matrix([
[    x, x + y],
[    y,     x]])

In [5]: t[0, 1] = y

In [6]: t
Out[6]:
Matrix([
[x, y],
[y, x]])
\end{verbatim}

All SymPy matrix types can do linear algebra including matrix addition,
multiplication, exponentiation, computing determinant, solving linear
systems and computing inverses using LU decomposition, LDL decomposition,
Gauss-Jordan elimination, Cholesky decomposition, Moore-Penrose pseudoinverse,
adjugate matrix.

Eigenvalues are computed symbolically as well. Eigenvalues are computed by
generating the characteristic polynomial using the Berkowitz algorithm and
then solving it using polynomial routines. Diagonalizable matrices can be
diagonalized first to compute the eigenvalues.

\begin{verbatim}
In [10]: t.eigenvals()
Out[10]: {x - y: 1, x + y: 1}
\end{verbatim}

Internally these matrices store the elements as a list making it a dense
representation. For storing sparse matrices, SparseMatrix and 
ImmutableSparseMatrix classes can be used. Sparse matrix classes store
the elements in Dictionary of Keys (DoK) format.

SymPy also supports matrices with unknown dimension values. MatrixSymbol
represents a matrix with dimensions \verb|m, n| where \verb|m| and \verb|n|
can be symbols or integers. Matrix addition and multiplication, scalar
operations, matrix inverse and transpose are stored symbolically as
matrix expressions. Mutable matrices are converted to corresponding immutable
types before interacting with matrix expressions

\begin{verbatim}
In [11]: m, n, p = symbols("m, n, p", integer=True)

In [12]: r, s = MatrixSymbol("r", m, n), MatrixSymbol("s", n, p)

In [13]: u = r * s + 2*MatrixSymbol("t", m, p)

In [14]: u.shape
Out[14]: (m, p)

In [15]: u[0, 1]
Out[15]: 2*t[0, 1] + Sum(r[0, _k]*s[_k, 1], (_k, 0, n - 1))
\end{verbatim}

Block matrices are also supported in SymPy. BlockMatrix elements can be
any matrix expression which includes immutable matrices, matrix symbols and
block matrices. All functionalities of matrix expressions are also present in
BlockMatrix.


\begin{verbatim}
>>> from sympy import (MatrixSymbol, BlockMatrix, symbols,
...     Identity, ZeroMatrix, block_collapse)
>>> n, m, l = symbols('n m l')
>>> X = MatrixSymbol('X', n, n)
>>> Y = MatrixSymbol('Y', m ,m)
>>> Z = MatrixSymbol('Z', n, m)
>>> B = BlockMatrix([[X, Z], [ZeroMatrix(m,n), Y]])
>>> print(B)
Matrix([
[X, Z],
[0, Y]])
>>> print(B[0, 0])
X[0, 0]
\end{verbatim}

% Physics module (some sampling, to show that it is there)
\subsection{Physics}

% Logic module
\subsection{Logic}

SymPy supports construction and manipulation of boolean expressions
through the \texttt{logic} module. SymPy symbols can be used as 
propositional variables and also be substituted as \texttt{True} 
or \texttt{False}. A good number of manipulation features for boolean 
expressions have been implemented in the \texttt{logic} module.

\subsubsection{Constructing boolean expressions}

A boolean variable can be declared as a SymPy symbol. Python
operators \&, \textbar  and \textasciitilde are overloaded for logical \texttt{And}, 
\texttt{Or} and \texttt{negate}. Several others like \texttt{Xor},
\texttt{Implies} can be constructed with \^{}, \textgreater\textgreater respectively.  
The above are just a shorthand, expressions can also be constructed
by directly calling \texttt{And()}, \texttt{Or()}, \texttt{Not()},
\texttt{Xor()}, \texttt{Nand()}, \texttt{Nor()}, etc.
The boolean symbols can also be substituted \texttt{True} or \texttt{False}

\begin{verbatim}
>>> e = (x & y) | z
>>> e.subs({x: True, y: True, z: False})
True
\end{verbatim}

\subsubsection{CNF and DNF}

Any boolean expression can be converted to conjunctive normal 
form, disjunctive normal form and negation normal form. The 
API also permits to check if a boolean expression is in any 
of the above mentioned forms.

\begin{verbatim}
>>> to_cnf((A & B) | C)
And(Or(A, C), Or(B, C))
>>> to_dnf(A & (B | C))
Or(And(A, B), And(A, C))
>>> is_cnf((x | y) & z)
True
>>> is_dnf((x & y) | z) 
True
\end{verbatim}

\subsubsection{Simplification and Equivalence}

The module supports simplification of given boolean expression
by making deductions on it. Equivalence of two expressions can
also be checked. If so, it is possible to return the mapping of 
variables of two expressions so as to represent the 
same logical behaviour.

\begin{verbatim}
>>> e = a & (~a | ~b) & (a | c)
>>> simplify(e)
And(Not(b), a)
>>> e1 = a & (b | c)
>>> e2 = (x & y) | (x & z)
>>> bool_map(e1, e2)
(And(Or(b, c), a), {b: y, a: x, c: z})
\end{verbatim}

\subsubsection{SAT solving}

The module also supports satisfiability checking of a given
boolean expression. If satisfiable, it is possible to return 
a model for which the expression is satisfiable. The API also
supports returning all possible models. The SAT solver has 
a clause learning DPLL algorithm implemented with watch 
literal scheme and VSIDS heuristic\cite{moskewicz2008method}.

\begin{verbatim}
>>> satisfiable(a & (~a | b) & (~b | c) & ~c)
False
>>> satisfiable(a & (~a | b) & (~b | c) & c)
{b: True, a: True, c: True}
\end{verbatim}

