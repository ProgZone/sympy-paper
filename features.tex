% Features to discuss in-depth:

\begin{table}
\label{features-table}
\caption{SymPy Features and Descriptions}
\begin{tabular}[htbc]{|l|l|}
\hline
\textbf{Feature} & \textbf{Description} \\
\hline
Discrete Math & Summations, products, binomial coefficients, prime number
    tools, integer factorization, Diophantine equation solving, and
    boolean logic representation, equivalence testing, and inference.\\
Concrete Math & Tools for determining whether summation and product
    expressions are convergent, absolutely convergent, hypergeometric, and
    other properties. May also compute Gosper's normal form
    \cite{petkovvsek1996bak} for two univariate polynomials.\\
Plotting & Hooks for visualizing expressions via matplotlib \cite{Hunter:2007}
    or as text drawings when lackinga graphical backend.\\
geometry & \\
statistics & \\
Polynomials & \\
Sets & \\
Series & \\
Vectors & \\
Matrices & \\
combinatorics & \\
Group Theory & \\
Code Generation & \\
Tensors & \\
Lie Algebras & \\
Cryptography & \\
category theory & \\
special functions & \\
\hline
\end{tabular}
\end{table}


% Basic operations (the core)
\subsection{Basic Operations}
% Including the assumptions system

% Calculus (differentiation, integration, limits). Note that algorithm
% descriptions will go in the algorithms section.
\subsection{Calculus}

% Solvers (regular equations, maybe also mention other types of solvers like ODEs/recurrence/Diophantine)
\subsection{Solvers}

% Matrices (worth including to stress that they are symbolic)
\subsection{Matrices}

% Physics module (some sampling, to show that it is there)
\subsection{Physics}

