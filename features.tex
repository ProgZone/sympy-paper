% Features to discuss in-depth:

% Basic operations (the core)
\subsection{Basic Operations}
% Including the assumptions system

% Calculus (differentiation, integration, limits). Note that algorithm
% descriptions will go in the algorithms section.
\subsection{Calculus}

% Solvers (regular equations, maybe also mention other types of solvers like ODEs/recurrence/Diophantine)
\subsection{Solvers}

% Matrices (worth including to stress that they are symbolic)
\subsection{Matrices}

% Physics module (some sampling, to show that it is there)
\subsection{Physics}

% Series module (Formal Power Series, Fourier Series)
\subsection{Series}

% Series expansion (Differentiate between the two approaches being used)
\subsubsection{Series Expansion}

SymPy has several different implementations for representing series expansions.

TODO

\subsubsection{Formal Power Series}

SymPy can be used for computing the Formal Power Series of a function.
The implementation is based on the algorithm described in the paper on Formal Power Series\cite{Gruntz93formalpower}.
The advantage of this approach is that an explicit formula for the coefficients
of the series expansion is generated rather than just computing a few terms.

\begin{verbatim}
>>> from sympy import fps, symbols, sin
>>> x, y = symbols('x')
>>> f = fps(sin(x), x, x0=0)
>>> f.truncate(6)
x - x**3/6 + x**5/120 + O(x**6)
>>> f[15]
-x**15/1307674368000
\end{verbatim}

\subsubsection{Fourier Series}

SymPy provides functionality to compute Fourier Series of a function using
the \texttt{fourier_series} function. Under the hood it just computes $a0$, $an$, $bn$ using
standard integration formulas.

\begin{verbatim}
>>> from sympy import fourier_series, symbols, Heaviside
>>> x, L = symbols('x L')
>>> f = fourier_series(2 * (Heaviside(x/L) - Heaviside(x/L - 1)) - 1, (x, 0, 2*L))
>>> f.truncate(3)
4*sin(pi*x/L)/pi + 4*sin(3*pi*x/L)/(3*pi) + 4*sin(5*pi*x/L)/(5*pi)
\end{verbatim}

% Features to list, but not discuss in-depth:

% discrete math, concrete math, plotting, geometry, statistics, polys,
% combinatorics/group theory, code generation, tensors, lie algebras,
% cryptography, category theory, special functions, sets, matrix expressions,
% series, or vectors.
\subsection{Other features}
