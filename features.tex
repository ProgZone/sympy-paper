% Features to discuss in-depth:

SymPy has an extensive feature set that encompasses too much to cover
in-depth here. Bedrock areas, such a Calculus, receive their own sub-sections
below. Additionally, Table~\ref{features-table} describes other capabilities
present in the SymPy code base. This gives a sampling from the breadth of
topics and application domains that SymPy services.


\begin{longtable}[htbc]{|l|p{0.7\linewidth}|}
\caption{SymPy Features and Descriptions\label{features-table}}\\
\hline
\textbf{Feature} & \textbf{Description} \\
\hline
Discrete Math & Summations, products, binomial coefficients,
    prime number tools, integer factorization, Diophantine equation solving, and
    boolean logic representation, equivalence testing, and inference.\\
Concrete Math & Tools for determining whether summation and product
    expressions are convergent, absolutely convergent, hypergeometric, and
    other properties. May also compute Gosper's normal form~\cite{petkovvsek1996bak} for two univariate polynomials.\\
Plotting & Hooks for visualizing expressions via matplotlib~\cite{Hunter:2007}
    or as text drawings when lacking a graphical back-end.\\
Geometry & Allows the creation of 2D geometrical entities,
    such as lines and circles. Enables queries on these entities, including
    asking the area of an ellipse, checking for collinearity of a set of
    points, or finding the intersection between two lines.\\
Statistics & Support for a random variable type as well as the ability to
    declare this variable from prebuilt distribution functions such as
    Normal, Exponential, Coin, Die, and other custom distributions.\\
Polynomials & Computes polynomial algebras over various coefficient domains
    ranging from the simple (e.g., polynomial division) to the advanced
    (e.g., Gr\"obner bases~\cite{adams1994introduction} and multivariate
    factorization over algebraic number domains).\\
Sets & Representations of empty, finite, and infinite sets. This includes
    special sets such as for all natural, integer, and complex numbers.\\
Series & Implements series expansion, sequences, and limit of sequences.
    This includes special series, such as Fourier and power series.\\
Vectors & Provides basic vector math and differential calculus with respect
    to 3D Cartesian coordinate systems.\\
Matrices & Tools for creating matrices of symbols and expressions.
    This is capable of both sparse and dense representations and performing
    symbolic linear algebraic operations (e.g., inversion and factorization).\\
Combinatorics \& Group Theory & Implements permutations, combinations,
    partitions, subsets,
    various permutation groups (such as polyhedral, Rubik, symmetric,
    and others), Gray codes~\cite{Nijenhuis1978combinatorial},
    and Prufer sequences~\cite{biggs1976graph}.\\
Code Generation & Enables generation of compilable and executable
    code in a variety of different programming languages directly from
    expressions. Target languages include C, Fortran, Julia, JavaScript,
    Mathematica, Matlab and Octave, Python, and Theano.\\
Tensors & Symbolic manipulation of indexed objects.\\
Lie Algebras & Represents Lie algebras and root systems.\\
Cryptography & Represents block and stream ciphers, including
    shift, Affine, substitution, Vigenere's, Hill's, bifid, RSA, Kid RSA,
    linear-feedback shift registers, and Elgamal encryption\\
Special Functions & Implements a number of well known special functions,
    including Dirac delta, Gamma, Beta, Gauss error functions, Fresnel
    integrals, Exponential integrals, Logarithmic integrals, Trigonometric
    integrals, Bessel, Hankel, Airy, B-spline, Riemann Zeta, Dirichlet eta,
    polylogarithm, Lerch transcendent, hypergeometric, elliptic integrals,
    Mathieu, Jacobi polynomials, Gegenbauer polynomial, Chebyshev polynomial,
    Legendre polynomial, Hermite polynomial, Laguerre polynomial, and
    spherical harmonic functions.\\
\hline

\end{longtable}

% Basic operations (the core)

% polynomial expressions

% functions

% expand( ), factor( ), collect( ), together( ), apart( )
%% maybe a table best suits this part.

% simplification: simplify, sqrt denest, fu, trigsimp

\subsection{Simplification}

The generic way to simplify an expression is by calling the \texttt{simplify}
function.
It must be emphasized that simplification is not an unambigously defined
mathematical operation~\cite{Carette2004understanding}.
The \texttt{simplify} function applies several simplification routines along
with some heuristics to make the output expression as ``simple'' as possible.

It is often preferable to apply more directed simplification functions. These
apply very specific rules to the input expression, and are often able to make
guarantees about the output (for instance, the \texttt{factor} function, given
a polynomial with rational coefficients in several variables, is guaranteed to
produce a factorization into irreducible factors).
Table~\ref{simplify-table} lists some common simplification functions.

% TODO: add a simple example for each
% TODO: fix the formatting
\label{simplify-table}
\begin{tabular}{l|p{0.7\linewidth}}
expand & expand the expression \\
factor & factor a polynomial into irreducibles \\
collect & collect polynomial coefficients \\
cancel & rewrite a rational function as $p/q$ with common factors canceled \\
apart & compute the partial fraction decomposition of a rational function \\
trigsimp & simplify trigonometric expressions~\cite{fu2006automated} \\
\end{tabular}

Substitutions are performed through the \texttt{.subs} method, which is
sensible to some mathematical properties while matching, such as
associativity, commutativity, additive and multiplicative inverses, and
matching of powers.

%% TODO: add examples

\subsection{Calculus}

% Calculus (differentiation, integration, limits). Note that algorithm
% descriptions will go in the algorithms section.

Derivatives can be computed with the \verb|diff| function.

\begin{verbatim}
>>> diff(sin(x), x)
cos(x)
\end{verbatim}

Unevaluated \verb|Derivative| objects are also supported.

\begin{verbatim}
>>> expr = Derivative(sin(x), x)
>>> expr
Derivative(sin(x), x)
\end{verbatim}

Unevaluated expressions can be evaluated with the \verb|doit| method.

\begin{verbatim}
>>> expr.doit()
cos(x)
\end{verbatim}

% TODO: A more interesting example here
Integrals can be analogously calculated either with the \verb|integrate|
function, or the unevaluated \verb|Integral| objects.
\begin{verbatim}
>>> integrate(sin(x), x)
-cos(x)
>>> expr = Integral(sin(x), x)
>>> expr
Integral(sin(x), x)
>>> expr.doit()
-cos(x)
\end{verbatim}
Definite integration can be calculated with the same method, by specifying a
range of the integration variable. The following computes $\int_0^1\sin(x)\,dx$.
\begin{verbatim}
>>> integrate(sin(x), (x, 0, 1))
-cos(1) + 1
\end{verbatim}

SymPy implements a combination of the Risch
algorithm~\cite{bronstein2005integration}, table lookups, a reimplementation
of Manuel Bronstein's ``Poor Man's Integrator''~\cite{Bronstein2005pmint}, and
an algorithm for computing integrals based on Meijer G-functions. These allow
SymPy to compute a wide variety of indefinite and definite integrals.
% TODO: What is the best citation for the Meijer G-function algorithm.
% TODO: Add some examples here.

Summations and products are also supported, via the evaluated \verb|summation|
and \verb|product| and unevaluated \verb|Sum| and \verb|Product|, and use the
same syntax as \verb|integrate|.

Summations are computed using a combination of Gosper's algorithm and an
algorithm that uses Meijer G-functions. Products are computed via some
heuristics.
% TODO: Citations for Gosper and Meijer G-function algorithms
% TODO: Are there other summation algorithms implemented?

\subsection{Limits}

SymPy calculates limits using the Gruntz algorithm, as described in%
~\cite{Gruntz1996limits}. The basic idea is as follows: any limit can be
converted to a limit $\lim\limits_{x\to\infty} f(x)$ by substitutions like
$x\to{1\over x}$. Then the most varying subexpression $\omega$ (that converges
to zero as $x\to\infty$ the fastest from all subexpressions) is identified in
$f(x)$, and $f(x)$ is expanded into a series with respect to $\omega$. Any
positive powers of $\omega$ converge to zero. If there are negative powers of
$\omega$, then the limit is infinite. The constant term (independent of
$\omega$, but could depend on $x$) then determines the limit (one might need to
recursively apply the Gruntz algorithm on this term to determine the limit).

To determine the most varying subexpression, the comparability classes must
first be defined, by calculating $L$:
\begin{equation}
L\equiv \lim_{x\to\infty} {\log |f(x)| \over \log |g(x)|}
\end{equation}
The relations $<$, $>$, and $\sim$ are defined as follows: $f>g$ when
$L=\pm\infty$ (it is said that $f$ is more rapidly varying than $g$, i.e., $f$
goes to $\infty$ or $0$ faster than $g$), $f<g$ when $L=0$ ($f$ is less
rapidly varying than $g$) and $f\sim g$ when $L\neq 0,\pm\infty$ (both $f$ and
$g$ are bounded from above and below by suitable integral powers of the
other). Note that if $f > g$, then $f > g^n$ for any $n$. Here
are some examples of comparability classes:
$$2 < x < e^x < e^{x^2} < e^{e^x}$$
$$2\sim 3\sim -5$$
$$x\sim x^2\sim x^3\sim {1\over x}\sim x^m\sim -x$$
$$e^x\sim e^{-x}\sim e^{2x}\sim e^{x+e^{-x}}$$
$$f(x)\sim{1\over f(x)}$$

The Gruntz algorithm is now illustrated on the following example:
\begin{equation}
    \label{gruntz_example_fn}
f(x) = e^{x+2e^{-x}} - e^x + {1\over x} \,.
\end{equation}
The goal is to calculate $\lim\limits_{x\to\infty} f(x)$.
First, the set of most rapidly varying subexpressions is determined---the so
called \textit{mrv set}. For~\eqref{gruntz_example_fn}, the mrv set
$\{e^x, e^{-x}, e^{x+2e^{-x}}\}$ is obtained. These are all subexpressions of%
~\eqref{gruntz_example_fn} and they all belong to the same comparability class.
This calculation can be done using SymPy as follows:

% dict_keys output order varies
% no-doctest
\begin{verbatim}
>>> from sympy.series.gruntz import mrv
>>> mrv(exp(x+2*exp(-x))-exp(x) + 1/x, x)[0].keys()
dict_keys([exp(x + 2*exp(-x)), exp(x), exp(-x)])
\end{verbatim}

Next, any item $\omega$ is taken from mrv that converges to zero for
$x\to\infty$. The item $\omega=e^{-x}$ is obtained. If such a term is not
present in the mrv set (i.e., all terms converge to infinity instead of zero),
the relation $f(x)\sim {1\over f(x)}$ can be used.

Next step is to rewrite the mrv in terms of $\omega$: $\{{1\over\omega},
\omega, {1\over\omega}e^{2\omega}\}$. Then the original subexpressions are
substituted back into $f(x)$ and expanded with respect to $\omega$:
\begin{equation}
    \label{gruntz_example_fn2}
f(x) = {1\over x}-{1\over\omega}+{1\over\omega}e^{2\omega}
     = 2+{1\over x} + 2\omega + O(\omega^2)
\end{equation}

Since $\omega$ is from the mrv set, then in the limit $x\to\infty$ it is
$\omega\to0$ and so $2\omega + O(\omega^2) \to 0$ in~\eqref{gruntz_example_fn2}:
\begin{equation}
f(x) = {1\over x}-{1\over\omega}+{1\over\omega}e^{2\omega}
    = 2+{1\over x} + 2\omega + O(\omega^2)
    \to 2 + {1\over x}
\end{equation}

Since the result ($2+{1\over x}$) still depends on $x$, the above procedure is
iterated on the result until just a number (independent of $x$) is obtained,
which is the final limit. In the above case the limit is $2$, as can be
verified by SymPy:

\begin{verbatim}
>>> limit(exp(x+2*exp(-x))-exp(x) + 1/x, x, oo)
2
\end{verbatim}

In general, when $f(x)$ is expanded in terms of $\omega$, it is obtained:
\begin{equation}
f(x) = \underbrace{O\left({1\over \omega^3}\right)}_\infty
    + \underbrace{C_{-2}(x)\over \omega^2}_\infty
    + \underbrace{C_{-1}(x)\over \omega}_\infty
    + {C_{0}(x)}
    + \underbrace{C_{1}(x)\omega}_0
    + \underbrace{O(\omega^2)}_0
\end{equation}
The positive powers of $\omega$ are zero. If there are any negative powers of
$\omega$, then the result of the limit is infinity, otherwise the limit is
equal to $\lim\limits_{x\to\infty} C_0(x)$. The expression $C_0(x)$ is simpler
than $f(x)$ and so the algorithm always converges. A proof of this, as well as
further details are given in Gruntz's PhD thesis~\cite{Gruntz1996limits}.


\subsection{Printers}

SymPy has a rich collection of expression printers for displaying expressions
to the user. By default, an interactive Python session will render the
\verb|str| form of an expression, which has been used in all the examples in
this paper so far.

\begin{verbatim}
>>> phi0 = Symbol('phi0')
>>> str(Integral(sqrt(phi0), phi0))
Integral(sqrt(phi0 + 1), x)
\end{verbatim}

Expressions can be printed with 2D monospace text with \verb|pprint|. This
uses Unicode characters to render mathematical symbols such as integral signs,
square roots, and parentheses. Greek letters and subscripts in symbol names
are rendered automatically.

Alternately, the \verb|use_unicode=False| flag can be set, which causes the
expression to be printed using only ASCII characters.

\begin{verbatim}
>>> pprint(Integral(sqrt(phi0 + 1), phi0), use_unicode=False)
  /
 |
 |   __________
 | \/ phi0 + 1  d(phi0)
 |
/
\end{verbatim}

The function \verb|latex| returns a \LaTeX{} representation of an expression.

\begin{verbatim}
>>> print(latex(Integral(sqrt(phi0 + 1), phi0)))
\int \sqrt{\phi_{0} + 1}\, d\phi_{0}
\end{verbatim}

Users are encouraged to run the \verb|init_printing| function at the beginning
of interactive sessions, which automatically enables the best pretty printing
supported by their environment. In the Jupyter notebook or
qtconsole~\cite{perez2007ipython} the \LaTeX{} printer is used to render
expressions using MathJax or \LaTeX{} if it is installed on the system. The 2D
text representation is used otherwise.

Other printers such as MathML are also available. SymPy uses an extensible
printer subsystem which allows users to customize the printing for any given
printer, and for custom objects to define their printing behavior for any
printer. SymPy's code generation capabilities, which we will not discuss
in-depth here, use the same printer model.


% Sets
\subsection{Sets}
%% Sets

SymPy supports representation of a wide variety of set, this is achieved by
first defining abstract representation for a smaller number of atomic set
classes and then combining and transforming them using various set operations.

Each of the set class inherits from the base Set class and defines
rules to check membership of a SymPy object, to calculate union, intersection,
set different. In cases we are not able to evaluate these
operations to atomic set classes they are represented as abstract unevaluated
objects.


\begin{itemize}

    % Description of Empty Set sounds too obvious, not sure if we need to keep
    % it.
    \item \textbf{Empty Set}: Nothing is a member of Empty Set. Union with
another set returns the other set and intersection leads to an Empty Set.

    \item \textbf{Universal Set}: Everything is a member of Universal Set.
Union of Universal Set with any set gives Universal Set and intersection leads
to the other set itself.

    \item \textbf{Finite Set} is functionally equivalent to python's set
object. Its members can be any object including strings and other sets
themselves.

    \item \textfb{Range} implements a range of integers and is defined by
specifying a start value, an end value and a step size. Range is functionally
equivalent to python's range except the fact that it accepts infinity at end
points allowing us to represent infinite ranges.


    \item \textbf{Real Interval} is specified by giving the start and end point
and specifying if it is open or closed in these respective ends. The set of
real numbers is represented as a special case of a real interval where the
start point is negative infinite and the end point is positive infinite.

\end{itemize}


%% Operations

Other than unevaluated classes of of Union, Intersection and Set Difference
operations, we have following set classes.

\begin{itemize}

    \item Product Set abstractly defines the Cartesian product of two or more
sets. Product Set is useful when representing higher dimensional spaces. For
example to represent a three dimensional space we simple take the Cartesian
product of three Real sets.

    \item Image Set represents the set of image a function when applied to a
particular set. In notation Image Set of a function F w.r.t a set S is \{ F(x)
| x \in S \} In particular we use Image Set to represent the set of infinite
solutions from trigonometric equations.


    \item Condition Set represents subset of a set who's members which
satisfies a particular condition. In notation Condition Set of set S w.r.t to a
condition H is \{x | H(x), x \in S \}. We use Condition Set to represent the
set of solutions of an equation or an inequality where the equation or the
inequality is the condition and the set is the domain in which we aim to find
the solution.


\end{itemize}





%% Explaining it later because it is a special case of Image Set rather being
%% something atomic

    \textbf{Complex Region}

%% Representations achievable through application of Operations on atomic set
%% types mentioned above.


%% Special Cases


% Solvers (regular equations, maybe also mention other types of solvers like ODEs/recurrence/Diophantine)
\subsection{Solvers}
SymPy includes a module of equation solvers for symbolic equations. There are two
functions for solving algebraic equations in SymPy: \texttt{solve}
and \texttt{solveset}.
\texttt{solveset} has several design changes with respect to the older
\texttt{solve} function. This distinction is present in order to resolve the
usability issues with the
previous \texttt{solve} function API while maintaining backward compatibility
with earlier versions of SymPy.
\texttt{solveset} only requires essential input information from the user.
The function signatures of \texttt{solve} and \texttt{solveset} are
\begin{verbatim}
solve(f, *symbols, **flags)
solveset(f, symbol, domain=S.Complexes)
\end{verbatim}
The \texttt{domain} parameter is typically either \texttt{S.Complexes} (the
default) or \texttt{S.Reals}; the latter causes \texttt{solveset} to only return real solutions.
Both functions implicitly assume that expressions are equal to 0. For
instance, \texttt{solveset(x - 1, x)} solves $x - 1 = 0$ for $x$.

It should be noted that \texttt{solve} has an inconsistent output API for different types
of inputs. For instance, depending on the input, it may return a Python
list or a Python dictionary. On the other hand,
\texttt{solveset} has a canonical output API: it always returns
a SymPy set object.

\texttt{solveset} is under active development as a planned replacement for
\texttt{solve}. There are certain features which are implemented in
\texttt{solve} that are not yet implemented in \texttt{solveset}. Notably,
these include nonlinear multivariate system and transcendental equations.
More examples of \texttt{solveset} and \texttt{solve} can be found in the
supplement.


Diophantine equations play a central and an important role in number theory.
A Diophantine equation has the form, $f(x_1, x_2, \ldots x_n) = 0$
where $n \geq 2$ and $x_1, x_2, \ldots x_n$ are integer variables. If we can find
$n$ integers $a_1, a_2, \ldots a_n$ such that $x_1 = a_1, x_2 = a_2, \ldots x_n = a_n$
satisfies the above equation, we say that the equation is solvable.

Currently, following five types of Diophantine equations can be solved using
SymPy's Diophantine module.

\begin{itemize}
    \item Linear Diophantine equations: $a_1x_1 + a_2x_2 + \cdots + a_{n}x_{n} = b$
    \item General binary quadratic equation: $ax^2 + bxy + cy^2 + dx + ey + f = 0$
    \item Homogeneous ternary quadratic equation: $ax^2 + by^2 + cz^2 + dxy + eyz + fzx = 0$
    \item Extended Pythagorean equation: $a_{1}x_{1}^2 + a_{2}x_{2}^2 + \cdots + a_{n}x_{n}^2 = a_{n+1}x_{n+1}^2$
    \item General sum of squares: $x_{1}^2 + x_{2}^2 + \cdots + x_{n}^2 = k$
\end{itemize}

When an equation is fed into Diophantine module, it factors the equation (if
possible) and solves each factor separately. Then all the results are combined to
create the final solution set. Following examples illustrate some of the basic
functionalities of the Diophantine module.

\begin{minted}{pycon}
>>> from sympy import symbols
>>> x, y, z = symbols("x, y, z", integer=True)

>>> diophantine(2*x + 3*y - 5)
set([(3*t_0 - 5, -2*t_0 + 5)])

>>> diophantine(2*x + 4*y - 3)
set()

>>> diophantine(x**2 - 4*x*y + 8*y**2 - 3*x + 7*y - 5)
set([(2, 1), (5, 1)])

>>> diophantine(x**2 - 4*x*y + 4*y**2 - 3*x + 7*y - 5)
set([(-2*t**2 - 7*t + 10, -t**2 - 3*t + 5)])

>>> diophantine(3*x**2 + 4*y**2 - 5*z**2 + 4*x*y - 7*y*z + 7*z*x)
set([(-16*p**2 + 28*p*q + 20*q**2, 3*p**2 + 38*p*q - 25*q**2, 4*p**2 - 24*p*q + 68*q**2)])

>>> from sympy.abc import a, b, c, d, e, f
>>> diophantine(9*a**2 + 16*b**2 + c**2 + 49*d**2 + 4*e**2 - 25*f**2)
set([(70*t1**2 + 70*t2**2 + 70*t3**2 + 70*t4**2 - 70*t5**2, 105*t1*t5, 420*t2*t5, 60*t3*t5, 210*t4*t5, 42*t1**2 + 42*t2**2 + 42*t3**2 + 42*t4**2 + 42*t5**2)])

>>> diophantine(a**2 + b**2 + c**2 + d**2 + e**2 + f**2 - 112)
set([(8, 4, 4, 4, 0, 0)])
\end{minted}


% Matrices (worth including to stress that they are symbolic)
\subsection{Matrices}

Besides being an important feature in its own right, computations on
matrices with symbolic entries are important for many algorithms
within SymPy.  The following code shows some basic usage of the
\texttt{Matrix} class.
\begin{verbatim}
>>> A = Matrix([[x, x + y], [y, x]])
>>> A
Matrix([
[x, x + y],
[y,     x]])
\end{verbatim}

SymPy matrices support common symbolic linear algebra manipulations, including
matrix addition, multiplication, exponentiation, computing determinants,
solving linear systems, and computing inverses using LU decomposition, LDL
decomposition, Gauss-Jordan elimination, Cholesky decomposition, Moore-Penrose
pseudoinverse, singular values, and adjugate matrix.

All operations are performed symbolically. For instance, eigenvalues are computed
by generating the characteristic polynomial using the Berkowitz algorithm and
then solving it using polynomial routines.

\begin{verbatim}
>>> A.eigenvals()
{x - sqrt(y*(x + y)): 1, x + sqrt(y*(x + y)): 1}
\end{verbatim}

Internally these matrices store the elements as Lists of Lists (LIL), meaning
the matrix is stored as a list of lists of entries (effectively, the
input format used to create the matrix \texttt{A} above), making it a
dense representation.\footnote{Similar to the polynomials module, dense here
  means that all entries are stored in memory, contrasted with a sparse
  representation where only nonzero entries are stored.} For storing sparse
matrices, the \verb|SparseMatrix| class can be used. Sparse matrices store
their elements in Dictionary of Keys (DOK) format, meaning entries are stored
as \texttt{(row, column)} pairs mapping to the elements.

SymPy also supports matrices with symbolic dimension values. \verb|MatrixSymbol|
represents a matrix with dimensions $m\times n$, where $m$ and $n$ can be
symbolic. Matrix addition and multiplication, scalar operations, matrix inverse,
and transpose are stored symbolically as matrix expressions.

Block matrices are also implemented in SymPy. \verb|BlockMatrix| elements can
be any matrix expression, including explicit matrices, matrix symbols, and
other block matrices. All functionalities of matrix expressions are also
present in \verb|BlockMatrix|.

When symbolic matrices are combined with the assumptions module for logical
inference, they provide powerful reasoning over invertibility,
semi-definiteness, orthogonality, etc., which are valuable in the construction
of numerical linear algebra systems.

More examples for \verb|Matrix| and \verb|BlockMatrix| may be found in the
supplement.

