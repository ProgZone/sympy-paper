% Features to discuss in-depth:

\begin{table}
\label{features-table}
\caption{SymPy Features and Descriptions}
\begin{tabular}[htbc]{|l|l|}
\hline
\textbf{Feature} & \textbf{Description} \\
\hline
Discrete Math & Summations, products, binomial coefficients, prime number
    tools, integer factorization, Diophantine equation solving, and
    boolean logic representation, equivalence testing, and inference.\\
Concrete Math & Tools for determining whether summation and product
    expressions are convergent, absolutely convergent, hypergeometric, and
    other properties. May also compute Gosper's normal form
    \cite{petkovvsek1996bak} for two univariate polynomials.\\
Plotting & Hooks for visualizing expressions via matplotlib \cite{Hunter:2007}
    or as text drawings when lackinga graphical backend.\\
Geometry & Allows the creation of 2D geometrical entities,
    such as lines and circles. Enables queries on these entities, including
    asking the area of an ellipse, checking for collinearity of a set of
    points, or finding the intersection between two lines.\\
Statistics & Support for a random variable type as well as the ability to
    declare this variable from prebuilt distribution functions such as
    Normal, Exponential, Coin, Die, and other custom distributions.\\
Polynomials & Computes polynomial algebras over various coefficient domains
    ranging from the simple (e.g. polynomial division) to the advanced
    (e.g. Gr\"obner bases \cite{adams1994introduction} and multivariate
    factorization over algebraic number domains).\\
Sets & Representations of empty, finite, and infinite sets. This includes
    special sets such as for all natural, integer, and complex numbers.\\
Series & Impelements series expansion, sequences, and limit of sequences.
    This includes special series, such as Fouier and power series.\\
Vectors & Provides basic vector math and differential calculus with respect
    to 3D Cartesian coordinate systems.\\
Matrices & \\
combinatorics & \\
Group Theory & \\
Code Generation & \\
Tensors & \\
Lie Algebras & \\
Cryptography & \\
category theory & \\
special functions & \\
\hline
\end{tabular}
\end{table}


% Basic operations (the core)
\subsection{Basic Operations}
% Including the assumptions system

% Calculus (differentiation, integration, limits). Note that algorithm
% descriptions will go in the algorithms section.
\subsection{Calculus}

% Solvers (regular equations, maybe also mention other types of solvers like ODEs/recurrence/Diophantine)
\subsection{Solvers}

% Matrices (worth including to stress that they are symbolic)
\subsection{Matrices}

% Physics module (some sampling, to show that it is there)
\subsection{Physics}

