% Features to discuss in-depth:

SymPy has an extensive feature set that encompasses too much to cover
in-depth here. Bedrock areas, such as calculus, receive their own sub-sections
below. Additionally, Table~\ref{features-table} describes other capabilities
present in the SymPy code base. This gives a sampling from the breadth of
topics and application domains that SymPy services.


\begin{longtable}[htbc]{lp{0.56\linewidth}}
\caption{SymPy Features and Descriptions\label{features-table}}\\
\toprule
\textbf{Feature} & \textbf{Description} \\
\midrule
Discrete Math & Summations, products, binomial coefficients,
    prime number tools, integer factorization, Diophantine equation solving, and
    boolean logic representation, equivalence testing, and inference.\\
Concrete Math & Tools for determining whether summation and product
    expressions are convergent, absolutely convergent, hypergeometric, and
    other properties. May also compute Gosper's normal form~\cite{petkovvsek1996bak} for two univariate polynomials.\\
Plotting & Hooks for visualizing expressions via matplotlib~\cite{Hunter:2007}
    or as text drawings when lacking a graphical back-end.\\
Geometry & Allows the creation of 2D geometrical entities,
    such as lines and circles. Enables queries on these entities, including
    asking the area of an ellipse, checking for collinearity of a set of
    points, or finding the intersection between two lines.\\
Statistics & Support for a random variable type as well as the ability to
    declare this variable from prebuilt distribution functions such as
    Normal, Exponential, Coin, Die, and other custom distributions.\\
Polynomials & Computes polynomial algebras over various coefficient domains
    ranging from the simple (e.g., polynomial division) to the advanced
    (e.g., Gr\"obner bases~\cite{adams1994introduction} and multivariate
    factorization over algebraic number domains).\\
Sets & Representations of empty, finite, and infinite sets. This includes
    special sets such as for all natural, integer, and complex numbers.\\
Series & Implements series expansion, sequences, and limit of sequences.
    This includes special series, such as Fourier and power series.\\
Vectors & Provides basic vector math and differential calculus with respect
    to 3D Cartesian coordinate systems.\\
Matrices & Tools for creating matrices of symbols and expressions.
    This is capable of both sparse and dense representations and performing
    symbolic linear algebraic operations (e.g., inversion and factorization).\\
Combinatorics \& Group Theory & Implements permutations, combinations,
    partitions, subsets,
    various permutation groups (such as polyhedral, Rubik, symmetric,
    and others), Gray codes~\cite{Nijenhuis1978combinatorial},
    and Prufer sequences~\cite{biggs1976graph}.\\
Code Generation & Enables generation of compilable and executable
    code in a variety of different programming languages directly from
    expressions. Target languages include C, Fortran, Julia, JavaScript,
    Mathematica, Matlab and Octave, Python, and Theano.\\
Tensors & Symbolic manipulation of indexed objects.\\
Lie Algebras & Represents Lie algebras and root systems.\\
Cryptography & Represents block and stream ciphers, including
    shift, Affine, substitution, Vigenere's, Hill's, bifid, RSA, Kid RSA,
    linear-feedback shift registers, and Elgamal encryption\\
Special Functions & Implements a number of well known special functions,
    including Dirac delta, Gamma, Beta, Gauss error functions, Fresnel
    integrals, Exponential integrals, Logarithmic integrals, Trigonometric
    integrals, Bessel, Hankel, Airy, B-spline, Riemann Zeta, Dirichlet eta,
    polylogarithm, Lerch transcendent, hypergeometric, elliptic integrals,
    Mathieu, Jacobi polynomials, Gegenbauer polynomial, Chebyshev polynomial,
    Legendre polynomial, Hermite polynomial, Laguerre polynomial, and
    spherical harmonic functions.\\
\bottomrule

\end{longtable}

\subsection{Simplification}

% polynomial expressions

% functions

% expand( ), factor( ), collect( ), together( ), apart( )
%% maybe a table best suits this part.

% simplification: simplify, sqrt denest, fu, trigsimp

The generic way to simplify an expression is by calling the \texttt{simplify}
function.
It must be emphasized that simplification is not a rigously defined
mathematical operation~\cite{Carette2004understanding}.
The \texttt{simplify} function applies several simplification routines along
with heuristics to make the output expression as ``simple'' as possible.

It is often preferable to apply more directed simplification functions. These
apply very specific rules to the input expression and are typically able to make
guarantees about the output. For instance, the \texttt{factor} function,
given a polynomial with rational coefficients in several variables,
is guaranteed to
produce a factorization into irreducible factors. Table~\ref{simplify-table}
lists common simplification functions.

\begin{longtable}[htbc]{lp{0.83\linewidth}}
\caption{Some SymPy Simplification Functions\label{simplify-table}}\\
\toprule
\verb|expand| & expand the expression \\
\verb|factor| & factor a polynomial into irreducibles \\
\verb|collect| & collect polynomial coefficients \\
\verb|cancel| & rewrite a rational function as $p/q$ with common factors
canceled \\
\verb|apart| & compute the partial fraction decomposition of a rational function
\\
\verb|trigsimp| & simplify trigonometric expressions~\cite{fu2006automated} \\
\bottomrule
\end{longtable}


\subsection{Calculus}

Integrals are calculated with the \verb|integrate| function. SymPy
implements a combination of the Risch
algorithm~\cite{bronstein2005integration}, table lookups, a reimplementation
of Manuel Bronstein's ``Poor Man's Integrator''~\cite{Bronstein2005pmint}, and
an algorithm for computing integrals based on Meijer G-functions~\cite{Roach1996hyper,roach1997meijerg}. These allow
SymPy to compute a wide variety of indefinite and definite integrals. The
Meijer G-function algorithm and the Risch algorithm are respectively
demonstrated below by the computation of \[\int_{0}^{\infty} e^{-s t}\log{\left (t \right )}\, dt = - \frac{ \log{\left (s \right )} + \gamma}{s}\] and \[\int \frac{- 2 x^{2} \left(\log{\left (x \right )} + 1\right) e^{x^{2}} + {\left(e^{x^{2}} + 1\right)}^{2}}{x {\left(e^{x^{2}} + 1\right)}^{2} \left(\log{\left (x \right )} + 1\right)}\, dx = \log{\left (\log{\left (x \right )} + 1 \right )} + \frac{1}{e^{x^{2}} + 1}.\]
\begin{verbatim}
>>> s, t = symbols('s t', positive=True)
>>> integrate(exp(-s*t)*log(t), (t, 0, oo)).simplify()
-(log(s) + EulerGamma)/s
>>> integrate((-2*x**2*(log(x) + 1)*exp(x**2) +
... (exp(x**2) + 1)**2)/(x*(exp(x**2) + 1)**2*(log(x) + 1)), x)
log(log(x) + 1) + 1/(exp(x**2) + 1)
\end{verbatim}

Derivatives are computed with the \verb|diff| function, which recursively uses
the various differentiation rules.
\begin{verbatim}
>>> diff(sin(x)*exp(x), x)
exp(x)*sin(x) + exp(x)*cos(x)
\end{verbatim}

Summations are computed with \verb|summation|  using a combination of Gosper's
algorithm~\cite{gosper1978decision}, an algorithm that uses Meijer
G-functions~\cite{Roach1996hyper,roach1997meijerg}, and heuristics. Products
are computed with \verb|product| function via a suite of heuristics.
% TODO: Are there other summation algorithms implemented?
% TODO: A good summation example or two
\begin{verbatim}
>>> i, n = symbols('i n')
>>> summation(2**i, (i, 0, n - 1))
2**n - 1
>>> summation(i*factorial(i), (i, 1, n))
n*factorial(n) + factorial(n) - 1
\end{verbatim}

Limits are computed with the \verb|limit| function. The limit module
implements the Gruntz algorithm~\cite{Gruntz1996limits} for computing symbolic
limits.
For example, the following computes
$\lim\limits_{x\to \infty} x\sin(\frac{1}{x})=1$. Note that SymPy denotes
$\infty$ as \verb|oo|.
\begin{verbatim}
>>> limit(x*sin(1/x), x, oo)
1
\end{verbatim}
As a more complex example, SymPy computes \[\lim\limits_{x\to 0}{\left(2 e^{\frac{1 - \cos{\left (x \right )}}{\sin{\left (x \right )}}} -
  1\right)}^{\frac{\sinh{\left (x \right )}}{\operatorname{atan}^{2}{\left (x
      \right )}}} = e.\]
\begin{verbatim}
>>> limit((2*E**((1-cos(x))/sin(x))-1)**(sinh(x)/atan(x)**2), x, 0)
E
\end{verbatim}

Integrals, derivatives, summations, products, and limits that cannot be
computed return unevaluated objects. These can also be created directly if the
user chooses.
\begin{verbatim}
>>> integrate(x**x, x)
Integral(x**x, x)
>>> Sum(2**i, (i, 0, n - 1))
Sum(2**i, (i, 0, n - 1))
\end{verbatim}


\subsection{Printers}


SymPy has a rich collection of expression printers for displaying expressions
to the user. By default, an interactive Python session will render the
\verb|str| form of an expression, which has been used in all the examples in
this paper so far. The \verb|str| form of an expression is valid Python and
roughly matches what a user would type to enter the expression.

\begin{verbatim}
>>> phi0 = Symbol('phi0')
>>> str(Integral(sqrt(phi0), phi0))
'Integral(sqrt(phi0), phi0)'
\end{verbatim}

Expressions can be printed with 2D monospace text with \verb|pprint|. This
uses Unicode characters to render mathematical symbols such as integral signs,
square roots, and parentheses. Greek letters and subscripts in symbol names
are rendered automatically.

\noindent
\includegraphics[width=1\textwidth]{pprint.pdf}
Alternately, the \verb|use_unicode=False| flag can be set, which causes the
expression to be printed using only ASCII characters.

\begin{verbatim}
>>> pprint(Integral(sqrt(phi0 + 1), phi0), use_unicode=False)
  /
 |
 |   __________
 | \/ phi0 + 1  d(phi0)
 |
/
\end{verbatim}

The function \verb|latex| returns a \LaTeX{} representation of an expression.

\begin{verbatim}
>>> print(latex(Integral(sqrt(phi0 + 1), phi0)))
\int \sqrt{\phi_{0} + 1}\, d\phi_{0}
\end{verbatim}

Users are encouraged to run the \verb|init_printing| function at the beginning
of interactive sessions, which automatically enables the best pretty printing
supported by their environment. In the Jupyter Notebook or Qt
Console~\cite{perez2007ipython} the \LaTeX{} printer is used to render
expressions using MathJax or \LaTeX{}, if it is installed on the system. The
2D text representation is used otherwise.

Other printers such as MathML are also available. SymPy uses an extensible
printer subsystem which allows users to customize the printing for any given
printer, and for custom objects to define their printing behavior for any
printer. SymPy's code generation capabilities, which we will not discuss
in-depth here, use this subsystem to convert expressions into code in various
languages.


% Solvers (regular equations, maybe also mention other types of solvers like ODEs/recurrence/Diophantine)
\subsection{Solvers}
SymPy includes a module of equation solvers for symbolic equations. There are two
functions for solving algebraic equations in SymPy: \texttt{solve}
and \texttt{solveset}.
\texttt{solveset} has several design changes with respect to the older
\texttt{solve} function. This distinction is present in order to resolve the
usability issues with the
previous \texttt{solve} function API while maintaining backward compatibility
with earlier versions of SymPy.
\texttt{solveset} only requires essential input information from the user.
The function signatures of \texttt{solve} and \texttt{solveset} are
\begin{verbatim}
solve(f, *symbols, **flags)
solveset(f, symbol, domain=S.Complexes)
\end{verbatim}
The \texttt{domain} parameter is typically either \texttt{S.Complexes} (the
default) or \texttt{S.Reals}; the latter causes \texttt{solveset} to only return real solutions.
Both functions implicitly assume that expressions are equal to 0. For
instance, \texttt{solveset(x - 1, x)} solves $x - 1 = 0$ for $x$.

It should be noted that \texttt{solve} has an inconsistent output API for different types
of inputs. For instance, depending on the input, it may return a Python
list or a Python dictionary. On the other hand,
\texttt{solveset} has a canonical output API: it always returns
a SymPy set object.

\texttt{solveset} is under active development as a planned replacement for
\texttt{solve}. There are certain features which are implemented in
\texttt{solve} that are not yet implemented in \texttt{solveset}. Notably,
these include nonlinear multivariate system and transcendental equations.
More examples of \texttt{solveset} and \texttt{solve} can be found in the
supplement.


% Matrices (worth including to stress that they are symbolic)
\subsection{Matrices}

Besides being an important feature in its own right, computations on
matrices with symbolic entries are important for many algorithms
within SymPy.  The following code shows some basic usage of the
\texttt{Matrix} class.
\begin{verbatim}
>>> A = Matrix([[x, x + y], [y, x]])
>>> A
Matrix([
[x, x + y],
[y,     x]])
\end{verbatim}

SymPy matrices support common symbolic linear algebra manipulations, including
matrix addition, multiplication, exponentiation, computing determinants,
solving linear systems, and computing inverses using LU decomposition, LDL
decomposition, Gauss-Jordan elimination, Cholesky decomposition, Moore-Penrose
pseudoinverse, singular values, and adjugate matrix.

All operations are performed symbolically. For instance, eigenvalues are computed
by generating the characteristic polynomial using the Berkowitz algorithm and
then solving it using polynomial routines.

\begin{verbatim}
>>> A.eigenvals()
{x - sqrt(y*(x + y)): 1, x + sqrt(y*(x + y)): 1}
\end{verbatim}

Internally these matrices store the elements as Lists of Lists (LIL), meaning
the matrix is stored as a list of lists of entries (effectively, the
input format used to create the matrix \texttt{A} above), making it a
dense representation.\footnote{Similar to the polynomials module, dense here
  means that all entries are stored in memory, contrasted with a sparse
  representation where only nonzero entries are stored.} For storing sparse
matrices, the \verb|SparseMatrix| class can be used. Sparse matrices store
their elements in Dictionary of Keys (DOK) format, meaning entries are stored
as \texttt{(row, column)} pairs mapping to the elements.

SymPy also supports matrices with symbolic dimension values. \verb|MatrixSymbol|
represents a matrix with dimensions $m\times n$, where $m$ and $n$ can be
symbolic. Matrix addition and multiplication, scalar operations, matrix inverse,
and transpose are stored symbolically as matrix expressions.

Block matrices are also implemented in SymPy. \verb|BlockMatrix| elements can
be any matrix expression, including explicit matrices, matrix symbols, and
other block matrices. All functionalities of matrix expressions are also
present in \verb|BlockMatrix|.

When symbolic matrices are combined with the assumptions module for logical
inference, they provide powerful reasoning over invertibility,
semi-definiteness, orthogonality, etc., which are valuable in the construction
of numerical linear algebra systems.

More examples for \verb|Matrix| and \verb|BlockMatrix| may be found in the
supplement.

