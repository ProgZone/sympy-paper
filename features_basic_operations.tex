
\subsubsection{Expression manipulation}

% symbols, various ways to declare them

% polynomial expressions

% functions

% expand( ), factor( ), collect( ), together( ), apart( )
%% maybe a table best suits this part.

% simplification: simplify, sqrt denest, fu, trigsimp

% substitution, wilds, replacement

% unevaluated expressions (evaluate=False), global_evaluate

\subsubsection{Assumptions system}

% Including the assumptions system


SymPy has two assumptions systems, referred to as new-style and old-style assumptions.

In the old-style assumptions system propositions are assigned to symbols
upon class construction, for example, to declare the symbol $i$ as positive integer,
one would call

\begin{verbatim}
i = Symbol("i", integer=True, positive=True)
\end{verbatim}

querying the assumptions is handled through attributes

\begin{verbatim}
i.is_positive
i.is_integer
\end{verbatim}

These methods return either a boolean, indicating whether the preposition is true or false,
or a None, when it is impossible to determine the truth value of the queried preposition.

Despite the fact that assumptions can only be declared on symbols, querying can
happen on every expression.

\begin{verbatim}
In [1]: x,y = symbols('x y', positive=True)

In [2]: (x*y).is_positive
Out[2]: True

In [3]: z = symbols('z')

In [4]: (x*z).is_positive

In [5]: w = symbols('w', positive=False)

In [6]: (x*w).is_positive
Out[6]: False
\end{verbatim}

The output 2 is true because SymPy's algorithms can deduce that the product of
two positive numbers is positive, while there is no output for input 4, as the
symbol $z$ doesn't have any information about its sign, and the product
$x\cdot z$ may be positive as well as negative.
Finally, output 6 is false as the product of positive and negative numbers is
negative.

The new-style assumptions are an assumptions system that exists alongside with
the old-style, but is significantly different in the way predicates are
used.
Predicates in the new-style assumptions system are located under the \textit{Q}
namespace, they appear as
\verb|Q.positive|, \verb|Q.integer| and so on.

Querying is provided through the \verb|ask| functions.
The previous example in the new-style assumptions can be written as
\begin{verbatim}
In [1]: ask(Q.positive(x*y), Q.positive(x) & Q.positive(y))
Out[1]: True

In [2]: ask(Q.positive(x*y), Q.positive(x))

In [3]: ask(Q.positive(x*y), Q.positive(x) & Q.negative(y))
Out[3]: False
\end{verbatim}
%
That is, \verb|ask| returns the truth value of its first parameter assuming
that its latter argument is true.

Expressions like \verb|Q.positive| are instances of the class \verb|Predicate|,
while the same expression with a parameter, such as \verb|Q.positive(x)| is an
instance of \verb|AppliedPredicate|.

Logical connectors can be expressed through operator overloading,
such as in \verb|Q.positive(x) & Q.positive(y)|,
or by directly constructing the identical expression through the 
logical connector class, in this case \verb|And(Q.positive(x), Q.positive(y))|.


\subsubsection{Calculus}

% Calculus (differentiation, integration, limits). Note that algorithm
% descriptions will go in the algorithms section.

Derivations can be computed with the \verb|diff| function,
or using the method with the same name on the expressions:

\begin{verbatim}
In [1]: diff(sin(x), x)
Out[1]: cos(x)

In [2]: sin(x).diff(x)
Out[2]: cos(x)
\end{verbatim}

The class \verb|Derivative| is a container for unevaluated derivatives

\begin{verbatim}
In [3]: expr = Derivative(sin(x), x)

In [4]: expr
Out[4]: 
d         
--(sin(x))
dx        
\end{verbatim}

To evaluate such a held expression, simply call the \verb|doit| method:

\begin{verbatim}
In [5]: expr.doit()
Out[5]: cos(x)
\end{verbatim}

Integrals can be analogously calculated either with the \verb|integrate| function
or with the method with the same name on expressions:
\begin{verbatim}
>>> integrate(sin(x), x)
-cos(x)
\end{verbatim}
This expression returns an expression whose derivative is the original expression.
Notice that integrals are defined up to an integration constant,
for the sake of simplicity SymPy will not display the full generic expression.

Definite integration can be calculated with the same method, by specifying a
range of the integration variable:
\begin{verbatim}
>>> integrate(sin(x), (x, 0, 1))
-cos(1) + 1
\end{verbatim}

To express unevaluated integrals, the class \verb|Integral| may help
\begin{verbatim}
Integral(sin(x), x)
\end{verbatim}
as in the case of derivatives, the method \verb|doit| will cause such an expression
to be evaluated.

Limits:
\begin{verbatim}
In [9]: limit(sin(x)/x, x, 0)
Out[9]: 1
\end{verbatim}
for unevaluated expressions, \verb|Limit|.

TODO: Sums and products.

\subsubsection{Expression outputs}

% LaTeX printer

% pretty printer
