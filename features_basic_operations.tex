
\subsubsection{Expression manipulation}

% symbols, various ways to declare them

All symbols in SymPy must be instantiated and assigned to a variable before
they can be used. This is typically done through the \texttt{symbols}
function, which creates multiple symbols at once. For instance,

\begin{verbatim}
>>> x, y, z = symbols('x y z')
\end{verbatim}

creates three symbols named \texttt{x}, \texttt{y}, and \texttt{z}, assigned
to Python variables of the same name. The Python variable names that symbols
are assigned to are immaterial---we could have just as well have written
\verb|a, b, c = symbol('x y z')|.

Expressions are created from symbols using Python syntax, which mirrors usual
mathematical notation. Note that in Python, exponentiation is \verb|**|.

\begin{verbatim}
>>> (x**2 - 2*x + 3)/y
(x**2 - 2*x + 3)/y
\end{verbatim}

% polynomial expressions

% functions

% expand( ), factor( ), collect( ), together( ), apart( )
%% maybe a table best suits this part.

% simplification: simplify, sqrt denest, fu, trigsimp

The generic way to simplify an expression is by calling the \texttt{simplify}
function.
It must be emphasized that simplification is not an unambigously defined
mathematical operation~\cite{Carette2004understanding}.
The \texttt{simplify} function applies several simplification routines along
with some heuristics to make the output expression as ``simple'' as possible.

It is often preferable to apply more directed simplification functions. These
apply very specific rules to the input expression, and are often able to make
guarantees about the output (for instance, the \texttt{factor} function, given
a polynomial with rational coefficients in several variables, is guaranteed to
produce a factorization into irreducible factors).
Table~\label{simplify-table} lists some common simplification functions.

% TODO: add a simple example for each
% TODO: fix the formatting
\ref{simplify-table}
\begin{tabular}{l|p{0.7\linewidth}}
expand & expand the expression \\
factor & factor a polynomial into irreducibles \\
collect & collect polynomial coefficients \\
cancel & rewrite a rational function as $p/q$ with common factors canceled \\
apart & compute the partial fraction decomposition of a rational function \\
trigsimp & simplify trigonometric expressions~\cite{fu2006automated} \\
sqrtdenest & denest radicals \\
\end{tabular}

% substitution, wilds, replacement

Substitutions are performed through the \texttt{.subs} method, which accepts
wildcards and is sensible to some mathematical properties while matching,
such as associativity, commutativity,  additive and multiplicative inverses,
and matching of powers.

\texttt{.replace} provides more basic matching algorithm,
though it allows for costum matching functions to be passed to it.

\texttt{.xreplace} is an expression tree structural replacement routing,
unaware of any mathematical property.

%% TODO: add examples

% unevaluated expressions (evaluate=False), global_evaluate

Expression constructors accept in most cases the boolean parameter \texttt{evaluate},
setting it to false will prevent automatic evaluation of the expression.
The \verb|global_evaluate| variable may be employed to globally block any kind
of evaluation.


\subsubsection{Assumptions system}

% Including the assumptions system


SymPy has two assumptions systems, referred to as new-style and old-style assumptions.

In the old-style assumptions system propositions are assigned to symbols
upon class construction, for example, to declare the symbol $i$ as positive integer,
one would call

\begin{verbatim}
i = Symbol("i", integer=True, positive=True)
\end{verbatim}

querying the assumptions is handled through attributes

\begin{verbatim}
i.is_positive
i.is_integer
\end{verbatim}

These methods return either a boolean, indicating whether the preposition is true or false,
or a None, when it is impossible to determine the truth value of the queried preposition.

Despite the fact that assumptions can only be declared on symbols, querying can
happen on every expression.

\begin{verbatim}
In [1]: x,y = symbols('x y', positive=True)

In [2]: (x*y).is_positive
Out[2]: True

In [3]: z = symbols('z')

In [4]: (x*z).is_positive

In [5]: w = symbols('w', positive=False)

In [6]: (x*w).is_positive
Out[6]: False
\end{verbatim}

The output 2 is true because SymPy's algorithms can deduce that the product of
two positive numbers is positive, while there is no output for input 4, as the
symbol $z$ doesn't have any information about its sign, and the product
$x\cdot z$ may be positive as well as negative.
Finally, output 6 is false as the product of positive and negative numbers is
negative.

The new-style assumptions are an assumptions system that exists alongside with
the old-style, but is significantly different in the way predicates are
used.
Predicates in the new-style assumptions system are located under the \textit{Q}
namespace, they appear as
\verb|Q.positive|, \verb|Q.integer| and so on.

Querying is provided through the \verb|ask| functions.
The previous example in the new-style assumptions can be written as
\begin{verbatim}
In [1]: ask(Q.positive(x*y), Q.positive(x) & Q.positive(y))
Out[1]: True

In [2]: ask(Q.positive(x*y), Q.positive(x))

In [3]: ask(Q.positive(x*y), Q.positive(x) & Q.negative(y))
Out[3]: False
\end{verbatim}
%
That is, \verb|ask| returns the truth value of its first parameter assuming
that its latter argument is true.

Expressions like \verb|Q.positive| are instances of the class \verb|Predicate|,
while the same expression with a parameter, such as \verb|Q.positive(x)| is an
instance of \verb|AppliedPredicate|.

Logical connectors can be expressed through operator overloading,
such as in \verb|Q.positive(x) & Q.positive(y)|,
or by directly constructing the identical expression through the
logical connector class, in this case \verb|And(Q.positive(x), Q.positive(y))|.


\subsubsection{Calculus}

% Calculus (differentiation, integration, limits). Note that algorithm
% descriptions will go in the algorithms section.

Derivations can be computed with the \verb|diff| function,
or using the method with the same name on the expressions:

\begin{verbatim}
In [1]: diff(sin(x), x)
Out[1]: cos(x)

In [2]: sin(x).diff(x)
Out[2]: cos(x)
\end{verbatim}

The class \verb|Derivative| is a container for unevaluated derivatives

\begin{verbatim}
In [3]: expr = Derivative(sin(x), x)

In [4]: expr
Out[4]:
d
--(sin(x))
dx
\end{verbatim}

To evaluate such a held expression, simply call the \verb|doit| method:

\begin{verbatim}
In [5]: expr.doit()
Out[5]: cos(x)
\end{verbatim}

Integrals can be analogously calculated either with the \verb|integrate| function
or with the method with the same name on expressions:
\begin{verbatim}
>>> integrate(sin(x), x)
-cos(x)
\end{verbatim}
This expression returns an expression whose derivative is the original expression.
Notice that integrals are defined up to an integration constant,
for the sake of simplicity SymPy will not display the full generic expression.

Definite integration can be calculated with the same method, by specifying a
range of the integration variable:
\begin{verbatim}
>>> integrate(sin(x), (x, 0, 1))
-cos(1) + 1
\end{verbatim}

To express unevaluated integrals, the class \verb|Integral| may help
\begin{verbatim}
Integral(sin(x), x)
\end{verbatim}
as in the case of derivatives, the method \verb|doit| will cause such an expression
to be evaluated.

Limits:
\begin{verbatim}
In [9]: limit(sin(x)/x, x, 0)
Out[9]: 1
\end{verbatim}
for unevaluated expressions, \verb|Limit|.

TODO: right and left limits.

Sums and products are handled by the \texttt{Sum} and \texttt{Product} classes,
respectively.
Analogously with \texttt{Integral}, the first argument is the expression to be
summed over, whereas the following arguments represent the summation and
multiplication indices, respectively, provided with integer ranges.

It may be noted the existence of the \texttt{IndexedBase} class,
which provides the construction of indexed symbols, that is symbols that are
treated as different if their indices are different.


\subsubsection{Expression outputs}

Alongside with its parsers, SymPy has a rich collection of expression printers.

% LaTeX printer
Expressions may be readily transformed into a LaTeX form with the \texttt{latex( )}
function.

% pretty printer
Pretty printer outputs the expression in traditional form with characters,
outputs can be visualized in monospace fonts.
