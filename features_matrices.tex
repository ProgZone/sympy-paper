

SymPy supports matrices with symbolic expressions as elements. There are two
types of matrices, Mutable and Immutable. Mutable classes are the default in
SymPy as mutability is important for performance, but it means that standard
matrices can not interact well with the rest of SymPy. This is because the
Basic object, from which most SymPy classes inherit, is immutable.

Immutable matrix classes inherit from Basic and can thus interact more
naturally with the rest of SymPy.

\begin{verbatim}
In [1]: from sympy import Matrix, symbols, MatrixSymbol

In [2]: x, y = symbols('x y', positive=True)

In [3]: t = Matrix(2, 2, [x, x + y, y, x])

In [4]: t

Out[4]:
Matrix([
[    x, x + y],
[    y,     x]])

In [5]: t[0, 1] = y

In [6]: t
Out[6]:
Matrix([
[x, y],
[y, x]])
\end{verbatim}

All SymPy matrix types can do linear algebra including matrix addition,
multiplication, exponentiation, computing determinant, solving linear
systems and computing inverses using LU decomposition, LDL decomposition,
Gauss-Jordan elimination, Cholesky decomposition, Moore-Penrose pseudoinverse,
adjugate matrix.

Eigenvalues are computed symbolically as well. Eigenvalues are computed by
generating the characteristic polynomial using the Berkowitz algorithm and
then solving it using polynomial routines. Diagonalizable matrices can be
diagonalized first to compute the eigenvalues.

\begin{verbatim}
In [10]: t.eigenvals()
Out[10]: {x - y: 1, x + y: 1}
\end{verbatim}

Internally these matrices store the elements as a list making it a dense
representation. For storing sparse matrices, SparseMatrix and 
ImmutableSparseMatrix classes can be used. Sparse matrix classes store
the elements in Dictionary of Keys (DoK) format.

SymPy also supports matrices with unknown dimension values. MatrixSymbol
represents a matrix with dimensions \verb|m, n| where \verb|m| and \verb|n|
can be symbols or integers. Matrix addition and multiplication, scalar
operations, matrix inverse and transpose are stored symbolically as
matrix expressions. Mutable matrices are converted to corresponding immutable
types before interacting with matrix expressions

\begin{verbatim}
In [11]: m, n, p = symbols("m, n, p", integer=True)

In [12]: r, s = MatrixSymbol("r", m, n), MatrixSymbol("s", n, p)

In [13]: u = r * s + 2*MatrixSymbol("t", m, p)

In [14]: u.shape
Out[14]: (m, p)

In [15]: u[0, 1]
Out[15]: 2*t[0, 1] + Sum(r[0, _k]*s[_k, 1], (_k, 0, n - 1))
\end{verbatim}

Block matrices are also supported in SymPy. BlockMatrix elements can be
any matrix expression which includes immutable matrices, matrix symbols and
block matrices. All functionalities of matrix expressions are also present in
BlockMatrix.


\begin{verbatim}
>>> from sympy import (MatrixSymbol, BlockMatrix, symbols,
...     Identity, ZeroMatrix, block_collapse)
>>> n, m, l = symbols('n m l')
>>> X = MatrixSymbol('X', n, n)
>>> Y = MatrixSymbol('Y', m ,m)
>>> Z = MatrixSymbol('Z', n, m)
>>> B = BlockMatrix([[X, Z], [ZeroMatrix(m,n), Y]])
>>> print(B)
Matrix([
[X, Z],
[0, Y]])
>>> print(B[0, 0])
X[0, 0]
\end{verbatim}