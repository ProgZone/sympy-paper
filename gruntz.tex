SymPy calculates limits using the Gruntz algorithm, as described in%
~\cite{Gruntz1996limits}. The basic idea is as follows: any limit can be
converted to a limit $\lim\limits_{x\to\infty} f(x)$ by substitutions like
$x\to{1\over x}$. Then the most varying subexpression $\omega$ (that converges
to zero as $x\to\infty$ the fastest from all subexpressions) is identified in
$f(x)$, and $f(x)$ is expanded into a series with respect to $\omega$. Any
positive powers of $\omega$ converge to zero. If there are negative powers of
$\omega$, then the limit is infinite. The constant term (independent of
$\omega$, but could depend on $x$) then determines the limit (one might need to
recursively apply the Gruntz algorithm on this term to determine the limit).

To determine the most varying subexpression, the comparability classes must
first be defined, by calculating $L$:
\begin{equation}
L\equiv \lim_{x\to\infty} {\log |f(x)| \over \log |g(x)|}
\end{equation}
And then operations $<$, $>$ and $\sim$ are defined as follows: $f>g$ when
$L=\pm\infty$ (it is said that $f$ is more rapidly varying than $g$, i.e., $f$
goes to $\infty$ or $0$ faster than $g$, $f$ is greater than any power of $g$),
$f<g$ when $L=0$ ($f$ is less rapidly varying than $g$) and $f\sim g$ when
$L\neq 0,\pm\infty$ (both $f$ and $g$ are bounded from above and below by
suitable integral powers of the other). Here are some examples of comparability
classes:
$$2 < x < e^x < e^{x^2} < e^{e^x}$$
$$2\sim 3\sim -5$$
$$x\sim x^2\sim x^3\sim {1\over x}\sim x^m\sim -x$$
$$e^x\sim e^{-x}\sim e^{2x}\sim e^{x+e^{-x}}$$
$$f(x)\sim{1\over f(x)}$$

The Gruntz algorithm is now illustrated on the following example:
\begin{equation}
    \label{gruntz_example_fn}
f(x) = e^{x+2e^{-x}} - e^x + {1\over x} \,.
\end{equation}
The goal is to calculate $\lim\limits_{x\to\infty} f(x)$.
First the set of most rapidly varying subexpressions is determined, the so
called \textit{mrv set}. For~\eqref{gruntz_example_fn}, the following mrv set
$\{e^x, e^{-x}, e^{x+2e^{-x}}\}$ is obtained. These are all subexpressions of%
~\eqref{gruntz_example_fn} and they all belong to the same comparability class.
This calculation can be done using SymPy as follows:

% no-doctest
\begin{verbatim}
>>> from sympy.series.gruntz import mrv
>>> mrv(exp(x+2*exp(-x))-exp(x) + 1/x, x)[0].keys()
dict_keys([exp(x + 2*exp(-x)), exp(x), exp(-x)])
\end{verbatim}

Next any item $\omega$ is taken from mrv that converges to zero for
$x\to\infty$. The item $\omega=e^{-x}$ is obtained. If such a term is not
present in the mrv set (i.e., all terms converge to infinity instead of zero),
the relation $f(x)\sim {1\over f(x)}$ can be used.

Next step is to rewrite the mrv in terms of $\omega$: $\{{1\over\omega},
\omega, {1\over\omega}e^{2\omega}\}$. Then the original subexpressions are
substituted back into $f(x)$ and expanded with respect to $\omega$:
\begin{equation}
    \label{gruntz_example_fn2}
f(x) = {1\over x}-{1\over\omega}+{1\over\omega}e^{2\omega}
     = 2+{1\over x} + 2\omega + O(\omega^2)
\end{equation}

Since $\omega$ is from the mrv set, then in the limit $x\to\infty$ it is
$\omega\to0$ and so $2\omega + O(\omega^2) \to 0$ in~\eqref{gruntz_example_fn2}:
\begin{equation}
f(x) = {1\over x}-{1\over\omega}+{1\over\omega}e^{2\omega}
    = 2+{1\over x} + 2\omega + O(\omega^2)
    \to 2 + {1\over x}
\end{equation}

Since the result ($2+{1\over x}$) still depends on $x$, the above procedure is
iterated on the result until just a number (independent of $x$) is obtained,
which is the final limit. In the above case the limit is $2$, as can be
verified by SymPy:

\begin{verbatim}
>>> limit(exp(x+2*exp(-x))-exp(x) + 1/x, x, oo)
2
\end{verbatim}

In general, when $f(x)$ is expanded in terms of $\omega$, it is obtained:
\begin{equation}
f(x) = \underbrace{O\left({1\over \omega^3}\right)}_\infty
    + \underbrace{C_{-2}(x)\over \omega^2}_\infty
    + \underbrace{C_{-1}(x)\over \omega}_\infty
    + {C_{0}(x)}
    + \underbrace{C_{1}(x)\omega}_0
    + \underbrace{O(\omega^2)}_0
\end{equation}
The positive powers of $\omega$ are zero. If there are any negative powers of
$\omega$, then the result of the limit is infinity, otherwise the limit is
equal to $\lim\limits_{x\to\infty} C_0(x)$. The expression $C_0(x)$ is simpler
than $f(x)$ and so the algorithm always converges. A proof of this, as well as
further details are given in Gruntz's Ph.D. thesis~\cite{Gruntz1996limits}.
