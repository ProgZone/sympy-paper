The limit module implements the Gruntz algorithm~\cite{Gruntz1996limits}.

Examples:
\begin{verbatim}
In [1]: limit(sin(x)/x, x, 0)
Out[1]: 1

In [2]: limit((2*E**((1-cos(x))/sin(x))-1)**(sinh(x)/atan(x)**2), x, 0)
Out[2]: E
\end{verbatim}

We first define comparability classes by calculating $L$:
\begin{equation}
L\equiv \lim_{x\to\infty} {\log |f(x)| \over \log |g(x)|}
\end{equation}
And then we define the $<$, $>$ and $\sim$ operations as follows: $f>g$ when
$L=\pm\infty$ ($f$ is more rapidly varying than $g$, i.e., $f$ goes to $\infty$
or $0$ faster than $g$, $f$ is greater than any power of $g$), $f<g$ when $L=0$
($f$ is less rapidly varying than $g$) and $f\sim g$ when $L\neq 0,\pm\infty$
(both $f$ and $g$ are bounded from above and below by suitable integral powers
of the other).

Examples:
$$2 < x < e^x < e^{x^2} < e^{e^x}$$
$$2\sim 3\sim -5$$
$$x\sim x^2\sim x^3\sim {1\over x}\sim x^m\sim -x$$
$$e^x\sim e^{-x}\sim e^{2x}\sim e^{x+e^{-x}}$$
$$f(x)\sim{1\over f(x)}$$

The Gruntz algorithm, on an example:
$$f(x) = e^{x+2e^{-x}} - e^x + {1\over x}$$
$$\lim_{x\to\infty} f(x) = ?$$

Strategy:
mrv set: the set of most rapidly varying subexpressions
$\{e^x, e^{-x}, e^{x+2e^{-x}}\}$, the same comparability class
Take an item $\omega$ from mrv, converging to 0 at infinity. Here
$\omega=e^{-x}$. If not present in the mrv set, use the relation
$f(x)\sim {1\over f(x)}$.

Rewrite the mrv set using $\omega$: $\{{1\over\omega}, \omega,
{1\over\omega}e^{2\omega}\}$, substitute back into $f(x)$ and expand in
$\omega$:
$$f(x) = {1\over x}-{1\over\omega}+{1\over\omega}e^{2\omega}
    = 2+{1\over x} + 2\omega + O(\omega^2)$$

The core idea of the algorithm: $\omega$ is from the mrv set, so in the limit
$\omega\to0$:
$$f(x) = {1\over x}-{1\over\omega}+{1\over\omega}e^{2\omega}
    = 2+{1\over x} + 2\omega + O(\omega^2)
    \to 2 + {1\over x}$$

We iterate until we get just a number, the final limit. Gruntz proved this
algorithm always works and converges in his Ph.D. thesis
\cite{Gruntz1996limits}.

Generally:
$$ f(x) = \underbrace{O\left({1\over \omega^3}\right)}_\infty
    + \underbrace{C_{-2}(x)\over \omega^2}_\infty
    + \underbrace{C_{-1}(x)\over \omega}_\infty
    + {C_{0}(x)}
    + \underbrace{C_{1}(x)\omega}_0
    + \underbrace{O(\omega^2)}_0
$$
we look at the lowest power of $\omega$. The limit is one of: $0$,
$\lim_{x\to\infty} C_0(x)$, $\infty$.
