SymPy is an full featured computer algebra system (CAS) written in the Python
language. It is open source, licensed under the extremely permissive BSD
license, which allows anyone to reuse SymPy or its source code, even for
commercial purposes. SymPy was started by Ond\v{r}ej \v{C}ert\'{\i}k in 2005, and
it has since grown into a large open source project, with over 500 contributors.

SymPy is written entirely in the Python programming language, which is also
the language used to interact with it, both programmatically and
interactively. Python is a popular dynamically typed programming language that
has a focus on ease of use and readability. Unlike many CASs, SymPy does not
invent its own programming language. We rather take the approach that Python
is already a well-designed programming language, both for building SymPy and
for interactive use. Python is also a very popular language for scientific
computing and data science, with a wide range of useful libraries.
% Cite numpy, scipy, pandas
% We could also cite
% https://stackoverflow.com/research/developer-survey-2016#most-popular-technologies-per-occupation

SymPy has a strong focus that it be usable as a library. This means that
extensibility is important in its API design. This is also one of the reasons
SymPy makes no attempt to extend the Python language itself. The goal is for
users of SymPy to be able to import SymPy alongside other Python libraries in
their workflow, whether that is an interactive workflow or programmatic use as
part of a larger library. This does mean, for instance, that unlike in most
CASs, all variables in SymPy must be defined before they are used (Python does
not provide a mechanism to automatically declare variables). This is a minor
inconvenience for interactive use, but we consider it to be good practice any
reproducible work, and it aligns with the philosophy of Python, ``Explicit is
better than implicit.''
% cite https://www.python.org/dev/peps/pep-0020/

SymPy does not have a builtin graphical user iterface (GUI), however, when
used in the Jupyter Notebook
% citation
SymPy expressions will pretty print using MathJax.
% citation
