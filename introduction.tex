SymPy is a full featured computer algebra system (CAS) written in the Python
programming language~\cite{lutz2013learning}.
It is free and open source software, being licensed under the 3-clause BSD
license.
% cite BSD?
The SymPy project was started by Ond\v{r}ej \v{C}ert\'{\i}k in 2005, and it has
since grown to over 500 contributors. Currently, SymPy is
developed on GitHub using a bazaar community
model~\cite{raymond1999cathedral}. The accessibility of the codebase and the
open community model allow SymPy to rapidly respond to the needs of the
community of users and developers.
% citation?

Python is a dynamically typed programming language that has a focus on
ease of use and readability. Due in part to this focus, it has become a popular
language for scientific
computing and data science, with a broad ecosystem of
libraries~\cite{oliphant2007python}. SymPy is itself used by many libraries
and tools to support research within a variety  domains, such as
Sage~\cite{SAGE} (pure mathematics),
yt~\cite{2011ApJS..192....9T} (astronomy and astrophysics),
PyDy~\cite{gede2013constrained} (multibody dynamics), and
SfePy~\cite{cimrman2014sfepy} (finite elements).

Unlike many CASs, SymPy does not invent its own programming language. Python
itself is used both for the internal implementation and the end user
interaction.  The exclusive usage of a single programming language makes it easier
for people already familiar with that language to use or develop SymPy.
Simultaneously, it enables developers to focus on mathematics, rather than
language design.

SymPy is designed with a strong focus on usability as a library.
Extensibility is important in its application program interface
(API) design, and thus SymPy makes no attempt to extend
the Python language itself. The goal is for users of SymPy to be able to
include SymPy alongside other Python libraries in their workflow, whether that
is in an interactive environment or programmatic use as part of a larger system.

As a library, SymPy does not have a built-in graphical user
interface (GUI). However, SymPy exposes a rich interactive display system,
including registering printers with Jupyter~\cite{perez2007ipython} frontends,
including the Notebook and Qt Console, which will render SymPy
expressions using MathJax~\cite{cervone2012mathjax} or \LaTeX{}.

The remainder of this paper discusses key components of the SymPy software.
Section~\ref{sec:architecture} discusses the architecture of SymPy.
Section~\ref{sec:features} enumerates the features of SymPy and takes a closer
look at some of the important ones. Following that, section~\ref{sec:numerics}
looks at the numerical features of SymPy and its dependency library, mpmath.
Section~\ref{sec:domain_specific} looks at the domain specific physics
submodules for performing symbolic and numerical calculations in classical mechanics
and quantum mechanics. Finally, section~\ref{sec:conclusion} concludes the paper
and discusses future work.
