SymPy is a full featured computer algebra system (CAS) written in the Python
programming language. It is open source, licensed under the extremely
permissive 3-clause BSD license.
% cite BSD?
SymPy was started by Ond\v{r}ej \v{C}ert\'{\i}k in 2005, and it has since
grown into a large open source project, with over 500 contributors. SymPy is
developed on GitHub using a bazaar community
model~\cite{raymond1999cathedral}. The accessibility of the codebase and the
open community model allow SymPy to rapidly respond to the needs of the
community of users, and has made the large contributor count possible.
% citation?

SymPy is written entirely in the Python programming language.
% cite Python?
Python is a popular dynamically typed programming language that has a focus
on ease of use and readability. It also a very popular language for scientific
computing and data science, with a wide range of useful
libraries~\cite{oliphant2007python}.
% Cite numpy, scipy, pandas
% We could also cite
% https://stackoverflow.com/research/developer-survey-2016#most-popular-technologies-per-occupation
Unlike many CASs, SymPy does not invent its own programming language. Python
is used both for the internal implementation and the user interaction.
Exclusively using Python in this way makes it easier for people already
familiar with the language to use or develop SymPy. It also lets the SymPy
developers focus on mathematics, rather than language design.

SymPy is designed with a strong focus that it be usable as a library. This
means that extensibility is important in its application program interface
(API) design. This is also one of the reasons SymPy makes no attempt to extend
the Python language itself. The goal is for users of SymPy to be able to
import SymPy alongside other Python libraries in their workflow, whether that
is an interactive workflow or programmatic use as part of a larger system.

SymPy does not have a built in graphical user interface (GUI), however, when
used in the Jupyter Notebook
% citation
SymPy expressions will pretty print using MathJax.
% citation
