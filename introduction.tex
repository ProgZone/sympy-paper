SymPy is a full featured computer algebra system (CAS) written in the
Python~\cite{lutz2013learning}
programming language.
It is free and open source software, licensed under the 3-clause BSD
license~\cite{rosen2005open}.
The SymPy project was started by Ond\v{r}ej \v{C}ert\'{\i}k in 2005, and it has
since grown to over 500 contributors. Currently, SymPy is
developed on GitHub using a bazaar community
model~\cite{raymond1999cathedral}. The accessibility of the codebase and the
open community model allow SymPy to rapidly respond to the needs of
users and developers.

Python is a dynamically typed programming language that has a focus on
ease of use and readability.\footnote{\label{note:python}This paper assumes a moderate
  familiarity with the Python programming language.} Due in part to this focus, it has become a popular
language for scientific
computing and data science, with a broad ecosystem of
libraries~\cite{oliphant2007python}. SymPy is itself used as a dependency
by many libraries
and tools to support research within a variety of domains, such as
SageMath~\cite{sagemath} (pure and applied mathematics),
yt~\cite{2011ApJS..192....9T} (astronomy and astrophysics),
PyDy~\cite{gede2013constrained} (multibody dynamics), and
SfePy~\cite{cimrman2014sfepy} (finite elements).

Unlike many CAS's, SymPy does not invent its own programming language. Python
itself is used both for the internal implementation and end user
interaction. By using the operator overloading functionality of Python, SymPy follows the embedded domain specific language paradigm proposed by Hudak~\cite{dsl-little-languages}.  The exclusive usage of a single programming language makes it easier
for people already familiar with that language to use or develop SymPy.
Simultaneously, it enables developers to focus on mathematics, rather than
language design.  SymPy officially supports Python 2.6, 2.7 and 3.2--3.5.

SymPy is designed with a strong focus on usability as a library.
Extensibility is important in its application program interface
(API) design. Thus, SymPy makes no attempt to extend
the Python language itself. The goal is for users of SymPy to be able to
include SymPy alongside other Python libraries in their workflow, whether that
be in an interactive environment or as a programmatic part in a larger system.

Being a library, SymPy does not have a built-in graphical user interface (GUI).
However, SymPy exposes a rich interactive display system, and supports
registering display formatters with Jupyter~\cite{kluyver2016jupyter} frontends,
including the Notebook and Qt Console, which will render SymPy expressions
using MathJax~\cite{cervone2012mathjax} or \LaTeX{}.

The remainder of this paper discusses key components of the SymPy library.
Section~\ref{sec:features} enumerates the features of SymPy and takes a closer
look at some of the important ones. The section~\ref{sec:numerics} looks at
the numerical features of SymPy and its dependency library, mpmath.
Section~\ref{sec:domain_specific} looks at the domain specific physics
submodules for performing symbolic and numerical calculations in classical
mechanics and quantum mechanics. Section~\ref{sec:architecture} discusses the
architecture of SymPy. Section~\ref{sec:other-proj} looks at a selection of
packages that depend on SymPy. Conclusions and future directions for SymPy are given
in section~\ref{sec:conclusion}. All examples in this paper use SymPy version
1.0 and mpmath version 0.19.

Additionally, the supplementary material takes a deeper look at a few SymPy
topics. Supplement section~\ref{S-suppsec:Gruntz} discusses the Gruntz
algorithm, which SymPy uses to calculate symbolic limits.
Sections~\ref{S-suppsec:Series}--\ref{S-suppsec:numsimpl} of the supplement
discuss the series, logic, Diophantine equations, sets, statistics, category
theory, tensor, and numerical simplification submodules of SymPy,
respectively. Supplement section~\ref{S-suppsec:examples} provides additional
examples for topics discussed in the main paper. Supplement
section~\ref{S-suppsec:sympy-gamma} discusses the SymPy Gamma project.
Finally, section~\ref{S-suppsec:comp-mma} of the supplement contains a brief
comparison of SymPy with Wolfram Mathematica.

The following statement imports all SymPy functions into the global Python
namespace.\footnote{\texttt{import *} has been used here to aid the
  readability of the paper, but is best to avoid such wildcard import
  statements in production code, as they make it unclear which names are
  present in the namespace. Furthermore, imported names could clash with
  already existing imports from another package. For example, SymPy, the
  standard Python \texttt{math} library, and NumPy all define the \texttt{exp}
  function, but only the SymPy one will work with SymPy symbolic expressions.}
From here on, all examples in this paper assume that this statement has been
executed:\footnote{\label{note:prompt} The three greater-than signs denote the user input for the
  Python interactive session, with the result, if there is one, shown on the
  next line.}

\begin{verbatim}
>>> from sympy import *
\end{verbatim}

All the examples in this paper can be tested on
\href{http://live.sympy.org}{SymPy Live}, an online Python shell that uses the
Google App Engine~\cite{ciurana2009google} to execute SymPy code. SymPy Live
is also integrated into the SymPy documentation at
\href{http://docs.sympy.org}{http://docs.sympy.org}.
