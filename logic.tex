
SymPy supports construction and manipulation of boolean expressions
through the \texttt{sympy.logic} submodule. SymPy symbols can be used as
propositional variables and subsequently be replaced with \texttt{True}
or \texttt{False} values. Many functions for manipulating boolean
expressions have been implemented in the \texttt{sympy.logic} submodule.

\subsection{Constructing boolean expressions}

A boolean variable can be declared as a SymPy \verb|Symbol|. Python operators
\texttt{\&}, \texttt{\textbar{}} and \texttt{\textasciitilde{}} are overridden
when using SymPy objects to use the SymPy functionality for logical
\texttt{And}, \texttt{Or}, and \texttt{Not}. Other logic functions are also
integrated into SymPy, including \texttt{Xor} and \texttt{Implies}, which are
constructed with \texttt{\^{}} and \texttt{\textgreater{}\textgreater{}},
respectively. Expressions can therefore be constructed either by using
the shortcut operator notation or by directly creating the relevant
objects: \verb|And()|,
\verb|Or()|, \verb|Not()|, \verb|Xor()|, \verb|Implies()|, \verb|Nand()|,
\verb|Nor()|, etc.:

\begin{verbatim}
>>> e = (x & y) | z
>>> e.subs({x: True, y: True, z: False})
True
\end{verbatim}

\subsection{CNF and DNF}

Any boolean expression can be converted to conjunctive normal form, disjunctive
normal form, or negation normal form. The API also exposes methods to check if
a boolean expression is in any of the aforementioned forms.

\begin{verbatim}
>>> from sympy.logic.boolalg import is_dnf, is_cnf
>>> to_cnf((x & y) | z)
And(Or(x, z), Or(y, z))
>>> to_dnf(x & (y | z))
Or(And(x, y), And(x, z))
>>> is_cnf((x | y) & z)
True
>>> is_dnf((x & y) | z)
True
\end{verbatim}

\subsection{Simplification and Equivalence}

The \texttt{sympy.logic} submodule supports simplification of given boolean expression by making
deductions from the expression. Equivalence of two logical expressions can also
be checked. In the case of equivalence, the function \texttt{bool\_map}
can be used to show which variables of the first expression correspond
to which variables of the second one.

\begin{verbatim}
>>> a, b, c = symbols('a b c')
>>> e = a & (~a | ~b) & (a | c)
>>> simplify(e)
And(Not(b), a)
>>> e1 = a & (b | c)
>>> e2 = (x & y) | (x & z)
>>> bool_map(e1, e2)
(And(Or(b, c), a), {a: x, b: y, c: z})
\end{verbatim}

\subsection{SAT solving}

The submodule also supports satisfiability (SAT) checking of a given boolean
expression. If an expression is satisfiable, it is possible to return
a variable assignment which satisfies it. The API also supports
listing all possible assignments.
The SAT solver has a clause learning DPLL algorithm implemented with a watch
literal scheme and VSIDS heuristic~\cite{moskewicz2008method}.

\begin{verbatim}
>>> satisfiable(a & (~a | b) & (~b | c) & ~c)
False
>>> satisfiable(a & (~a | b) & (~b | c) & c)
{a: True, b: True, c: True}
\end{verbatim}
