SymPy supports matrices with symbolic expressions as elements.

\begin{verbatim}
>>> x, y = symbols('x y')
>>> A = Matrix(2, 2, [x, x + y, y, x])
>>> A
Matrix([
[    x, x + y],
[    y,     x]])
\end{verbatim}

All SymPy matrix types can do linear algebra including matrix addition,
multiplication, exponentiation, computing determinant, solving linear systems,
and computing inverses using LU decomposition, LDL decomposition, Gauss-Jordan
elimination, Cholesky decomposition, Moore-Penrose pseudoinverse, and adjugate
matrix.

All operations are computed symbolically. Eigenvalues are computed by generating
the characteristic polynomial using the Berkowitz algorithm and then solving it
using polynomial routines. Diagonalizable matrices can be diagonalized first to
compute the eigenvalues.
\begin{verbatim}
>>> A.eigenvals()
{x - sqrt(y*(x + y)): 1, x + sqrt(y*(x + y)): 1}
\end{verbatim}

Internally these matrices store the elements as a list, making it a dense
representation. For storing sparse matrices, the \verb|SparseMatrix| class can
be used. Sparse matrices store the elements in a dictionary of keys (DoK)
format.

SymPy also supports matrices with symbolic dimension values. \verb|MatrixSymbol|
represents a matrix with dimensions $m\times n$, where $m$ and $n$ can be
symbolic. Matrix addition and multiplication, scalar operations, matrix inverse,
and transpose are stored symbolically as matrix expressions.
\begin{verbatim}
>>> m, n, p = symbols("m, n, p", integer=True)
>>> R = MatrixSymbol("R", m, n)
>>> S = MatrixSymbol("S", n, p)
>>> T = MatrixSymbol("t", m, p)
>>> U = R*S + 2*T
>>> u.shape
(m, p)
>>> U[0, 1]
2*T[0, 1] + Sum(R[0, _k]*S[_k, 1], (_k, 0, n - 1))
\end{verbatim}

Block matrices are also supported in SymPy. \verb|BlockMatrix| elements can be any
matrix expression which includes explicit matrices, matrix symbols, and block
matrices. All functionalities of matrix expressions are also present in
\verb|BlockMatrix|.


\begin{verbatim}
>>> n, m, l = symbols('n m l')
>>> X = MatrixSymbol('X', n, n)
>>> Y = MatrixSymbol('Y', m ,m)
>>> Z = MatrixSymbol('Z', n, m)
>>> B = BlockMatrix([[X, Z], [ZeroMatrix(m, n), Y]])
>>> B
Matrix([
[X, Z],
[0, Y]])
>>> B[0, 0]
X[0, 0]
>>> B.shape
(m + n, m + n)
\end{verbatim}
