Floating point numbers in SymPy are implemented by the \texttt{Float} class,
which represents an arbitrary-precision binary floating-point number by
storing its value and precision (in bits). This representation is distinct
from the Python built-in \texttt{float} type, which is a wrapper around
machine \texttt{double} types and uses a fixed precision (53-bit).

Because Python \texttt{float} literals are limited in precision, strings
should be used to input precise decimal values:
\begin{verbatim}
>>> Float(1.1)
1.10000000000000
>>> Float(1.1, 30)   # precision equivalent to 30 digits
1.10000000000000008881784197001
>>> Float("1.1", 30)
1.10000000000000000000000000000
\end{verbatim}
The \texttt{evalf} method converts a constant symbolic expression to a
\texttt{Float} with the specified precision, here 25 digits:
\begin{verbatim}
>>> (pi + 1).evalf(25)
4.141592653589793238462643
\end{verbatim}
\texttt{Float} numbers do not track their accuracy,
and should be used with caution within symbolic expressions
since familiar dangers of floating-point arithmetic apply~\cite{goldberg1991every}.
A notorious case is that of catastrophic cancellation:
\begin{verbatim}
>>> cos(exp(-100)).evalf(25) - 1
0
\end{verbatim}
Applying the \texttt{evalf} method to the whole expression solves
this problem. Internally, \texttt{evalf} estimates the number of accurate
bits of the floating-point
approximation for each sub-expression, and adaptively increases the
working precision until the estimated accuracy of the
final result matches the sought number of decimal digits:
\begin{verbatim}
>>> (cos(exp(-100)) - 1).evalf(25)
-6.919482633683687653243407e-88
\end{verbatim}
The \texttt{evalf} method works with complex numbers and supports
more complicated expressions, such as
special functions, infinite series, and integrals.
The internal error tracking does not provide rigorous error bounds
(in the sense of interval arithmetic) and cannot be used to accurately track
uncertainty in measurement data;
the sole purpose is to mitigate loss of accuracy that typically occurs
when converting symbolic expressions to numerical values.

\subsection{The mpmath library}
\label{sec:mpmath}

The implementation of arbitrary-precision floating-point arithmetic is
supplied by the mpmath library~\cite{mpmath}. Originally, it was developed as a SymPy
module but has subsequently been moved to a standalone pure-Python package.
The basic datatypes in mpmath are \texttt{mpf} and \texttt{mpc}, which
respectively act as multiprecision substitutes for Python's \texttt{float} and
\texttt{complex}. The floating-point precision is controlled by a global
context:

% doctest printer doesn't display "mpf"
% no-doctest
\begin{verbatim}
>>> import mpmath
>>> mpmath.mp.dps = 30    # 30 digits of precision
>>> mpmath.mpf("0.1") + mpmath.exp(-50)
mpf('0.100000000000000000000192874984794')
>>> print(_)   # pretty-printed
0.100000000000000000000192874985
\end{verbatim}

Like SymPy, mpmath is a pure Python library.
Internally, mpmath represents a floating-point number
${(-1)}^s x \cdot 2^y$ by a tuple $(s, x, y, b)$ where
$x$ and $y$ are arbitrary-size Python integers
and the redundant integer $b$ stores the bit length of $x$ for quick access.
If GMPY~\cite{GMPY} is installed, mpmath automatically uses
the \texttt{gmpy.mpz} type for~$x$, and GMPY methods
for rounding-related operations, improving performance.

Most mpmath and SymPy functions use the same naming scheme, although this is
not true in every case. For example, the symbolic SymPy summation expression
\texttt{Sum(f(x), (x, a, b))} representing $\sum_{x=a}^b f(x)$ is represented
in mpmath as \texttt{nsum(f, (a, b))}, where \texttt{f} is a numeric Python
function.

The mpmath library supports
special functions, root-finding, linear algebra, polynomial approximation,
and numerical computation of limits, derivatives, integrals, infinite
series, and solving ODEs. All features work in arbitrary precision
and use algorithms that allow computing hundreds of digits rapidly
(except in degenerate cases).

The double exponential (tanh-sinh) quadrature is used for numerical
integration by default. For smooth integrands, this algorithm usually
converges extremely rapidly, even when the integration interval is infinite
or singularities are present at the endpoints~\cite{takahasi1974double,bailey2005comparison}.
However, for good performance, singularities
in the middle of the interval must be specified
by the user.
To evaluate slowly converging limits and infinite series, mpmath
automatically tries Richardson extrapolation and the
Shanks transformation
(Euler-Maclaurin summation can also be used)~\cite{BenderOrszag1999}.
A function to evaluate oscillatory integrals by means of convergence
acceleration is also available.

A wide array of higher mathematical functions is implemented
with full support for complex values of all parameters and arguments,
including complete and incomplete gamma functions,
Bessel functions, orthogonal polynomials, elliptic functions and integrals,
zeta and polylogarithm functions,
the generalized hypergeometric function, and the Meijer G-function.
The Meijer G-function instance
$G_{1, 3}^{3, 0}\left(0 ; \tfrac{1}{2}, -1, - \tfrac{3}{2} | x \right)$
is a good test case~\cite{Toth2007}; past versions of both Maple and
Mathematica produced incorrect numerical values for large $x > 0$.
Here, mpmath automatically removes an internal singularity
and compensates for cancellations (amounting to 656 bits
of precision when $x = 10000$), giving correct values:
% doctest printer doesn't display "mpf"
% no-doctest
\begin{verbatim}
>>> mpmath.mp.dps = 15
>>> mpmath.meijerg([[],[0]], [[-0.5,-1,-1.5],[]], 10000)
mpf('2.4392576907199564e-94')
\end{verbatim}

Equivalently, with SymPy's interface this function can be evaluated as:
\begin{verbatim}
>>> meijerg([[],[0]], [[-S(1)/2,-1,-S(3)/2],[]], 10000).evalf()
2.43925769071996e-94
\end{verbatim}

Symbolic integration and summation often produce hypergeometric
and Meijer G-function closed forms (see Subsection~\ref{sec:calculus});
numerical evaluation of such special functions is a useful complement
to direct numerical integration and summation.
