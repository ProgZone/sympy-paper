There are several projects that use SymPy as a library for implementing
a part of their project, or even as a part of back-end for their 
application as well.
\newline
Some of them are listed below-:

\begin{itemize}
\item
  \href{http://cadabra.science/index.html}{\textbf{Cadabra}}: Cadabra is
  a symbolic computer algebra system (CAS) designed specifically for the
  solution of problems encountered in field theory.
\item
  \href{http://octave.sourceforge.net/symbolic/}{\textbf{Octave Symbolic}}:
  The Octave-Forge Symbolic package adds symbolic calculation features
  to GNU Octave. These include common Computer Algebra System tools such
  as algebraic operations, calculus, equation solving, Fourier and
  Laplace transforms, variable precision arithmetic and other features.
\item
  \href{https://github.com/jverzani/SymPy.jl}{\textbf{SymPy.jl}}:
  Provides a Julia interface to SymPy using PyCall.
\item
  \href{https://mathics.github.io/}{\textbf{Mathics}}: Mathics is a
  free, general-purpose online CAS featuring Mathematica compatible
  syntax and functions. It is backed by highly extensible Python code,
  relying on SymPy for most mathematical tasks.
\item
  \href{http://mathpix.com/}{\textbf{Mathpix}}: An iOS App, that uses
  Artificial Intelligence to detect handwritten math as input, and uses
  SymPy Gamma, to evaluate the math input and generate the relevant
  steps to solve the problem.
\item
  \href{http://www.sagemath.org/}{\textbf{Sage}}: A CAS, visioned to be
  a viable free open source alternative to Magma, Maple, Mathematica and
  Matlab.
\item
  \href{https://cloud.sagemath.com}{\textbf{SageMathCloud}}:
  SageMathCloud is a web-based cloud computing and course management
  platform for computational mathematics.
\item
  \href{http://www.pydy.org/}{\textbf{PyDy}}: Multibody Dynamics with
  Python.
\item
  \href{https://github.com/brombo/galgebra}{\textbf{galgebra}}:
  Geometric algebra (previously sympy.galgebra).
\item
  \href{http://yt-project.org/}{\textbf{yt}}: Python package for
  analyzing and visualizing volumetric data (yt.units uses SymPy).
\item
  \href{http://sfepy.org/}{\textbf{SfePy}}: Simple finite elements in
  Python.
\item
  \href{http://quameon.sourceforge.net/}{\textbf{Quameon}}: Quantum
  Monte Carlo in Python.
\item
  \href{http://lcapy.elec.canterbury.ac.nz/}{\textbf{Lcapy}}:
  Experimental Python package for teaching linear circuit analysis.
\item
  \href{http://digitalcommons.calpoly.edu/cgi/viewcontent.cgi?article=1072\&context=physsp/}{\textbf{Quantum
  Programming in Python}}: Quantum 1D Simple Harmonic Oscillator and
  Quantum Mapping Gate.
\item
  \href{http://mech.fsv.cvut.cz/~stransky/software/latexexpr/doc/}{\textbf{LaTeX
  Expression project}}: Easy LaTeX typesetting of algebraic expressions
  in symbolic form with automatic substitution and result computation).
\item
  \href{https://www.researchgate.net/publication/260585491_Symbolic_Statistics_with_SymPy/}{\textbf{Symbolic
  statistical modeling}}: Adding statistical operations to complex
  physical models.
\end{itemize}