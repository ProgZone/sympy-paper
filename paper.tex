\documentclass{article}


\begin{document}
\section{Introduction}

%% What sympy is, where to download etc.
%%
%% List other major CASs.
%%
%% Why SymPy.

\section{Architecture}

%% I volunteer to write this section. --Aaron
%%
%% Representing symbolic expressions using Python objects

Every symbolic expression in SymPy is an instance of a Python class.
Expressions are represented by expression trees. The operators is represented
by the type of an expression and the child nodes are stored in the
\texttt{args} attribute. A leaf node in the expression tree has an empty
\texttt{args}. For example, take the expression $xy + 1$.

\begin{verbatim}
>>> from sympy import *
>>> x, y = symbols('x y')
>>> expr = x*y + 1
\end{verbatim}

The expression \texttt{expr} is an addition, so it is of type \texttt{Add}. The child
nodes of \texttt{expr} are \texttt{x*y} and \texttt{1}.

\begin{verbatim}
>>> type(expr)
<class 'sympy.core.add.Add'>
>>> expr.args
(1, x*y)
\end{verbatim}

We can dig further into the expression tree to see the full expression. For
example, the first child node, given by \texttt{expr.args[0]} is
\texttt{1}. Its class is \texttt{Integer}, and it has empty \texttt{args},
indicating that it is a leaf node

\begin{verbatim}
>>> expr.args[0]
1
>>> type(expr.args[0])
<class 'sympy.core.numbers.One'>
>>> expr.args[0].args
()
\end{verbatim}

Every SymPy expression satisfies a key invariant, namely,
\texttt{expr.func(*expr.args) == expr}. This means that expressions are
rebuildable from their \texttt{args}.\footnote{\texttt{expr.func} is used
  instead of \texttt{type(expr)} to allow the function of an expression to be
  distinct from its actual Python class. In most cases the two are the same.}
Here, we note that in SymPy, the \texttt{==} operator represents exact
structural equality, not mathematical equality. This allows us to test if any
two expressions are equal to one another as expression trees.


%%
%% Assumptions
%%
%% Extensibility

\section{Algorithms}

%% Description of some algorithms (example: integration with Risch, Meijer G, Gruntz, polys)
%%
%% Description of numerics/mpmath (Fredrik)

\section{Features}

%% List of Features and how to use
%%
%% Quick overview of the main modules, what it can do and so on. It should probably provide examples how to use sympy.
%%
%% See also the supplement (below)

\section{Conclusion and future work}

\section{References}

\end{document}
