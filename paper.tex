% SIAM Article Template
\documentclass[review]{siamart0216}

\usepackage[utf8]{inputenc}

\usepackage{hyperref}
\usepackage{graphicx}
\usepackage{amsmath}
\usepackage{caption}
\graphicspath{ {images/} }

\usepackage{amsmath}
\usepackage{url}
\usepackage{hyperref}

% for nice source code syntax highlighting, also provides Listing env
\usepackage{minted}

% this is required for all the \url{} commands in the bib file
%\usepackage{hyperref}

% for nice units
\usepackage{siunitx}

% for images: png, pdf, etc
\usepackage{graphicx}

% for nice table formatting, i.e., /toprule, /midrule, etc
\usepackage{booktabs}

% to allow for \verb++ declarations in captions.
\usepackage{cprotect}

% to allow usage of \mathbb symbols
\usepackage{amssymb}

\usepackage{longtable}

\title{SymPy: Symbolic Computing in Python}

\input{authors}

\begin{document}
\maketitle

\section{Introduction}

%% What sympy is, where to download etc.
%%
%% List other major CASs.
%%
%% Why SymPy.

SymPy is a full featured computer algebra system (CAS) written in the Python
programming language. It is open source, being licensed under the extremely
permissive 3-clause BSD license.
% cite BSD?
SymPy was started by Ond\v{r}ej \v{C}ert\'{\i}k in 2005, and it has since
grown into a large open source project, with over 500 contributors. SymPy is
developed on GitHub using a bazaar community
model~\cite{raymond1999cathedral}. The accessibility of the codebase and the
open community model allows SymPy to rapidly respond to the needs of the
community of users, and has made the large contributor count possible.
% citation?

SymPy is written entirely in the Python programming language.
% cite Python?
Python is a popular dynamically typed programming language that has a focus on
ease of use and readability. It also a very popular language for scientific
computing and data science, with a wide range of useful
libraries~\cite{oliphant2007python}. SymPy is itself used by many libraries
and tools across many domains, such as Sage~\cite{SAGE} (pure mathematics),
yt~\cite{2011ApJS..192....9T} (astronomy and astrophysics),
PyDy~\cite{gede2013constrained} (multibody dynamics), and
SfePy~\cite{cimrman2014sfepy} (finite elements).

Unlike many CASs, SymPy does not invent its own programming language. Python
is used both for the internal implementation and the user interaction.
Exclusively using Python in this way makes it easier for people already
familiar with the language to use or develop SymPy. It also lets the SymPy
developers focus on mathematics, rather than language design.

SymPy is designed with a strong focus that it be usable as a library. This
means that extensibility is important in its application program interface
(API) design. This is also one of the reasons SymPy makes no attempt to extend
the Python language itself. The goal is for users of SymPy to be able to
import SymPy alongside other Python libraries in their workflow, whether that
is an interactive workflow or programmatic use as part of a larger system.

Being developed as a library, SymPy does not have a built-in graphical user
interface (GUI). However, SymPy exposes a rich interactive display system,
including registering printers with Jupyter~\cite{perez2007ipython} frontends,
including the Notebook and Qt Console, which will pretty print SymPy
expressions using MathJax~\cite{cervone2012mathjax} or \LaTeX{} rendering.

Section~\ref{sec:architecture} discusses the architecture of SymPy. Following
that, Section~\ref{sec:numerics} looks at the numerical features of SymPy and
its dependency library, mpmath. Section~\ref{sec:features} enumerates the
features of SymPy and takes a closer look at some of the important ones.
Section~\ref{sec:domain_specific} looks at the domain specific physics
submodules for doing classical mechanics and quantum mechanics. Finally,
Section~\ref{sec:conclusion} concludes the paper and discusses future work.


\section{Architecture}

\subsection{Basic Usage}

% symbols, various ways to declare them

Being built on Python, SymPy requires that all variable names be defined
before they can be used. The statement
\begin{verbatim}
>>> from sympy import *
\end{verbatim}
will import all SymPy functions into the global Python namespace. All the
examples in this paper assume that this has been run.

Symbolic variables, called symbols, must be defined and assigned to
Python variables before they can be used. This is typically done through the
\texttt{symbols} function, which creates multiple symbols at once. For
instance,
\begin{verbatim}
>>> x, y, z = symbols('x y z')
\end{verbatim}
creates three symbols representing variables named $x$, $y$, and $z$, assigned
to Python variables of the same name. The Python variable names that symbols
are assigned to are immaterial---we could have just as well have written
\verb|a, b, c = symbol('x y z')|. All the examples in this paper will assume
that the symbols \verb|x|, \verb|y|, and \verb|z| have been assigned as above.

Expressions are created from symbols using Python syntax, which mirrors usual
mathematical notation. Note that in Python, exponentiation is \verb|**|. For
instance, the following creates the expresion $(x^2 - 2x + 3)/y$.
\begin{verbatim}
>>> (x**2 - 2*x + 3)/y
(x**2 - 2*x + 3)/y
\end{verbatim}

All SymPy expressions are immutable. This simplifies the design by allowing
interning. It also allows expressions to be hashed and stored in a Python
dictionary, which enables caching and other features.

%% I volunteer to write this section. --Aaron
%%
%% Representing symbolic expressions using Python objects

\subsection{The Core}

The core of a computer algebra system (CAS) refers to the module that is in
charge of representing symbolic expressions and performing basic manipulations
with them. In SymPy, every symbolic expression is an instance of a Python class.
Expressions are represented by expression trees. The operators are represented
by the type of an expression and the child nodes are stored in the
\texttt{args} attribute. A leaf node in the expression tree has an empty
\texttt{args}.
The \texttt{args} attribute is provided by the class \texttt{Basic},
which is a superclass of all SymPy objects and
provides common methods to all SymPy tree-elements.
For example, consider the expression $xy + 2$:
\begin{verbatim}
>>> expr = x*y + 2
\end{verbatim}
By order of operations, the parent of the expression tree for \texttt{expr} is
an addition, so it is of type \texttt{Add}. The child nodes of \texttt{expr} are
\texttt{2} and \texttt{x*y}.
\begin{verbatim}
>>> type(expr)
<class 'sympy.core.add.Add'>
>>> expr.args
(2, x*y)
\end{verbatim}

One can dig further into the expression tree to see the full expression. For
example, the first child node, given by \texttt{expr.args[0]} is
\texttt{2}. Its class is \texttt{Integer}, and it has empty \texttt{args},
indicating that it is a leaf node.
\begin{verbatim}
>>> expr.args[0]
2
>>> type(expr.args[0])
<class 'sympy.core.numbers.Integer'>
>>> expr.args[0].args
()
\end{verbatim}

A useful way to view an expression tree is with the \texttt{srepr} function,
which returns a string representation of an expression as valid Python code
with all the nested class constructor calls to create the given expression.
\begin{verbatim}
>>> srepr(expr)
"Add(Mul(Symbol('x'), Symbol('y')), Integer(2))"
\end{verbatim}

Every SymPy expression satisfies a key invariant, namely,
\verb|expr.func(*expr.args) == expr|. This means that expressions are
rebuildable from their \texttt{args}~\footnote{\texttt{expr.func} is used
instead of \texttt{type(expr)} to allow the function of an expression to be
distinct from its actual Python class. In most cases the two are the same.}.
Here, we note that in SymPy, the \texttt{==} operator represents exact
structural equality, not mathematical equality. This allows one to test if
any two expressions are equal to one another as expression trees.

Python allows classes to override mathematical operators. The Python
interpreter translates the above \texttt{x*y + 2} to, roughly,
\verb|(x.__mul__(y)).__add__(2)|. Both \texttt{x} and \texttt{y}, returned
from the \texttt{symbols} function, are \texttt{Symbol} instances. The
\texttt{2} in the expression is processed by Python as a literal, and is
stored as Python's builtin \texttt{int} type. When \texttt{2} is passed to the
\verb|__add__| method of \texttt{Symbol}, it is converted to the SymPy type
\verb|Integer(2)| before being stored in the resulting expression tree. In
this way, SymPy expressions can be built in the natural way using Python
operators and numeric literals.

One must be careful in one particular instance: Python does not have a builtin
rational literal type. Given a fraction of integers such as \texttt{1/2},
Python will perform floating point division and produce
\texttt{0.5}~\footnote{This is the behavior in Python 3. In Python 2,
  \texttt{1/2} will perform integer division and produce \texttt{0}, unless
  one uses \texttt{from \_\_future\_\_ import division}.}. Python uses eager
evaluation, so expressions like \texttt{x + 1/2} will produce \texttt{x +
  0.5}, because by the time any SymPy function sees the \texttt{1/2} it will
have already been converted to \texttt{0.5} by Python. However, for a CAS like
SymPy, one typically wants to work with exact rational numbers whenever
possible. Working around this is simple, however: one can wrap one of the
integers with \texttt{Integer}, like \verb|x + Integer(1)/2|, or using
\verb|x + Rational(1, 2)|. SymPy provides a function \texttt{S} which can be
used to convert objects to SymPy types with minimal typing, such as
\verb|x + S(1)/2|. This gotcha is a small downside to using Python directly
instead of a custom domain specific language (DSL), and we consider it to be
worth it for the advantages of using Python.

%%
%% Assumptions
\subsection{Assumptions}

An important feature of the SymPy core is the assumptions system. The
assumptions system allows users to specify that symbols have certain common
mathematical properties, such as being positive, imaginary, or integral. SymPy
is careful to never perform simplifications on an expression unless the
assumptions allow them. For instance, the identity $\sqrt{t^2} = t$ holds if
$t$ is nonnegative ($t\ge 0$). If $t$ is real, the identity $\sqrt{t^2}=|t|$
holds. However, for general complex $t$, no such identity holds.

By default, SymPy performs all calculations assuming that symbols are
complex valued. This assumption makes it easier to treat mathematical problems
in full generality.
\begin{verbatim}
>>> t = Symbol('t')
>>> sqrt(t**2)
sqrt(t**2)
\end{verbatim}

By assuming the most general case, that symbols are complex by default, SymPy
avoids performing mathematically invalid operations. However, in many cases
users will wish to simplify expressions containing terms like $\sqrt{t^2}$.

Assumptions are set on \texttt{Symbol} objects when they are created. For
instance \verb|Symbol('t', positive=True)| will create a symbol named
\texttt{t} that is assumed to be positive.
\begin{verbatim}
>>> t = Symbol('t', positive=True)
>>> sqrt(t**2)
t
\end{verbatim}
Some of the common assumptions that SymPy allows are \texttt{positive},
\texttt{negative}, \texttt{real}, \texttt{nonpositive}, \texttt{nonnegative},
\texttt{real}, \texttt{integer}, and \texttt{commutative}~\footnote{If $A$ and
$B$ are Symbols created with \texttt{commutative=False} then SymPy will keep
$A\cdot B$ and $B\cdot A$ distinct.}. Assumptions on any object can be checked with the
\verb|is_|\texttt{\textit{assumption}} attributes, like \verb|t.is_positive|.

Assumptions are only needed to restrict a domain so that certain
simplifications can be performed. It is not required to make the domain match
the input of a function. For instance, one can create the object
$\sum_{n=0}^m f(n)$ as \verb|Sum(f(n), (n, 0, m))| without setting
\texttt{integer=True} when creating the Symbol object \texttt{n}.

The assumptions system additionally has deductive capabilities. The
assumptions use a three-valued logic using the Python builtin objects
\texttt{True}, \texttt{False}, and \texttt{None}. \texttt{None} represents the
``unknown'' case. This could mean that the given assumption could be either
true or false under the given information, for instance,
\verb|Symbol('x', real=True).is_positive| will give \texttt{None} because a real
symbol might be positive or it might not. It could also mean not enough is
implemented to compute the given fact. For instance,
\verb|(pi + E).is_irrational| gives \texttt{None}, because SymPy does not know
how to determine if $\pi + e$ is rational or irrational, indeed, it is an open
problem in mathematics.
% TODO: ref?


Basic implications between the facts are used to deduce assumptions. For
instance, the assumptions system knows that being an integer implies being
rational, so \verb|Symbol('x', integer=True).is_rational| returns
\texttt{True}. Furthermore, expressions compute the assumptions on themselves
based on the assumptions of their arguments. For instance, if \texttt{x} and
\texttt{y} are both created with \texttt{positive=True}, then
\verb|(x + y).is_positive| will be \texttt{True}.

SymPy also has an experimental assumptions system where facts are stored
separate from objects, and deductions are made with a SAT solver. We will not
discuss this system here.

%%
%% Extensibility
\subsection{Extensibility}

Extensibility is an important feature for SymPy. Because the same language,
Python, is used both for the internal implementation and the external usage by
users, all the extensibility capabilities available to users are also used by
functions that are part of SymPy.

The typical way to create a custom SymPy object is to subclass an existing
SymPy class, generally either \texttt{Basic}, \texttt{Expr}, or
\texttt{Function}. All SymPy classes used for expression trees~\footnote{Some
  internal classes, such as those used in the polynomial module, do not follow
  this rule for efficiency reasons.} should be subclasses of the base class
\texttt{Basic}, which defines some basic methods for symbolic expression
trees. \texttt{Expr} is the subclass for mathematical expressions that can be
added and multiplied together. Instances of \texttt{Expr} typically represent
complex numbers, but may also include other ``rings'' like matrix expressions.
Not all SymPy classes are subclasses of \texttt{Expr}. For instance, logic expressions, such
as \verb|And(x, y)| are subclasses of \texttt{Basic} but not of \texttt{Expr}.

The \texttt{Function} class is a subclass of \texttt{Expr} which makes it
easier to define mathematical functions called with arguments. This includes
named functions like $\sin(x)$ and $\log(x)$ as well as undefined functions
like $f(x)$. Subclasses of \texttt{Function} should define a
class method \texttt{eval}, which returns values for which the function should
be automatically evaluated, and \texttt{None} for arguments that should not be
automatically evaluated.

Many SymPy functions require various evaluations down the expression tree.  The
evaluation of such functions on of classes in SymPy is performed by defining a
relevant \verb|_eval_|\texttt{\textit{*}} method on the class. For instance, an
object can signal to SymPy's \texttt{diff} function how to take the derivative of
itself by defining the \verb|_eval_derivative(self, x)| method, which may in
turn call \texttt{diff} on its \texttt{args}. The most common
\verb|_eval_|\texttt{\textit{*}} methods relate to the assumptions.
\verb|_eval_is_|\texttt{\textit{assumption}} defines the assumptions for
\textit{assumption}.

As an example of the notions presented in this section, we present below
a stripped down version of the gamma function $\Gamma(x)$ from SymPy,
which evaluates itself on positive integer arguments, has the positive and
real assumptions defined, can be rewritten in terms of factorial with
\verb|gamma(x).rewrite(factorial)|, and can be differentiated.
\texttt{fdiff} is a convenience method for subclasses of \texttt{Function}.
\texttt{fdiff} returns the derivative of the function without worrying about
the chain rule. \texttt{self.func} is used throughout instead of referencing
\texttt{gamma} explicitly so that potential subclasses of \texttt{gamma} can
reuse the methods.
\begin{verbatim}
from sympy import Integer, Function, floor, factorial, polygamma

class gamma(Function)
    @classmethod
    def eval(cls, arg):
        if isinstance(arg, Integer) and arg.is_positive:
            return factorial(arg - 1)

    def _eval_is_real(self):
        x = self.args[0]
        # noninteger means real and not integer
        if x.is_positive or x.is_noninteger:
            return True

    def _eval_is_positive(self):
        x = self.args[0]
        if x.is_positive:
            return True
        elif x.is_noninteger:
            return floor(x).is_even

    def _eval_rewrite_as_factorial(self, z):
        return factorial(z - 1)

    def fdiff(self, argindex=1):
        from sympy.core.function import ArgumentIndexError
        if argindex == 1:
            return self.func(self.args[0])*polygamma(0, self.args[0])
        else:
            raise ArgumentIndexError(self, argindex)
\end{verbatim}
The actual gamma function defined in SymPy has many more capabilities, such as
evaluation at rational points and series expansion.


\section{Algorithms}

%% Description of some algorithms (example: integration with Risch, Meijer G, Gruntz, polys)
%%
%% Description of numerics/mpmath (Fredrik)

% A description of some of the algorithms in SymPy. The list is not
% exhaustive.

% The sections here are preliminary. We may end up needing to cut some of
% this.

\subsection{Numerics}

\subsection{Polynomials}

\subsection{The Risch Algorithm}

\subsection{The Gruntz Algorithm}

\subsection{Logic}

\subsection{Other}


\section{Features}

%% List of Features and how to use
%%
%% Quick overview of the main modules, what it can do and so on. It should probably provide examples how to use sympy.
%%
%% See also the supplement (below)

% Features to discuss in-depth:

SymPy has an extensive feature set that encompasses too much to cover
in-depth here. Bedrock areas, such as calculus, receive their own sub-sections
below. Additionally, Table~\ref{features-table} describes other capabilities
present in the SymPy code base. This gives a sampling from the breadth of
topics and application domains that SymPy services.


\begin{longtable}[htbc]{p{0.25\linewidth}p{0.68\linewidth}}
\caption{SymPy Features and Descriptions\label{features-table}}\\
\toprule
\textbf{Feature} & \textbf{Description} \\
\midrule
Discrete Math & Summations, products, binomial coefficients,
    prime number tools, integer factorization, Diophantine equation solving, and
    boolean logic representation, equivalence testing, and inference.\\
Concrete Math & Tools for determining whether summation and product
    expressions are convergent, absolutely convergent, hypergeometric, and
    other properties. May also compute Gosper's normal form~\cite{petkovvsek1996bak} for two univariate polynomials.\\
Plotting & Hooks for visualizing expressions via matplotlib~\cite{Hunter:2007}
    or as text drawings when lacking a graphical back-end.\\
Geometry & Allows the creation of 2D geometrical entities,
    such as lines and circles. Enables queries on these entities, including
    asking the area of an ellipse, checking for collinearity of a set of
    points, or finding the intersection between two lines.\\
Statistics & Support for a random variable type as well as the ability to
    declare this variable from prebuilt distribution functions such as
    Normal, Exponential, Coin, Die, and other custom distributions.\\
Polynomials & Computes polynomial algebras over various coefficient domains
    ranging from the simple (e.g., polynomial division) to the advanced
    (e.g., Gr\"obner bases~\cite{adams1994introduction} and multivariate
    factorization over algebraic number domains).\\
Sets & Representations of empty, finite, and infinite sets. This includes
    special sets such as for all natural, integer, and complex numbers.\\
Series & Implements series expansion, sequences, and limit of sequences.
    This includes special series, such as Fourier and power series.\\
Vectors & Provides basic vector math and differential calculus with respect
    to 3D Cartesian coordinate systems.\\
Matrices & Tools for creating matrices of symbols and expressions.
    This is capable of both sparse and dense representations and performing
    symbolic linear algebraic operations (e.g., inversion and factorization).\\
Combinatorics \& Group Theory & Implements permutations, combinations,
    partitions, subsets,
    various permutation groups (such as polyhedral, Rubik, symmetric,
    and others), Gray codes~\cite{Nijenhuis1978combinatorial},
    and Prufer sequences~\cite{biggs1976graph}.\\
Code Generation & Enables generation of compilable and executable
    code in a variety of different programming languages directly from
    expressions. Target languages include C, Fortran, Julia, JavaScript,
    Mathematica, Matlab and Octave, Python, and Theano.\\
Tensors & Symbolic manipulation of indexed objects.\\
Lie Algebras & Represents Lie algebras and root systems.\\
Cryptography & Represents block and stream ciphers, including
    shift, Affine, substitution, Vigenere's, Hill's, bifid, RSA, Kid RSA,
    linear-feedback shift registers, and Elgamal encryption\\
Special Functions & Implements a number of well known special functions,
    including Dirac delta, Gamma, Beta, Gauss error functions, Fresnel
    integrals, Exponential integrals, Logarithmic integrals, Trigonometric
    integrals, Bessel, Hankel, Airy, B-spline, Riemann Zeta, Dirichlet eta,
    polylogarithm, Lerch transcendent, hypergeometric, elliptic integrals,
    Mathieu, Jacobi polynomials, Gegenbauer polynomial, Chebyshev polynomial,
    Legendre polynomial, Hermite polynomial, Laguerre polynomial, and
    spherical harmonic functions.\\
\bottomrule

\end{longtable}

\subsection{Simplification}

% polynomial expressions

% functions

% expand( ), factor( ), collect( ), together( ), apart( )
%% maybe a table best suits this part.

% simplification: simplify, sqrt denest, fu, trigsimp

The generic way to simplify an expression is by calling the \texttt{simplify}
function.
It must be emphasized that simplification is not a rigously defined
mathematical operation~\cite{Carette2004understanding}.
The \texttt{simplify} function applies several simplification routines along
with heuristics to make the output expression as ``simple'' as possible.

It is often preferable to apply more directed simplification functions. These
apply very specific rules to the input expression and are typically able to make
guarantees about the output. For instance, the \texttt{factor} function,
given a polynomial with rational coefficients in several variables,
is guaranteed to
produce a factorization into irreducible factors. Table~\ref{simplify-table}
lists common simplification functions.

\begin{longtable}[htbc]{lp{0.83\linewidth}}
\caption{Some SymPy Simplification Functions\label{simplify-table}}\\
\toprule
\verb|expand| & expand the expression \\
\verb|factor| & factor a polynomial into irreducibles \\
\verb|collect| & collect polynomial coefficients \\
\verb|cancel| & rewrite a rational function as $p/q$ with common factors
canceled \\
\verb|apart| & compute the partial fraction decomposition of a rational function
\\
\verb|trigsimp| & simplify trigonometric expressions~\cite{fu2006automated} \\
\bottomrule
\end{longtable}


\subsection{Calculus}
\label{sec:calculus}
Integrals are calculated with the \verb|integrate| function. SymPy
implements a combination of the Risch
algorithm~\cite{bronstein2005integration}, table lookups, a reimplementation
of Manuel Bronstein's ``Poor Man's Integrator''~\cite{Bronstein2005pmint}, and
an algorithm for computing integrals based on Meijer G-functions~\cite{Roach1996hyper,roach1997meijerg}. These allow
SymPy to compute a wide variety of indefinite and definite integrals. The
Meijer G-function algorithm and the Risch algorithm are respectively
demonstrated below by the computation of \[\int_{0}^{\infty} e^{-s t}\log{\left (t \right )}\, dt = - \frac{ \log{\left (s \right )} + \gamma}{s}\] and \[\int \frac{- 2 x^{2} \left(\log{\left (x \right )} + 1\right) e^{x^{2}} + {\left(e^{x^{2}} + 1\right)}^{2}}{x {\left(e^{x^{2}} + 1\right)}^{2} \left(\log{\left (x \right )} + 1\right)}\, dx = \log{\left (\log{\left (x \right )} + 1 \right )} + \frac{1}{e^{x^{2}} + 1}.\]
\begin{verbatim}
>>> s, t = symbols('s t', positive=True)
>>> integrate(exp(-s*t)*log(t), (t, 0, oo)).simplify()
-(log(s) + EulerGamma)/s
>>> integrate((-2*x**2*(log(x) + 1)*exp(x**2) +
... (exp(x**2) + 1)**2)/(x*(exp(x**2) + 1)**2*(log(x) + 1)), x)
log(log(x) + 1) + 1/(exp(x**2) + 1)
\end{verbatim}

Derivatives are computed with the \verb|diff| function, which recursively uses
the various differentiation rules.
\begin{verbatim}
>>> diff(sin(x)*exp(x), x)
exp(x)*sin(x) + exp(x)*cos(x)
\end{verbatim}

Summations are computed with \verb|summation|  using a combination of Gosper's
algorithm~\cite{gosper1978decision}, an algorithm that uses Meijer
G-functions~\cite{Roach1996hyper,roach1997meijerg}, and heuristics. Products
are computed with \verb|product| function via a suite of heuristics.
% TODO: Are there other summation algorithms implemented?
% TODO: A good summation example or two
\begin{verbatim}
>>> i, n = symbols('i n')
>>> summation(2**i, (i, 0, n - 1))
2**n - 1
>>> summation(i*factorial(i), (i, 1, n))
n*factorial(n) + factorial(n) - 1
\end{verbatim}

Limits are computed with the \verb|limit| function. The limit module
implements the Gruntz algorithm~\cite{Gruntz1996limits} for computing symbolic
limits.
For example, the following computes
$\lim\limits_{x\to \infty} x\sin(\frac{1}{x})=1$. Note that SymPy denotes
$\infty$ as \verb|oo|.
\begin{verbatim}
>>> limit(x*sin(1/x), x, oo)
1
\end{verbatim}
As a more complex example, SymPy computes \[\lim\limits_{x\to 0}{\left(2 e^{\frac{1 - \cos{\left (x \right )}}{\sin{\left (x \right )}}} -
  1\right)}^{\frac{\sinh{\left (x \right )}}{\operatorname{atan}^{2}{\left (x
      \right )}}} = e.\]
\begin{verbatim}
>>> limit((2*E**((1-cos(x))/sin(x))-1)**(sinh(x)/atan(x)**2), x, 0)
E
\end{verbatim}

Integrals, derivatives, summations, products, and limits that cannot be
computed return unevaluated objects. These can also be created directly if the
user chooses.
\begin{verbatim}
>>> integrate(x**x, x)
Integral(x**x, x)
>>> Sum(2**i, (i, 0, n - 1))
Sum(2**i, (i, 0, n - 1))
\end{verbatim}


\subsection{Printers}


SymPy has a rich collection of expression printers for displaying expressions
to the user. By default, an interactive Python session will render the
\verb|str| form of an expression, which has been used in all the examples in
this paper so far. The \verb|str| form of an expression is valid Python and
roughly matches what a user would type to enter the expression.

\begin{verbatim}
>>> phi0 = Symbol('phi0')
>>> str(Integral(sqrt(phi0), phi0))
'Integral(sqrt(phi0), phi0)'
\end{verbatim}

Expressions can be printed with 2D monospace text with \verb|pprint|. This
uses Unicode characters to render mathematical symbols such as integral signs,
square roots, and parentheses. Greek letters and subscripts in symbol names
are rendered automatically.

\noindent
\includegraphics[width=1\textwidth]{pprint.pdf}
Alternately, the \verb|use_unicode=False| flag can be set, which causes the
expression to be printed using only ASCII characters.

\begin{verbatim}
>>> pprint(Integral(sqrt(phi0 + 1), phi0), use_unicode=False)
  /
 |
 |   __________
 | \/ phi0 + 1  d(phi0)
 |
/
\end{verbatim}

The function \verb|latex| returns a \LaTeX{} representation of an expression.

\begin{verbatim}
>>> print(latex(Integral(sqrt(phi0 + 1), phi0)))
\int \sqrt{\phi_{0} + 1}\, d\phi_{0}
\end{verbatim}

Users are encouraged to run the \verb|init_printing| function at the beginning
of interactive sessions, which automatically enables the best pretty printing
supported by their environment. In the Jupyter Notebook or Qt
Console~\cite{perez2007ipython} the \LaTeX{} printer is used to render
expressions using MathJax or \LaTeX{}, if it is installed on the system. The
2D text representation is used otherwise.

Other printers such as MathML are also available. SymPy uses an extensible
printer subsystem which allows users to customize the printing for any given
printer, and for custom objects to define their printing behavior for any
printer. SymPy's code generation capabilities, which we will not discuss
in-depth here, use this subsystem to convert expressions into code in various
languages.


% Solvers (regular equations, maybe also mention other types of solvers like ODEs/recurrence/Diophantine)
\subsection{Solvers}
SymPy includes a module of equation solvers for symbolic equations. There are two
functions for solving algebraic equations in SymPy: \texttt{solve}
and \texttt{solveset}.
\texttt{solveset} has several design changes with respect to the older
\texttt{solve} function. This distinction is present in order to resolve the
usability issues with the
previous \texttt{solve} function API while maintaining backward compatibility
with earlier versions of SymPy.
\texttt{solveset} only requires essential input information from the user.
The function signatures of \texttt{solve} and \texttt{solveset} are
\begin{verbatim}
solve(f, *symbols, **flags)
solveset(f, symbol, domain=S.Complexes)
\end{verbatim}
The \texttt{domain} parameter is typically either \texttt{S.Complexes} (the
default) or \texttt{S.Reals}; the latter causes \texttt{solveset} to only return real solutions.
Both functions implicitly assume that expressions are equal to 0. For
instance, \texttt{solveset(x - 1, x)} solves $x - 1 = 0$ for $x$.

It should be noted that \texttt{solve} has an inconsistent output API for different types
of inputs. For instance, depending on the input, it may return a Python
list or a Python dictionary. On the other hand,
\texttt{solveset} has a canonical output API: it always returns
a SymPy set object.

\texttt{solveset} is under active development as a planned replacement for
\texttt{solve}. There are certain features which are implemented in
\texttt{solve} that are not yet implemented in \texttt{solveset}. Notably,
these include nonlinear multivariate system and transcendental equations.
More examples of \texttt{solveset} and \texttt{solve} can be found in the
supplement.


% Matrices (worth including to stress that they are symbolic)
\subsection{Matrices}

Besides being an important feature in its own right, computations on
matrices with symbolic entries are important for many algorithms
within SymPy.  The following code shows some basic usage of the
\texttt{Matrix} class.
\begin{verbatim}
>>> A = Matrix([[x, x + y], [y, x]])
>>> A
Matrix([
[x, x + y],
[y,     x]])
\end{verbatim}

SymPy matrices support common symbolic linear algebra manipulations, including
matrix addition, multiplication, exponentiation, computing determinants,
solving linear systems, and computing inverses using LU decomposition, LDL
decomposition, Gauss-Jordan elimination, Cholesky decomposition, Moore-Penrose
pseudoinverse, singular values, and adjugate matrix.

All operations are performed symbolically. For instance, eigenvalues are computed
by generating the characteristic polynomial using the Berkowitz algorithm and
then solving it using polynomial routines.

\begin{verbatim}
>>> A.eigenvals()
{x - sqrt(y*(x + y)): 1, x + sqrt(y*(x + y)): 1}
\end{verbatim}

Internally these matrices store the elements as Lists of Lists (LIL), meaning
the matrix is stored as a list of lists of entries (effectively, the
input format used to create the matrix \texttt{A} above), making it a
dense representation.\footnote{Similar to the polynomials module, dense here
  means that all entries are stored in memory, contrasted with a sparse
  representation where only nonzero entries are stored.} For storing sparse
matrices, the \verb|SparseMatrix| class can be used. Sparse matrices store
their elements in Dictionary of Keys (DOK) format, meaning entries are stored
as \texttt{(row, column)} pairs mapping to the elements.

SymPy also supports matrices with symbolic dimension values. \verb|MatrixSymbol|
represents a matrix with dimensions $m\times n$, where $m$ and $n$ can be
symbolic. Matrix addition and multiplication, scalar operations, matrix inverse,
and transpose are stored symbolically as matrix expressions.

Block matrices are also implemented in SymPy. \verb|BlockMatrix| elements can
be any matrix expression, including explicit matrices, matrix symbols, and
other block matrices. All functionalities of matrix expressions are also
present in \verb|BlockMatrix|.

When symbolic matrices are combined with the assumptions module for logical
inference, they provide powerful reasoning over invertibility,
semi-definiteness, orthogonality, etc., which are valuable in the construction
of numerical linear algebra systems.

More examples for \verb|Matrix| and \verb|BlockMatrix| may be found in the
supplement.



SymPy includes several packages that allow users to solve domain specific
problems. For example, a comprehensive physics package is included that is
useful for solving problems in classical mechanics, optics, and quantum
mechanics along with support for manipuating physical quantities with units.

\subsection{Vector Algebra}

The \verb|sympy.physics.vector| package provides reference frame, time, and
space aware vector and dyadic objects that allow for three dimensional
operations such as addition, subtraction, scalar multiplication, inner and
outer products, cross products, etc. Both of these objects can be written in
very compact notation that make it easy to express the vectors and dyadics in
terms of multiple reference frames with arbitrarily defined relative
orientations. The vectors are used to specify the positions, velocities, and
accelerations of points, orientations, angular velocities, and angular
accelerations of reference frames, and force and torques. The dyadics are
essentially reference frame aware $3 \times 3$ tensors. The vector and dyadic
objects can be used for any one-, two-, or three-dimensional vector algebra and
they provide a strong framework for building physics and engineering tools.

The following Python interpreter session showing how a vector is created using
the orthogonal unit vectors of three reference frames that are oriented with
respect to each other and the result of expressing the vector in the $A$
frame. The $B$ frame is oriented with respect to the $A$ frame using Z-X-Z
Euler Angles of magnitude $\pi$, $\frac{\pi}{2}$, and
$\frac{\pi}{3}$\si{\radian}, respectively whereas the $C$ frame is oriented
with respect to the $B$ frame through a simple rotation about the $B$ frame's
X unit vector through $\frac{\pi}{2}$\si{\radian}.

\begin{verbatim}
>>> from sympy import pi
>>> from sympy.physics.vector import ReferenceFrame
>>> A = ReferenceFrame('A')
>>> B = ReferenceFrame('B')
>>> C = ReferenceFrame('C')
>>> B.orient(A, 'body', (pi, pi / 3, pi / 4), 'zxz')
>>> C.orient(B, 'axis', (pi / 2, B.x))
>>> v = 1 * A.x + 2 * B.z + 3 * C.y
>>> v
A.x + 2*B.z + 3*C.y
>>> v.express(A)
A.x + 5*sqrt(3)/2*A.y + 5/2*A.z
\end{verbatim}

\subsection{Classical Mechanics}

The \verb|physics.mechanics| package utilizes the \verb|physics.vector| package
to populate time aware particle and rigid body objects to fully describe the
kinematics and kinetics of a rigid multi-body system. These objects store all
of the information needed to derive the ordinary differential or differential
algebraic equations that govern the motion of the system, i.e., the equations
of motion. These equations of motion abide by Newton's laws of motion and can
handle any arbitrary kinematical constraints or complex loads. The package
offers two automated methods for formulating the equations of motion based on
Lagrangian Dynamics~\cite{Lagrange1811} and Kane's Method~\cite{Kane1985}. Lastly, there
are automated linearization routines for constrained dynamical
systems based on~\cite{Peterson2014}.

\subsection{Quantum Mechanics}

The \verb|sympy.physics.quantum| package provides quantum functions, states,
operators, and computation of standard quantum models.

% TODO : This needs some help from someone that knows something about quantum
% physics. I wasn't able to understand much from the documentation.

\subsection{Optics}

The \verb|physics.optics| package provides Gaussian optics functions.

% TODO : This needs some help from someone that knows something about optics.

\subsection{Units}

The \verb|physics.units| module provides around two hundred predefined prefixes
and SI units that are commonly used in the sciences. Additionally, it provides
the \verb|Unit| class which allows the user to define their own units.  These
prefixes and units are multiplied by standard SymPy objects to make expressions
unit aware, allowing for algebraic and calculus manipulations to be applied to
the expressions while the units are tracked in the manipulations.  The units of
the expressions can be easily converted to other desired units.  There is also
a new units system in \verb|sympy.physics.unitsystems| that allows the user to
work in specified unit systems.

\subsection{Tensors}

Ongoing work to provide the capabilities of tensor computer algebra has so far
produced the \verb|tensor| module.  It is composed of three separated
submodules, whose purposes are quite different: support indexed symbols,
$N$-dimensional arrays and abstract tensors.  The abstract tensors subsection
is inspired by xAct\cite{xAct} and Cadabra\cite{Peeters2007cadabra}.
Canonicalization based on the Butler-Portugal\cite{ManssurPortugal1999}
algorithm is supported in SymPy.  It is currently limited to polynomial tensor
expressions.



\section{Other Projects that use SymPy}

There are several projects that depend on SymPy as a library for implementing
a part of their functionality. Some of them are listed below:

\begin{itemize}
\item
  \href{http://cadabra.science/index.html}{\textbf{Cadabra}}: Cadabra is
  a CAS designed specifically for the
  resolution of problems encountered in field theory.
\item
  \href{http://octave.sourceforge.net/symbolic/}{\textbf{Octave Symbolic}}:
  The Octave-Forge Symbolic package adds symbolic calculation features
  to GNU Octave. These include common CAS tools such
  as algebraic operations, calculus, equation solving, Fourier and
  Laplace transforms, variable precision arithmetic, and other features.
\item
  \href{https://github.com/jverzani/SymPy.jl}{\textbf{SymPy.jl}}:
  Provides a Julia interface to SymPy using PyCall.
\item
  \href{https://mathics.github.io/}{\textbf{Mathics}}: Mathics is a
  free, general-purpose online CAS featuring Mathematica compatible
  syntax and functions. It is backed by highly extensible Python code,
  relying on SymPy for most mathematical tasks.
\item
  \href{http://mathpix.com/}{\textbf{Mathpix}}: An iOS App, that detects handwritten math as input, and uses
  SymPy Gamma to evaluate the math input and generate the relevant
  steps to solve the problem.
\item
  \href{http://openrave.org/docs/0.8.2/openravepy/ikfast/}{\textbf{IKFast}}:
  IKFast is a robot kinematics compiler provided by
  \href{http://openrave.org/}{OpenRAVE}. It analytically solves robot inverse
  kinematics equations and generates optimized C++ files. It uses SymPy for
  its internal symbolic mathematics.
\item
  \href{http://www.sagemath.org/}{\textbf{Sage}}: A CAS, visioned to be
  a viable free open source alternative to Magma, Maple, Mathematica and
  MATLAB\@. Sage includes many open source mathematical libraries, including
  SymPy.
\item
  \href{https://cloud.sagemath.com}{\textbf{SageMathCloud}}:
  SageMathCloud is a web-based cloud computing and course management
  platform for computational mathematics.
\item
  \href{http://www.pydy.org/}{\textbf{PyDy}}: Multibody Dynamics with
  Python.
\item
  \href{https://github.com/brombo/galgebra}{\textbf{galgebra}}:
  Geometric algebra (previously \texttt{sympy.galgebra}).
\item
  \href{http://yt-project.org/}{\textbf{yt}}: Python package for
  analyzing and visualizing volumetric data (\texttt{yt.units} uses SymPy).
\item
  \href{http://sfepy.org/}{\textbf{SfePy}} (Simple finite elements in Python),
  cf.~\cite{cimrman2014sfepy}, is a Python package for solving partial
  differential equations (PDEs) in 1D, 2D and 3D by the finite element (FE)
  method~\cite{Zienkiewicz2013FEM}. SymPy is used within this package mostly for
  code generation and testing.
\item
  \href{http://quameon.sourceforge.net/}{\textbf{Quameon}}: Quantum
  Monte Carlo in Python.
\item
  \href{http://lcapy.elec.canterbury.ac.nz/}{\textbf{Lcapy}}:
  Experimental Python package for teaching linear circuit analysis.
\item
  \href{http://digitalcommons.calpoly.edu/cgi/viewcontent.cgi?article=1072\&context=physsp/}{\textbf{Quantum
  Programming in Python}}: Quantum 1D Simple Harmonic Oscillator and
  Quantum Mapping Gate.
\item
  \href{http://mech.fsv.cvut.cz/~stransky/software/latexexpr/doc/}{\textbf{LaTeX
  Expression project}}: Easy \LaTeX{} typesetting of algebraic expressions
  in symbolic form with automatic substitution and result computation.
\item
  \href{https://www.researchgate.net/publication/260585491_Symbolic_Statistics_with_SymPy/}{\textbf{Symbolic
  statistical modeling}}: Adding statistical operations to complex
  physical models.
\end{itemize}


\section{Comparison with other CAS}


Wolfram Mathematica is a popular proprietary CAS.\@
It features highly advanced algorithms.
Mathematica has a core implemented in C++~\cite{Wolfram2016}
which interprets its own programming language (know as Wolfram language).

% M-expressions

Analogously to Lisp's S-expressions,
Mathematica uses its own style of M-expressions,
which are arrays of either atoms or other M-expression.
The first element of the expression identifies the type of the expression
and is indexed by zero, whereas the first argument is indexed by one.
Notice that SymPy expression arguments are stored in a Python tuple
(that is, an immutable array),
while the expression type is identified by the type of the object storing the
expression.

% Attributes

Mathematica can associate attributes to its atoms.
Attributes may define mathematical properties and behavior of the nodes
associated to the atom.
In SymPy, the usage of static class fields is roughly similar to Mathematica's
attributes, though other programming patterns may also be used the achieve an
equivalent behavior, such as class inheritance.

% Expression mutability

Unlike SymPy, Mathematica's expressions are mutable,
that is one can change parts of the expression tree without the need of
creating a new object.
The reactivity of Mathematica allows for a lazy updating of any references
to that data structure.

% * comparison with Mathematica: commutativity, associative expressions, one-identity. Advantage of SymPy: multiplicative commutativity defined on symbols.
% Products and commutativity

Products in Mathematica are determined by some builtin node types,
such as \texttt{Times}, \texttt{Dot}, and others.
\texttt{Times} is overloaded by the * operator,
and is always meant to represent a commutative operator.
The other notable product is \texttt{Dot}, overloaded by the \texttt{.} operator.
This product represents matrix multiplication,
it is not commutative.
SymPy uses the same node for both scalar and matrix multiplication,
the only exception being with abstract matrix symbols.
Unlike Mathematica, SymPy determines commutativity with respect to
multiplication from the factor's expression type.
Mathematica puts the \texttt{Orderless} attribute on the expression
type.

% Associative expressions.

Regarding associative expressions,
SymPy handles associativity by making associative expressions inherit the
class \texttt{AssocOp},
while Mathematica specifies the \texttt{Flat}\cite{WolframRefFlat} attribute on the expression type.

% One identity


% Pattern matching

Mathematica relies heavily on pattern matching:
even the so-called equivalent of function declaration is in reality
the definition of a pattern matching generating an expression tree transformation
on input expressions.
%
Mathematica's pattern matching is sensitive to associative\cite{WolframRefFlat}, commutative\cite{WolframRefOrderless},
and one-identity\cite{WolframRefOneIdentity} properties of its expression tree nodes\cite{WolframRefFlatAndOrderlessFunctions}.
%
SymPy has various ways to perform pattern matching.
All of them play a lesser role in the CAS than in Mathematica
and are basically available as a tool to rewrite expressions.
The differential equation solver in SymPy somewhat relies on pattern matching to
identify the kind of differential equation, but it is envisaged to replace
that strategy with analysis of Lie symmetries in the future.
Mathematica's real advantage is the ability to add new overloading to the
expression builder at runtime, or for specific subnodes.
Consider for example
\begin{verbatim}
In[1]:= Unprotect[Plus]

Out[1]= {Plus}

In[2]:= Sin[x_]^2 + Cos[y_]^2 := 1

In[3]:= x + Sin[t]^2 + y + Cos[t]^2

Out[3]= 1 + x + y
\end{verbatim}
This expression in Mathematica defines a substitution rule that overloads
the functionality of the \texttt{Plus} node (the node for additions in Mathematica).
The trailing underscore after a symbol means that it is to be considered a
wildcard.
This example may not be practical, one may wish to keep this identity
unevaluated, nevertheless it clearly illustrates the potentiality to define
one's own immediate transformation rules.
In SymPy the operations constructing the addition node in the expression tree
are Python class constructors,
and cannot be modified at runtime.\footnote{In reality, Python supports monkey patching,
nonetheless it is a discouraged programming pattern.}
The way SymPy deals with extending the missing runtime overloadability functionality
is by subclassing the node types.
Subclasses may overload the class constructor to yield the proper
extended functionality.


%% TODO list:
% * comparison with Mathematica: MatrixExp, product not always commutative, type inheritance (polymorphism) and advantage in unifying the product symbol * for symbols and matrices, pattern matching vs. single dispatch.

% Type inheritance and polymorphism

Unlike SymPy, Mathematica does not support type inheritance or polymorphism~\cite{Fateman1992}.
% cite examples of class inheritance in SymPy:
%
SymPy relies heavily on class inheritance, but for the most part,
class inheritance is used to make sure that SymPy objects inherit the proper
methods and implement the basic hashing system.
Associativity of expressions can be achieved by inheriting the class \texttt{AssocOp},
which may appear a more cumbersome operation than Mathematica's attribute setting.
%There are also cases where inheritance is used to extend the mathematical meaning of an expression.

% Matrices

Matrices in SymPy are types on their own.
In Mathematica, nested lists are interpreted as matrices whenever the sublists
have the same length.
The main difference to SymPy is that ordinary operators and functions
do not get generalized the same way as used in traditional mathematics.
Using the standard multiplication in Mathematica performs an elementwise
product, this is compatible with Mathematica's convention of commutativity of
\texttt{Times} nodes.
Matrix product is expressed by the \textit{dot} operator,
or the \texttt{Dot} node.
The same is true for the other operators, and even functions,
most notably calling the exponential function \texttt{Exp} on a matrix
returns an elementwise exponentiation of its elements.
The real matrix exponentiationl is available through the \texttt{MatrixExp}
function.

% * comparison with Mathematica: avoid misspelling variables through forced declaration (check that you can't do it in Mathematica).
% * evaluate=False vs HoldForm

Unevaluated expressions can be achieved in various ways,
most commonly with the \texttt{HoldForm} or \texttt{Hold} nodes,
that block the evaluation of subnodes by the parser.
Note that such a node cannot be expressed in Python, because of greedy evaluation.
Whenever needed in SymPy, it is necessary to add the parameter \texttt{evaluate=False}
to all subnodes, or put the input expression in a string.

% * comparison with Mathematica: == is structural equality, not

The operator == returns a boolean whenever it is able to immediately evaluate
the truthness of the equality, otherwise it returns an \texttt{Equal} expression.
In SymPy == means structural equality and is always guaranteed to return a
boolean expression.
To express an equality in SymPy it is necessary to explicitly construct the
\texttt{Equality} class.

% * comparison with Mathematica: polynomial module.
% * comparison with Mathematica: space is product, ** vs ^

SymPy, in accordance with Python and unlike the usual programming convention,
uses ** to express the power operator, while Mathematica uses the more
common \verb|^|.

% * comparison with Mathematica: ( ) is Sequence, functions are generally uppercase.
% * comparsion with Mathematica: table of comparison?
% * comparison with Mathematica: Wolfram language has loads of operator overloading, functional paradigm.


\section{Conclusion and future work}

SymPy is a robust computer algebra system that provides a wide spectrum of
features both in traditional computer algebra and in a plethora of scientific
disciplines. This allows SymPy to be used in a first-class way with other
Python projects, including the scientific Python stack. Unlike many other CASs, SymPy
is designed to be used in an extensible way: both as an end-user
application and as a library.

SymPy expressions are immutable trees of Python objects. SymPy uses Python both
as the internal language and the user language. This permits users to access to
the same
methods that the library implements in order to extend it for their needs.
Additionally, SymPy has a powerful assumptions
system for declaring and deducing mathematical properties of expressions.

SymPy supports a wide array of mathematical facilities. This includes functions for
simplifying expressions, performing common calculus operations, pretty printing
expressions, solving equations, and representing symbolic matrices. Other supported
facilities
include discrete math, concrete math, plotting, geometry, statistics,
polynomials, sets, series, vectors, combinatorics, group theory, code
generation, tensors, Lie algebras, cryptography, and special functions.
Additionally, SymPy contains submodules targeting certain specific domains,
such as classical mechanics and quantum mechanics.  This breadth of domains has
been engendered by a strong and vibrant user community.
Anecdotally, these users likely chose SymPy because of its ease of access.

% Future work:

Some of the planned future work for SymPy includes work on improving code
generation, improvements to the speed of SymPy using SymEngine, improving the
assumptions system, and improving the solvers submodule.

% TODO: Maybe one sentence for each item

% Feel free to add stuff here.


\section{References}

\bibliographystyle{siamplain}
\bibliography{paper}

\end{document}
