% Polynomials

SymPy implements a suite of algorithms for polynomial manipulation,
which ranges from relatively simple algorithms for doing arithmetic of
polynomials, to advanced methods for factoring multivariate polynomials
into irreducibles, symbolically determining real and complex root isolation
intervals, or computing Gr\"{o}bner bases.

Polynomial manipulation is useful on its own. Within SymPy, though, it is mostly used
indirectly as a tool in other areas of the library. In fact, many mathematical
problems in symbolic computing are first expressed using entities from the
symbolic core, preprocessed, and then transformed into a problem in the
polynomial algebra, where generic and efficient algorithms are used to solve
the problem and solutions to the original problem are recovered.
This is a common scheme in symbolic integration or summation algorithms.

SymPy implements dense and sparse polynomial representations. Both are used in
the univariate and multivariate cases. The dense representation is the default
for univariate polynomials. For multivariate polynomials, the choice of
representation is based on the application. The most common case for the sparse
representation is algorithms for computing Gr\"{o}bner bases (Buchberger, F4,
and F5).
% TODO: citation
This is
because different monomial orderings can be expressed easily in this
representation. However, algorithms for computing multivariate GCDs or
factorizations, at least those currently implemented in SymPy,
% TODO: citation
are better expressed when the representation is dense. The dense multivariate
representation is specifically a recursively dense representation, where
polynomials in $K[x_0, x_1, \dotsc, x_n]$ are viewed as a polynomials in
$K[x_0][x_1]\dotso[x_n]$. Note that despite this, the coefficient domain $K$,
can be a multivariate polynomial domain as well. The dense recursive
representation in Python gets inefficient when the number of variables gets
high.

Some examples for the \texttt{sympy.polys} module can be found in the
supplement.
