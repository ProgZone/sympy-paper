
SymPy has a rich collection of expression printers for displaying expressions
to the user. By default, an interactive Python session will render the
\verb|str| form of an expression, which has been used in all the examples in
this paper so far.

\begin{verbatim}
>>> phi0 = Symbol('phi0')
>>> str(Integral(sqrt(phi0), phi0))
Integral(sqrt(phi0 + 1), x)
\end{verbatim}

Expressions can be printed with 2D monospace text with \verb|pprint|. This
uses Unicode characters to render mathematical symbols such as integral signs,
square roots, and parentheses. Greek letters and subscripts in symbol names
are rendered automatically.

Alternately, the \verb|use_unicode=False| flag can be set, which causes the
expression to be printed using only ASCII characters.

\begin{verbatim}
>>> pprint(Integral(sqrt(phi0 + 1), phi0), use_unicode=False)
  /
 |
 |   __________
 | \/ phi0 + 1  d(phi0)
 |
/
\end{verbatim}

The function \verb|latex| returns a \LaTeX{} representation of an expression.

\begin{verbatim}
>>> print(latex(Integral(sqrt(phi0 + 1), phi0)))
\int \sqrt{\phi_{0} + 1}\, d\phi_{0}
\end{verbatim}

Users are encouraged to run the \verb|init_printing| function at the beginning
of interactive sessions, which automatically enables the best pretty printing
supported by their environment. In the Jupyter notebook or
qtconsole~\cite{perez2007ipython} the \LaTeX{} printer is used to render
expressions using MathJax or \LaTeX{} if it is installed on the system. The 2D
text representation is used otherwise.

Other printers such as MathML are also available. SymPy uses an extensible
printer subsystem which allows users to customize the printing for any given
printer, and for custom objects to define their printing behavior for any
printer. SymPy's code generation capabilities, which we will not discuss
in-depth here, use the same printer model.
