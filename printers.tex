SymPy has a rich collection of expression printers. By default, an interactive
Python session will render the \verb|str| form of an expression, which has
been used in all the examples in this paper so far. The \verb|str| form of an
expression is valid Python and roughly matches what a user would type to enter
the expression.\footnote{\label{note:repr}Many Python libraries distinguish the \texttt{str}
  form of an object, which is meant to be human-readable, and the
  \texttt{repr} form, which is mean to be valid Python that recreates the
  object. In SymPy, \texttt{str(expr) == repr(expr)}. In other words, the
  string representation of an expression is designed to be compact,
  human-readable, and valid Python code that could be used to recreate the
  expression. As noted in section~\ref{sec:core}, the \texttt{srepr}
  function prints the exact, verbose form of an expression.}

\begin{verbatim}
>>> phi0 = Symbol('phi0')
>>> str(Integral(sqrt(phi0), phi0))
'Integral(sqrt(phi0), phi0)'
\end{verbatim}

A two-dimensional (2D) textual representation of the expression can
be printed with monospace fonts via \verb|pprint|.
Unicode characters are used for rendering mathematical symbols such as integral signs,
square roots, and parentheses. Greek letters and subscripts in symbol names
that have Unicode code points associated
are also rendered automatically.
% TODO cite unicode?

\noindent
\includegraphics[width=1\textwidth]{pprint.pdf}
Alternately, the \verb|use_unicode=False| flag can be set, which causes the
expression to be printed using only ASCII characters.
% TODO cite ASCII

\begin{verbatim}
>>> pprint(Integral(sqrt(phi0 + 1), phi0), use_unicode=False)
  /
 |
 |   __________
 | \/ phi0 + 1  d(phi0)
 |
/
\end{verbatim}

The function \verb|latex| returns a \LaTeX{} representation of an expression.
% TODO cite latex

\begin{verbatim}
>>> print(latex(Integral(sqrt(phi0 + 1), phi0)))
\int \sqrt{\phi_{0} + 1}\, d\phi_{0}
\end{verbatim}

Users are encouraged to run the \verb|init_printing| function at the beginning
of interactive sessions, which automatically enables the best pretty printing
supported by their environment. In the Jupyter Notebook or Qt
Console~\cite{perez2007ipython}, the \LaTeX{} printer is used to render
expressions using MathJax or \LaTeX{}, if it is installed on the system. The
2D text representation is used otherwise.

Other printers such as MathML are also available. SymPy uses an extensible
printer subsystem for customizing any given
printer, and allows custom objects to define their printing behavior for any
printer. The code generation functionality of SymPy
relies on this subsystem to convert expressions into code in various target
programming languages.
