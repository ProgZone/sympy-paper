There are several projects that depend on SymPy as a library for implementing
a part of their functionality. A selection of these projects are listed in
Table~\ref{projects-table}.

\begin{longtable}[htbc]{>{\raggedright}p{0.14\linewidth}p{0.14\linewidth}p{0.63\linewidth}}
\caption{Selected projects that depend on SymPy.\label{projects-table}}\\
\toprule
\textbf{Project name} & \textbf{Domain} & \textbf{Description} \\
\midrule
\href{http://live.sympy.org/}{\textbf{SymPy Live}} &  & SymPy Live an online
  Python shell, which uses the Google App Engine to executes SymPy code. It is
  integrated in the SymPy documentation examples at
  \href{http://docs.sympy.org}{http://docs.sympy.org}. SymPy Live is
  maintained by the SymPy community. \\

\href{http://sympygamma.com/}{\textbf{SymPy Gamma}} &  & SymPy Gamma is a
  web application that executes and displays results for SymPy expressions, in
  a fashon similar to that of Wolfram|Alpha. SymPy Gamma is maintained by the
  SymPy community. See the supplementary material for more information about
  SymPy Gamma. \\

\href{http://cadabra.science/index.html}{\textbf{Cadabra}}~\cite{Peeters2007cadabra} &  &
  Cadabra is a CAS designed specifically for the resolution of problems
  encountered in field theory. \\

\href{https://github.com/cbm755/octsympy}{\textbf{OctSymPy}}~\cite{OctSymPy} &  &
  The OctSymPy package adds symbolic calculation features
  to GNU Octave. These include common CAS tools such
  as algebraic operations, calculus, equation solving, Fourier and
  Laplace transforms, variable precision arithmetic, and other features. \\

\href{https://github.com/jverzani/SymPy.jl}{\textbf{SymPy.jl}}~\cite{SymPy.jl} &  &
  Provides a Julia interface to SymPy using PyCall. \\

\href{https://mathics.github.io/}{\textbf{Mathics}}~\cite{Mathics} &  & Mathics is a
  free, general-purpose online CAS featuring Mathematica compatible
  syntax and functions. It is backed by highly extensible Python code,
  relying on SymPy for most mathematical tasks.[<8;32;30m \\

\href{http://mathpix.com/}{\textbf{Mathpix}}~\cite{Mathpix} &  & An iOS App, that detects handwritten math as input, and uses
  SymPy Gamma to evaluate the math input and generate the relevant
  steps to solve the problem. \\

\href{http://openrave.org/docs/latest_stable/openravepy/ikfast/}{\textbf{IKFast}}~\cite{diankov2010ikfast} &  &
  IKFast is a robot kinematics compiler provided by
  \href{http://openrave.org/}{OpenRAVE}. It analytically solves robot inverse
  kinematics equations and generates optimized C++ files. It uses SymPy for
  its internal symbolic mathematics. \\

\href{http://www.sagemath.org/}{\textbf{Sage}}~\cite{sagemath} &  & A CAS, visioned to be
  a viable free open source alternative to Magma, Maple, Mathematica and
  MATLAB\@. Sage includes many open source mathematical libraries, including
  SymPy. \\

\href{http://www.pydy.org/}{\textbf{PyDy}}~\cite{gede2013constrained} &  & Multibody Dynamics with
  Python. \\

\href{https://github.com/brombo/galgebra}{\textbf{galgebra}}~\cite{galgebra} &  &
  A module for geometric algebra (previously \texttt{sympy.galgebra}). \\

\href{http://yt-project.org/}{\textbf{yt}}~\cite{2011ApJS..192....9T} &  & Python package for
  analyzing and visualizing volumetric data (\texttt{yt.units} uses SymPy). \\

\href{http://sfepy.org/}{\textbf{SfePy}}~\cite{cimrman2014sfepy} &  & (Simple finite elements in Python),
  cf.~\cite{cimrman2014sfepy}, is a Python package for solving partial
  differential equations (PDEs) in 1D, 2D and 3D by the finite element (FE)
  method~\cite{Zienkiewicz2013FEM}. SymPy is used within this package mostly for
  code generation and testing. \\

\href{http://quameon.sourceforge.net/}{\textbf{Quameon}}~\cite{quameon} &  & Quantum
  Monte Carlo in Python. \\

\href{http://lcapy.elec.canterbury.ac.nz/}{\textbf{Lcapy}}~\cite{lcapy} &  &
  Experimental Python package for teaching linear circuit analysis. \\
\bottomrule
\end{longtable}
