There are several projects that depend on SymPy as a library for implementing
a part of their functionality. A selection of these projects are listed in
Table~\ref{projects-table}.

\begin{longtable}[htbc]{>{\raggedright}p{0.2\linewidth}p{0.8\linewidth}}
\caption{Selected projects that depend on SymPy.\label{projects-table}}\\
\toprule
\textbf{Project name} & \textbf{Description} \\
\midrule

\href{http://sympygamma.com/}{\textbf{SymPy Gamma}} & An open source
  analog of the Wolfram|Alpha.  See more in the supplementary material. \\

\href{http://cadabra.science/index.html}{\textbf{Cadabra}}~\cite{Peeters2007cadabra} &
  CAS designed specifically for the resolution of problems
  encountered in field theory. \\

\href{https://github.com/cbm755/octsympy}{\textbf{GNU Octave Symbolic Package}}~\cite{OctSymPy} &
  An implementation of a symbolic toolbox for Octave using SymPy. \\

\href{https://github.com/jverzani/SymPy.jl}{\textbf{SymPy.jl}}~\cite{SymPy.jl} &
  Provides a Julia interface to SymPy using PyCall. \\

\href{https://mathics.github.io/}{\textbf{Mathics}}~\cite{Mathics} &
  A free, online CAS featuring Mathematica compatible
  syntax and functions. \\

\href{http://mathpix.com/}{\textbf{Mathpix}}~\cite{Mathpix} & An iOS App, that detects handwritten math as input, and uses
  SymPy Gamma to evaluate the math input and generate the relevant
  steps to solve the problem. \\

\href{http://openrave.org/docs/latest_stable/openravepy/ikfast/}{\textbf{IKFast}}~\cite{diankov2010ikfast} &
  IKFast is a robot kinematics compiler provided by
  \href{http://openrave.org/}{OpenRAVE}. \\

\href{http://www.sagemath.org/}{\textbf{SageMath}}~\cite{sagemath} &
  A free open-source mathematics software system, builds on top of many
  existing open-source packages, including SymPy. \\

\href{http://www.pydy.org/}{\textbf{PyDy}}~\cite{gede2013constrained} & Multibody Dynamics with
  Python. \\

\href{https://github.com/brombo/galgebra}{\textbf{galgebra}}~\cite{galgebra} &
  A module for geometric algebra (previously \texttt{sympy.galgebra}). \\

\href{http://yt-project.org/}{\textbf{yt}}~\cite{2011ApJS..192....9T} & Python package for
  analyzing and visualizing volumetric data. \\

\href{http://sfepy.org/}{\textbf{SfePy}}~\cite{cimrman2014sfepy} &
  A Python package for solving partial
  differential equations (PDEs) in 1D, 2D and 3D by the finite element (FE)
  method~\cite{Zienkiewicz2013FEM}. \\

\href{http://quameon.sourceforge.net/}{\textbf{Quameon}}~\cite{quameon} & Quantum
  Monte Carlo in Python. \\

\href{http://lcapy.elec.canterbury.ac.nz/}{\textbf{Lcapy}}~\cite{lcapy} &
  Experimental Python package for teaching linear circuit analysis. \\
\bottomrule
\end{longtable}
