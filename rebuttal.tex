\documentclass[answers,12pt]{exam}
\usepackage{xcolor}
\usepackage{hyperref}

\unframedsolutions
\shadedsolutions
\renewcommand{\solutiontitle}{}

\begin{document}

Dear Editors, \bigskip

Please read below our answer about the
questions made by you and reviewers. I hope that you agree
with all our comments. \bigskip

Best regards,\bigskip

The authors

\pagebreak

\section{Comments from Editor}

\begin{questions}
\question I found the \url{http://live.sympy.org/} site very convenient for trying things
out in SymPy, but it is only mentioned in the supplement, which might be
overlooked by readers. I suggest moving that section into the main paper, as
it provides a great way to play along with the examples while reading the
paper.
\begin{solution}
  % TODO
\end{solution}
\question The Basic Usage section omits what I think is an important point:
how does one distinguish between evaluating exp(1) in Python and exp(1) in
SymPy? In other words, how are symbolic constants specified?
\begin{solution}
  % TODO. I'm unclear what the editor is asking here.
\end{solution}
\question The first
thing I tried after seeing Table 2 (simplication functions) did not work:

\begin{verbatim}
>>> trigsimp (exp( Matrix(2, 2, [0, -y, y, 0])) )
\end{verbatim}
fails to recognize cos and
sin. Am I expecting too much?
\begin{solution}
  This indeed does not work. The implementation in \verb|trigsimp| primarily
  thus far has focused on transforming trigonometric functions into other
  trigonometric functions. The ability to simplify complex exponentials is
  something that we would like to work.
  \href{https://github.com/sympy/sympy/issues/11459}{Issue 11459} in our
  public issue tracker tracks this feature.
\end{solution}
\question Section 3.6: are eigenvalues and singular
values included? If not, why not?
\begin{solution}
  % TODO
\end{solution}
\question Like one of the referees, I expected to
see MPFR mentioned in Section 4.1, at least to mention the pros and cons and
say why it isn’t used.
\begin{solution}
  % TODO
\end{solution}
\question Section 4.1: please state whether the syntax for
functions in mpmath is identical, or not, to that in Sympy.
\begin{solution}
  % TODO
\end{solution}
\question Unless I have
missed something, a weakness of the paper is that speed is mentioned only
twice, on lines 496 and 666. My assumption is that SymPy is slow compared with
its commercial competitors. Please comment on the speed issue, and not just in
the conclusions.
\begin{solution}
  % TODO
\end{solution}
\question Editors are needed for [20].
\begin{solution}
  % TODO: I do not know how to make bibtex (with apalike) show the editor
  % here.
\end{solution}
\end{questions}

\section{Reviewer 1 (James Davenport)}
\subsection{Basic reporting}
\begin{questions}
\question Generally good. A part that confused me is the assertion (footnote 3) that "If
A and B are Symbols created with commutative=False then SymPy will keep A·B
and B·A distinct." Does that mean that BOTH of them must be created this way,
and that A and x (if x is created normally) will commute? Is there any way to
declare commutators? How does one guarantee that other pieces of code, e.g.
Gaussian elimination, respect non-commutativity?
\begin{solution}
  % TODO.
\end{solution}

\question References reasonable, though [8], an excellent reference, makes almost
precisely the opposite point to that for which it is cited --- ``simplification
is not well-defined''. I think the authors are trying to follow [8]'s
definition of simplification.
\begin{solution}
  % TODO. I agree that [8] is a bad reference, because we are not actually
  % using their definition. The aim was just to have a reference for
  % "simplification is not well-defined", but that paper aims to define it.
  % What is a better reference here? Perhaps Moses [19] from the Carette
  % paper? Perhaps we can remove the reference (the referenced statement is
  % "it must be emphasized that simplification is not a rigorously defined
  % mathematical operation").
\end{solution}
\end{questions}

\subsection{Experimental design}
\begin{questions}
\question The referee is asked to comment whether the research is withinthe scope of the
journal, defined as "PeerJ is an Open Access, peer-reviewed, scholarly
journal. It considers articles in the Biological Sciences, Medical Sciences,
and Health Sciences. PeerJ does not publish in the Physical Sciences, the
Mathematical Sciences, the Social Sciences, or the Humanities (except where
articles in those areas have clear applicability to the core areas of
Biological, Medical or Health sciences)."

Hence I fear a software description on the maths/cmputing boundary with no
cited applications is out of scope.

\begin{solution}
  The editor has confirmed that the paper is in scope for PeerJ Computer Science.
\end{solution}
\end{questions}

\subsection{Validity of the findings}
\begin{questions}
\question There are no findings as such, since this is a software description, not an
experimental paper.
\end{questions}

\subsection{Comments for the author}
\begin{questions}
\question Nice paper - pity it seems totally out of scope. Why not J. Symbolic
Computation or some such?
\begin{solution}
  See above.
\end{solution}
\end{questions}

\section{Reviewer 2 (Michael Croucher)}

\subsection{Basic reporting}
No Comments.

\subsection{Experimental design}
No Comments

\subsection{Validity of the findings}
\begin{questions}
\question There is a minor problem with the supplementary notebook

The final cell is

\begin{verbatim}
>>> circuit_plot(fourier, nqubits=3);
plt.savefig('./images/circuitplot-qft.pdf', format='pdf')
\end{verbatim}

This fails if the user does not have an images folder.
\begin{solution}
We have added

\begin{verbatim}
%mkdir -p './images'
\end{verbatim}

to the cell before the call to \verb|savefig|.
\end{solution}

\question I suggest removing the >>> before each line. It looks strange in a notebook.
\begin{solution}
We have done this.
\end{solution}
\end{questions}
\subsection{Comments for the author}
Sympy is a superb package and I am happy to see that there is now a paper that describes its current state. Here are a few comments that you may wish to consider before publication.

\begin{questions}
\question Line 105
It is generally frowned upon to import all symbols from a Python module in this manner.
I understand why you are doing it in this paper, it makes subsequent sympy commands less verbose.
It might, however, encourage bad practice. It may also lead to a poor user experience for newbies.

For example, say I had the following code

\begin{verbatim}
𝔬mνmpyimportsin,arraytest=array(123)sin(test)
\end{verbatim}
[sic]

I decide that I want to use sympy for something and, following your example, do

\begin{verbatim}
𝔬mνmpyimportsin,array𝔬msympyimport⋅test=array(123)sin(test)x,y,z=symbols(′xyz′)
\end{verbatim}
[sic]

The code will break because sympy has it's own sin that gets imported that doesn't work on numpy arrays. As a newbie, I might not know this.
I tested using Sympy 1.0 and Python 3.

\begin{solution}
We have used \verb|import *| on purpose, as we felt that explicitly importing
names would unnecessarily distract from the content of the paper, but the
reviewer is absolutely right that this is bad practice for actual code. We
have added a footnote:

Later in the text, a similar assertion is made by you (line 485) in reference to a different package that may break sympy.

\begin{quote}\texttt{import *} has been used here to aid the
  readability of the paper, but is best to avoid such wildcard import
  statements in production code, as they make it unclear which names are
  present in the namespace. Furthermore, imported names could clash with
  already existing imports from another package. For example, SymPy, the
  standard Python \texttt{math} library, and NumPy all define the \texttt{exp}
  function, but only the SymPy one will work with SymPy symbolic expressions.
\end{quote}

We have also removed the reference to \texttt{from mpmath import *} from line 485.
\end{solution}

\question Lines 151-153
Minor comment: srepr is a useful command. As someone who also uses Mathematica, I wonder if there is a sympy version of the TreeForm command which produces a visualisation of the expression tree, or maybe an output format that I could pass to a graph library for visualisation?

\begin{solution}
We have added a footnote about the \texttt{dotprint} function (footnote 3),
which outputs expressions in the dot format that can be rendered with Graphviz.
\end{solution}

\question Section 2.3 - Assumptions
Is there a way of the user listing all available assumptions?

\begin{solution}

\end{solution}
\question Section 2.4
Line 216 - This is due in part because the same language, Python, is used both for the internal implementation and the external usage by users.

This reads badly. Perhaps the following might be better?

This is due, in part, to the fact that the same language, Python, is used both for the internal implementation and the external usage by users.

\begin{solution}

\end{solution}
\question Line 221:
the phrase 'Expression tree' is cited but this is not the first time you've used it. Perhaps cite earlier? Line 129 perhaps?

\begin{solution}

\end{solution}
\question Page 8: Footnote 5
The line reads

The measure parameter of the simplify function lets specify the Python function used to determine how

should this be..

The measure parameter of the simplify function lets the user specify the Python function used to determine how

\begin{solution}

\end{solution}
\question Section 4.1
Should mpmath be cited here?

\begin{solution}

\end{solution}
\end{questions}

\section{Reviewer 3}

\begin{questions}
\question one
\begin{solution}
 We opted to follow the ABC style of ...
\end{solution}
\question  two
\begin{solution}
Section 2 has been updated and the discussion ...
\end{solution}
\end{questions}


\end{document}
