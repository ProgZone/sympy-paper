% Series expansion (Differentiate between the two approaches being used)
\subsubsection{Series Expansion}

SymPy is able to calculate the symbolic series expansion of an arbitrary series
or expression involving elementary and special functions and multiple
variables. For this it has two different implementations.

The first approach stores a series as a core object. Each function has its
specific implementation of its expansion which is able to evaluate the Puiseux
series expansion about any point.

\begin{verbatim}
>>> from sympy import symbols, series
>>> x, y = symbols('x, y')
>>> series(sin(x+y) + cos(x*y), x, 0, 2)
1 + sin(y) + x*cos(y) + O(x**2)
\end{verbatim}

The newer and much faster approach called Ring Series makes use of the
observation that a truncated Taylor series, is in fact a polynomial.
Ring Series uses the efficient representation and operations of sparse
polynomials. The choice of sparse polynomials is deliberate as it performs
well in a wider range of cases than a dense representation. Ring Series gives 
the user the freedom to choose the type of coefficients he wants to have in
his series, allowing the use of faster operations on certain types.

For this, several low level methods for expansion of elementary functions are
implemented using semi-numerical algorithms. All these support Puiseux series
expansion.

\begin{verbatim}
>>> from sympy import ring
>>> from sympy.polys.ring_series import rs_sin
>>> R, x = ring('x', QQ)
>>> rs_sin(x**2 + x, x, 5)
-1/2*x**4 - 1/6*x**3 + x**2 + x
\end{verbatim}

The function \texttt{rs\_series} makes use of these elementary functions to
intelligently expand an arbitrary SymPy expression. Currently it only supports
expansion about 0 and is under active development. Typical speedup over
the older SymPy implementation is in the range of 20-100x with larger
speedup observed with larger series.

\begin{verbatim}
>>> from sympy.polys.ring_series import rs_series
>>> from sympy.abc import a, b
>>> from sympy import sin, cos
>>> rs_series(sin(a + b), a, 4)
-1/2*(sin(b))*a**2 + (sin(b)) - 1/6*(cos(b))*a**3 + (cos(b))*a
\end{verbatim}

\subsubsection{Formal Power Series}

SymPy can be used for computing the Formal Power Series of a function.
The implementation is based on the algorithm described in the paper on Formal Power Series\cite{Gruntz93formalpower}.
The advantage of this approach is that an explicit formula for the coefficients
of the series expansion is generated rather than just computing a few terms.

\begin{verbatim}
>>> f = fps(sin(x), x, x0=0)
>>> f.truncate(6)
x - x**3/6 + x**5/120 + O(x**6)
>>> f[15]
-x**15/1307674368000
\end{verbatim}

\subsubsection{Fourier Series}

SymPy provides functionality to compute Fourier Series of a function using
the \texttt{fourier\_series} function. Under the hood it just computes $a0$, $an$, $bn$ using
standard integration formulas.

\begin{verbatim}
>>> L = symbols('L')
>>> f = fourier_series(2 * (Heaviside(x/L) - Heaviside(x/L - 1)) - 1, (x, 0, 2*L))
>>> f.truncate(3)
4*sin(pi*x/L)/pi + 4*sin(3*pi*x/L)/(3*pi) + 4*sin(5*pi*x/L)/(5*pi)
\end{verbatim}
