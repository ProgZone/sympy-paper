%% Sets

SymPy supports representation of a wide variety of set, this is achieved by
first defining abstract representation for a smaller number of atomic set
classes and then combining and transforming them using various set operations.

Each of the set class inherits from the base Set class and defines
rules to check membership of a SymPy object, to calculate union, intersection,
set different and power set. In cases we are not able to evaluate these
operations to atomic set classes they are represented as abstract unevaluated
objects.


\begin{itemize}

    % Description of Empty Set sounds too obvious, not sure if we need to keep
    % it.
    \item \textbf{Empty Set}: Nothing is a member of Empty Set. Union with
another set returns the other set and intersection leads to an Empty Set.

    \item \textbf{Universal Set}: Everything is a member of Universal Set.
Union of Universal Set with any set gives Universal Set and intersection leads
to the other set itself.

    \item \textbf{Finite Set} is functionally equivalent to python's set
object. Its members can be any object including strings and other sets
themselves.

    \item \textfb{Range} implements a range of integers and is defined by
specifying a start value, an end value and a step size. Range is functionally
equivalent to python's range except the fact that it accepts infinity at end
points allowing us to represent infinite ranges.


    \item \textbf{Real Interval} is specified by giving the start and end point
and specifying if it is open or closed in these respective ends.

\end{itemize}


%% Operations

\begin{itemize}
    \item Finite Union
    \item Finite Intersection
    \item Set Difference
    \item Product Set
    \item Image Set
    \item Condition Set
\end{itemize}


%% Explaining it later because it is a special case of Image Set rather being
%% something atomic

    \textbf{Complex Region}

%% Representations achievable through application of Operations on atomic set
%% types mentioned above.


%% Special Cases
