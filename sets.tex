%% Sets

SymPy supports representation of a wide variety of mathematical sets. This is
achieved by first defining abstract representations of atomic set classes and
then combining and transforming them using various set operations.

Each of the set classes inherits from the base class \texttt{Set} and defines
methods to check membership and calculate unions, intersections, and set
differences. When these methods are not able to evaluate to atomic set
classes, they are represented as abstract unevaluated objects.

SymPy has the following atomic set classes:

\begin{itemize}

    \item \verb|EmptySet| represents the empty set $\emptyset$.

    \item \verb|UniversalSet| is an abstract ``universal set'' for which
      everything is a member. The union of the universal set with any set
      gives the universal set and the intersection gives to the other set
      itself.

    \item \verb|FiniteSet| is functionally equivalent to Python's built
      in\texttt{set} object. Its members can be any SymPy object including
      other sets themselves.

    \item \verb|Integers| represents the set of Integers $\mathbb{Z}$.

    \item \verb|Naturals| represents the set of Natural numbers $\mathbb{N}$,
      i.e., the set of positive integers.

    \item \verb|Naturals0| represents the whole numbers, which are all the
      non-negative integers.

    \item \verb|Range| represents a range of integers. A range is defined by
      specifying a start value, an end value, and a step size. Range is
      functionally equivalent to Python's \texttt{range} except it supports
      infinite endpoints, allowing the representation of infinite ranges.

    \item \verb|Interval| represents an interval of real numbers. It is
      specified by giving the start and end point and specifying if it is open
      or closed in the respective ends.


\end{itemize}


%% Operations

Other than unevaluated classes of Union, Intersection and Set Difference
operations, we have following set classes.

\begin{itemize}

    \item \verb|ProductSet| defines the Cartesian product of two
        or more sets. The product set is useful when representing higher
        dimensional spaces. For example to represent a three-dimensional space
        we simply take the Cartesian product of three real sets.

      \item \verb|ImageSet| represents the image of a function when applied to
        a particular set. In notation, the image set of a function $F$ with
        respect to a set $S$ is $\{ F(x) | x \in S \}$. SymPy uses image sets
        to represent sets of infinite solutions equations such as $\sin(x)=0$.


      \item \verb|ConditionSet| represents subset of a set whose members
        satisfies a particular condition. In notation, the condition set of
        the set $S$ with respect to the condition $H$ is
        $\{x | H(x), x \in S \}$. SymPy uses condition sets to represent the
        set of solutions of equations and inequalities, where the equation or
        the inequality is the condition and the set is the domain being solved
        over.


\end{itemize}

A few other classes are implemented as special cases of the classes described
above. The set of real numbers, \verb|Reals| is implemented as a special case
of \verb|Interval|, $(-\infty, \infty)$. \verb|ComplexRegion| is implemented
as a special case of \verb|ImageSet|. \verb|ComplexRegion| supports both polar
and rectangular representation of regions on the complex plane.
