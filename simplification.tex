% polynomial expressions

% functions

% expand( ), factor( ), collect( ), together( ), apart( )
%% maybe a table best suits this part.

% simplification: simplify, sqrt denest, fu, trigsimp

The generic way to simplify an expression is by calling the \texttt{simplify}
function.
It must be emphasized that simplification is not a rigorously defined
mathematical operation~\cite{Moses1971simplification}.
The \texttt{simplify} function applies several simplification routines along
with heuristics to make the output expression ``simple''.\footnote{The
\texttt{measure} parameter of the \texttt{simplify}
function lets the user specify the Python function used to determine how complex an
expression is.  The default measure function returns the total number of
operations in the expression.}

It is often preferable to apply more directed simplification functions. These
apply very specific rules to the input expression and are typically able to make
guarantees about the output. For instance, the \texttt{factor} function,
given a polynomial with rational coefficients in several variables,
is guaranteed to
produce a factorization into irreducible factors. Table~\ref{simplify-table}
lists common simplification functions.

\begin{center}
\begin{table}[htbc]
\caption{Some SymPy Simplification Functions}\label{simplify-table}
\begin{tabular}{ll}
\toprule
\verb|expand| & expand the expression \\
\verb|factor| & factor a polynomial into irreducibles \\
\verb|collect| & collect polynomial coefficients \\
\verb|cancel| & rewrite a rational function as $p/q$ with common factors
canceled \\
\verb|apart| & compute the partial fraction decomposition of a rational function
\\
\verb|trigsimp| & simplify trigonometric expressions~\cite{fu2006automated} \\
\verb|hyperexpand| & expand hypergeometric functions~\cite{Roach1996hyper,roach1997meijerg} \\
\bottomrule
\end{tabular}
\end{table}
\end{center}

Examples for these simplification functions can be found in the supplement.
