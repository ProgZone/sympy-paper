
% polynomial expressions

% functions

% expand( ), factor( ), collect( ), together( ), apart( )
%% maybe a table best suits this part.

% simplification: simplify, sqrt denest, fu, trigsimp

The generic way to simplify an expression is by calling the \texttt{simplify}
function.
It must be emphasized that simplification is not an unambigously defined
mathematical operation~\cite{Carette2004understanding}.
The \texttt{simplify} function applies several simplification routines along
with some heuristics to make the output expression as ``simple'' as possible.

It is often preferable to apply more directed simplification functions. These
apply very specific rules to the input expression, and are often able to make
guarantees about the output (for instance, the \texttt{factor} function, given
a polynomial with rational coefficients in several variables, is guaranteed to
produce a factorization into irreducible factors). Table~\ref{simplify-table}
lists some common simplification functions.

% TODO: remove the empty lines between rows
\begin{longtable}[htbc]{lp{0.83\linewidth}}
\caption{Some SymPy Simplification Functions\label{simplify-table}}\\
\toprule
\verb|expand| & expand the expression
\begin{verbatim}
>>> expand((x + y)**3)
x**3 + 3*x**2*y + 3*x*y**2 + y**3
\end{verbatim}
\\
\verb|factor| & factor a polynomial into irreducibles
\begin{verbatim}
>>> factor(x**3 + 3*x**2*y + 3*x*y**2 + y**3)
(x + y)**3
\end{verbatim}
\\
\verb|collect| & collect polynomial coefficients
\begin{verbatim}
>>> collect(y*x**2 + 3*x**2 - x*y + x - 1, x)
x**2*(y + 3) + x*(-y + 1) - 1
\end{verbatim}
\\
\verb|cancel| & rewrite a rational function as $p/q$ with common factors
canceled
\begin{verbatim}
>>> cancel((x**2 + 2*x + 1)/(x**2 - 1))
(x + 1)/(x - 1)
\end{verbatim}
\\
\verb|apart| & compute the partial fraction decomposition of a rational function
\begin{verbatim}
>>> apart((x**3 + 4*x - 1)/(x**2 - 1))
x + 3/(x + 1) + 2/(x - 1)
\end{verbatim}
\\
\verb|trigsimp| & simplify trigonometric expressions~\cite{fu2006automated}
\begin{verbatim}
>>> trigsimp(cos(x)**2*tan(x) - sin(2*x))
-sin(2*x)/2
\end{verbatim}
\\
\bottomrule
\end{longtable}

Substitutions are performed through the \texttt{.subs} method.
\begin{verbatim}
>>> (sin(x) + x**2 + 1).subs(x, y + 1)
(y + 1)**2 + sin(y + 1) + 1
\end{verbatim}

%% TODO: add examples
