SymPy has a module of equation solvers for symbolic equations. There are two
functions for solving algebraic equations in SymPy: \texttt{solve}
and \texttt{solveset}.
\texttt{solveset} has several design changes with respect to the older
\texttt{solve} function. This distiction is present in order to resolve the
usability issues with the
previous \texttt{solve} function API while maintaining backward compatibility
with earlier versions of SymPy.
\texttt{solveset} only requires the necessary input information from the user.
The function signatures of \texttt{solve} and \texttt{solveset} are
\begin{verbatim}
solve(f, *symbols, **flags)
solveset(f, symbol, domain=S.Complexes)
\end{verbatim}
The \texttt{domain} parameter is typically either \texttt{S.Complexes} (the
default) or \texttt{S.Reals}, which causes it to only return real solutions.

Additionally, \texttt{solve} has an inconsistent output API for various types
of inputs. For instance, depending on the input, sometimes it returns a Python
list and sometimes it returns a Python dictionary. On the other hand, the
\texttt{solveset} has a canonical output API.\ \texttt{solveset} always returns
a SymPy set object.

Both functions implicitly assume that expressions are equal to 0. For
instance, \texttt{solveset(x - 1, x)} solves $x - 1 = 0$ for $x$.

\noindent Single solution:
\begin{verbatim}
>>> solveset(x - 1, x)
{1}
\end{verbatim}

\noindent Finite solution set, quadratic equation:
\begin{verbatim}
>>> solveset(x**2 - pi**2, x)
{-pi, pi}
\end{verbatim}

\noindent No solution:
\begin{verbatim}
>>> solveset(1, x)
EmptySet()
\end{verbatim}

\noindent Interval solution:
\begin{verbatim}
>>> solveset(x**2 - 3 > 0, x, domain=S.Reals)
(-oo, -sqrt(3)) U (sqrt(3), oo)
\end{verbatim}

\noindent Infinitely many solutions:
% SymPy 1.0 sstr() prints S.Complexes incorrectly
% no-doctest
\begin{verbatim}
>>> solveset(sin(x) - 1, x, domain=S.Reals)
ImageSet(Lambda(_n, 2*_n*pi + pi/2), Integers())
>>> solveset(x - x, x, domain=S.Reals)
(-oo, oo)
>>> solveset(x - x, x, domain=S.Complexes)
S.Complexes
\end{verbatim}

Linear systems are solved with \texttt{linsolve}. Finite and infinite solution for
determined, under determined, and over determined problems are supported.
\begin{verbatim}
>>> A = Matrix([[1, 2, 3], [4, 5, 6], [7, 8, 10]])
>>> b = Matrix([3, 6, 9])
>>> linsolve((A, b), x, y, z)
{(-1, 2, 0)}
>>> linsolve(Matrix(([1, 1, 1, 1], [1, 1, 2, 3])), (x, y, z))
{(-y - 1, y, 2)}
\end{verbatim}

\texttt{solveset} is under active development as a planned replacement for
\texttt{solve}. There are certain features which are implemented in
\texttt{solve} that are not yet implemented in \texttt{solveset}. Notably,
these include nonlinear multivariate system and transcendental equations.
Below are examples of \textbf{solve} applied to these two cases.

\noindent Nonlinear (multivariate) system of equations (the intersection of a circle
and a parabola):
\begin{verbatim}
>>> solve([x**2 + y**2 - 16, 4*x - y**2 + 6], x, y)
[(-2 + sqrt(14), -sqrt(-2 + 4*sqrt(14))),
 (-2 + sqrt(14), sqrt(-2 + 4*sqrt(14))),
 (-sqrt(14) - 2, -I*sqrt(2 + 4*sqrt(14))),
 (-sqrt(14) - 2, I*sqrt(2 + 4*sqrt(14)))]
\end{verbatim}

\noindent Transcendental equations:
\begin{verbatim}
>>> solve((x + log(x))**2 - 5*(x + log(x)) + 6, x)
[LambertW(exp(2)), LambertW(exp(3))]
>>> solve(x**3 + exp(x))
[-3*LambertW((-1)**(2/3)/3)]
\end{verbatim}
