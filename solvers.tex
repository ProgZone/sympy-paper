%% Solvers in SymPy


SymPy has a solvers module to solve equations symbolically. There are two
submodules to solve algebraic equations in SymPy, referred to as old solve
i.e. \texttt{solve} and new solve i.e. \texttt{solveset}. Solveset is
introduced with several design changes with respect to old \texttt{solve} to
resolve the issues with old \texttt{solve}, for example old \texttt{solve}'s
input API has many flags which are not needed and they make it hard for the
user and the developers to work on solvers. In contrast to old solve, the
\texttt{solveset} has a clean input API, It only asks for the much needed
information from the user, following are the function signatures of old and new
solve:

\begin{verbatim}
solve(f, *symbols, **flags)  # old solve
solveset(f, symbol, domain)  # new solve
\end{verbatim}

The old \texttt{solve} has an inconsistent output API for various types of
inputs, whereas the \texttt{solveset} has a canonical output API which is
achieved using sets. It can consistently return various types of solutions.

\begin{itemize}
\item Single solution
\end{itemize}
\begin{verbatim}
>>> solveset(x - 1)
>>> {1}
\end{verbatim}

\begin{itemize}
\item Finite set of solution, quadratic equation
\end{itemize}
\begin{verbatim}
>>> solveset(x**2 - pi**2, x)
{-pi, pi}
\end{verbatim}

\begin{itemize}
\item No Solution
\end{itemize}
\begin{verbatim}
>>> solveset(1, x)
EmptySet()
\end{verbatim}

\begin{itemize}
\item Interval of solution
\end{itemize}
\begin{verbatim}
>>> solveset(x**2 - 3 > 0, x, domain=S.Reals)
(-oo, -sqrt(3)) U (sqrt(3), oo)
\end{verbatim}

\begin{itemize}
\item Infinitely many solution
\end{itemize}
\begin{verbatim}
>>> solveset(sin(x) - 1, x, domain=S.Reals)
ImageSet(Lambda(_n, 2*_n*pi + pi/2), Integers())
>>> solveset(x - x, x, domain=S.Reals)
(-oo, oo)
>>> solveset(x - x, x, domain=S.Complexes)
S.Complexes
\end{verbatim}
