SymPy has a module of equation solvers for symbolic equations. There are two
functions for solveing algebraic equations in SymPy. \texttt{solve}, which has
existed in SymPy for many years, and \texttt{solveset}, which is new in SymPy 1.0.
\texttt{solveset} has several design changes with respect to the old
\texttt{solve} function to resolve some of the issues with old \texttt{solve}
function. For example, the input API of \texttt{solve} has many flags, which
complicate it for both users and developers. In contrast, \texttt{solveset} has a
cleaner input API\@:  it only asks for the necessary information from the user.
The function signatures of \texttt{solve} and \texttt{solveset} are
\begin{verbatim}
solve(f, *symbols, **flags)
solveset(f, symbol, domain=S.Complexes)
\end{verbatim}
The \texttt{domain} parameter is typically either \texttt{S.Complexes} (the
default) or \texttt{S.Reals}, which causes it to only return real solutions.

Additionally, \texttt{solve} has an inconsistent output API for various types
of inputs. For instance, depending on the input, sometimes it returns a Python
list and sometimes it returns a Python dictionary. On the other hand, the
\texttt{solveset} has a canonical output API.\ \texttt{solveset} always returns
a SymPy set object.

Both functions implicitly assume that expressions are equal to 0. For
instance, \texttt{solveset(x - 1, x)} solves $x - 1 = 0$ for $x$.

\noindent Single solution:
\begin{verbatim}
>>> solveset(x - 1, x)
{1}
\end{verbatim}

\noindent Finite set of solution, quadratic equation:
\begin{verbatim}
>>> solveset(x**2 - pi**2, x)
{-pi, pi}
\end{verbatim}

\noindent No solution:
\begin{verbatim}
>>> solveset(1, x)
EmptySet()
\end{verbatim}

\noindent Interval solution:
\begin{verbatim}
>>> solveset(x**2 - 3 > 0, x, domain=S.Reals)
(-oo, -sqrt(3)) U (sqrt(3), oo)
\end{verbatim}

\noindent Infinitely many solutions:
% SymPy 1.0 sstr() prints S.Complexes incorrectly
% no-doctest
\begin{verbatim}
>>> solveset(sin(x) - 1, x, domain=S.Reals)
ImageSet(Lambda(_n, 2*_n*pi + pi/2), Integers())
>>> solveset(x - x, x, domain=S.Reals)
(-oo, oo)
>>> solveset(x - x, x, domain=S.Complexes)
S.Complexes
\end{verbatim}

Linear systems are solved with \texttt{linsolve}. Finite and infinite solution for determined, under
determined, and over determined problems are supported.
\begin{verbatim}
>>> A = Matrix([[1, 2, 3], [4, 5, 6], [7, 8, 10]])
>>> b = Matrix([3, 6, 9])
>>> linsolve((A, b), x, y, z)
{(-1, 2, 0)}
>>> linsolve(Matrix(([1, 1, 1, 1], [1, 1, 2, 3])), (x, y, z))
{(-y - 1, y, 2)}
\end{verbatim}

The new solve i.e. \textbf{solveset} is under active development and is a
planned replacement for \textbf{solve}, Hence there are some features which are
implemented in solve and is not yet implemented in solveset. The table below
show the current state of old and new solve functions.

\hfill

\begin{tabular}{ |p{4cm}|p{3cm}|p{3cm}|  }
\hline
\multicolumn{3}{|c|}{Solveset vs Solve} \\
\hline
Feature& solve &solveset \\
\hline
Consistent Output API & No & Yes \\
Consistent Input API & No & Yes \\
Univariate & Yes & Yes\\
Linear System& Yes & Yes (linsolve) \\
Non Linear System& Yes & Not yet \\
Transcendental& Yes & Not yet \\
\hline
\end{tabular}

\hfill \break{}

Below are some of the examples of \textbf{solve}, which are not yet supported
by \texttt{solveset}.

\noindent Nonlinear (multivariate) system of equations (the intersection of a circle
and a parabola):
\begin{verbatim}
>>> solve([x**2 + y**2 - 16, 4*x - y**2 + 6], x, y)
[(-2 + sqrt(14), -sqrt(-2 + 4*sqrt(14))),
 (-2 + sqrt(14), sqrt(-2 + 4*sqrt(14))),
 (-sqrt(14) - 2, -I*sqrt(2 + 4*sqrt(14))),
 (-sqrt(14) - 2, I*sqrt(2 + 4*sqrt(14)))]
\end{verbatim}

\noindent Transcendental equations:
\begin{verbatim}
>>> solve((x + log(x))**2 - 5*(x + log(x)) + 6, x)
[LambertW(exp(2)), LambertW(exp(3))]
>>> solve(x**3 + exp(x))
[-3*LambertW((-1)**(2/3)/3)]
\end{verbatim}
