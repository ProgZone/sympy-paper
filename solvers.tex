%% Solvers in SymPy


SymPy has module of equation solvers for symbolic equations. There are two
submodules to solve algebraic equations in SymPy, referred to as old solve
function, \texttt{solve}, and new solve function, \texttt{solveset}.
Solveset is introduced with several design changes with respect to old
\texttt{solve} function to resolve the issues with old \texttt{solve} function,
for example old \texttt{solve} function's input API has many flags which are
not needed and they make it hard for the user and the developers to work on
solvers. In contrast to old solve function, the \texttt{solveset} has a clean
input API, It only asks for the much needed information from the user, following
are the function signatures of old and new solve function:

\begin{verbatim}
solve(f, *symbols, **flags)  # old solve function
solveset(f, symbol, domain)  # new solve function
\end{verbatim}

The old \texttt{solve} function has an inconsistent output API for various types
of inputs, whereas the \texttt{solveset} has a canonical output API which is
achieved using sets. It can consistently return various types of solutions.

\begin{itemize}
\item Single solution
\end{itemize}
\begin{verbatim}
>>> solveset(x - 1)
>>> {1}
\end{verbatim}

\begin{itemize}
\item Finite set of solution, quadratic equation
\end{itemize}
\begin{verbatim}
>>> solveset(x**2 - pi**2, x)
{-pi, pi}
\end{verbatim}

\begin{itemize}
\item No Solution
\end{itemize}
\begin{verbatim}
>>> solveset(1, x)
EmptySet()
\end{verbatim}

\begin{itemize}
\item Interval of solution
\end{itemize}
\begin{verbatim}
>>> solveset(x**2 - 3 > 0, x, domain=S.Reals)
(-oo, -sqrt(3)) U (sqrt(3), oo)
\end{verbatim}

\begin{itemize}
\item Infinitely many solutions
\end{itemize}
\begin{verbatim}
>>> solveset(sin(x) - 1, x, domain=S.Reals)
ImageSet(Lambda(_n, 2*_n*pi + pi/2), Integers())
>>> solveset(x - x, x, domain=S.Reals)
(-oo, oo)
>>> solveset(x - x, x, domain=S.Complexes)
S.Complexes
\end{verbatim}

\begin{itemize}
\item Linear system: finite and infinite solution for determined, under
determined and over determined problems.
\end{itemize}
\begin{verbatim}
>>> A = Matrix([[1, 2, 3], [4, 5, 6], [7, 8, 10]])
>>> b = Matrix([3, 6, 9])
>>> linsolve((A, b), x, y, z)
{(−1,2,0)}
>>> linsolve(Matrix(([1, 1, 1, 1], [1, 1, 2, 3])), (x, y, z))
{(-y - 1, y, 2)}
\end{verbatim}

The new solve i.e. \textbf{solveset} is under active development and is a
planned replacement for \textbf{solve}, Hence there are some features which are
implemented in solve and is not yet implemented in solveset. The table below
show the current state of old and new solve functions.

\hfill

\begin{tabular}{ |p{4cm}|p{3cm}|p{3cm}|  }
\hline
\multicolumn{3}{|c|}{Solveset vs Solve} \\
\hline
Feature& solve &solveset \\
\hline
Consistent Output API & No & Yes \\
Consistent Input API & No & Yes \\
Univariate & Yes & Yes\\
Linear System& Yes & Yes (linsolve) \\
Non Linear System& Yes & Not yet \\
Transcendental& Yes & Not yet \\
\hline
\end{tabular}

\hfill \break

Below are some of the examples of old \textbf{solve} function:

\begin{itemize}
\item Non Linear (multivariate) System of Equation: Intersection of a circle
and a parabola.
\end{itemize}
\begin{verbatim}
>>> solve([x**2 + y**2 - 16, 4*x - y**2 + 6], x, y)
[(-2 + sqrt(14), -sqrt(-2 + 4*sqrt(14))),
 (-2 + sqrt(14), sqrt(-2 + 4*sqrt(14))),
 (-sqrt(14) - 2, -I*sqrt(2 + 4*sqrt(14))),
 (-sqrt(14) - 2, I*sqrt(2 + 4*sqrt(14)))]
\end{verbatim}

\begin{itemize}
\item Transcendental Equation
\end{itemize}
\begin{verbatim}
>>> solve(x + log(x))**2 - 5*(x + log(x)) + 6, x)
[LambertW(exp(2)), LambertW(exp(3))]
>>> solve(x**3 + exp(x))
[-3*LambertW((-1)**(2/3)/3)]
\end{verbatim}
